\documentstyle[12pt]{article}
\renewcommand{\baselinestretch}{1.2}
\begin{document}
\centerline{\bf Statistical computing: Homework 2}

\vspace{2ex}

This second homework is due Monday, Jan 26th.  All these should be
programmed using a functional programming style and written up carefully.

\begin{enumerate}
\item Code up the bisection algorithm for finding $f(x) = 0$.  It
should accept a function and a range and find where the function
crosses zero in that range.  If the function evaluated at the
end-points both have the same sign, then the algorithm should call
{\tt stop(``Function might not have a zero.'')}.
\begin{enumerate}
\item Test your root finder with $f_1(x) = x^3 - 8$, and $f_2(x) =
10^{-30}x - 1$ and $f_3(x) = \cos(x) + x$, and $f_4(x) = \cos(x) + 2$
and $f_5(x) = \sin(x)$ for $x$ in $[3,4]$.  Finally, try your function
on $I(x>1) - .5$ for the interval $[0,2]$.  Does it do the right thing?
\item Write a function what will accept as input one function $f$ and will
find a value $M$ such that $f(-M)f(M) \le 0$.  In other words, if $f$
is continuous there will be a zero somewhere in the interval $[-M,M]$.
Try to come up with an input function for force $M$ to be very large.
\item Write a function called {\tt root} that will accept as input a
function and return a zero of that function or an error condition.
\item Write a function called {\tt solve} that will accept as input a
function $f$ and a value $y$ and will return an $x$ such that $f(x) = y$.
\item Test your two new functions.  In writing your test functions try
to come up with ones that will break it.  Special conditions are
always good.  Extreme cases.  Illegal inputs etc.
\end{enumerate}

\item The following Splus defintion will compute numerical
derivatives (recall $f' \approx \frac{f(x+h) - f(x)}{h}$):
 
{\tt d <- function(f) \\
\{ substitute( function(x) (f(x+.001)-f(x))/.001 ) \}
}

The above function will take a function $f$ and return a numerical
approximation to $f'$.
\begin{enumerate} 
\item Compute the derivative of the $\sin()$ function and plot it.
\item Try this again with values of ``$h$'' set at $10^3$, $10^{-0}$,
$10^{-3}$, $10^{-6}$, $10^{-9}$, $10^{-12}$, $10^{-15}$, $10^{-18}$.
Which value seems to work best?  
\item Compute the 4th derivative of the $\sin()$ function and plot
it for the various values of $h$.  What seems optimal now?
\item Combine this derivative function with your root finder from the
previous problem to find the arg-max of $\sin(x)$.
\item Write a function {\tt arg-max()} that will take as input a
single function and find the location of the maximum point.
\item Write a function {\tt max()} that will compute maximum of a
function that is passed in.
\item Test your function on various cases.
\item Try your function on $f(x) = x^2$.  What does it do?  
\end{enumerate}
\item Consider a random walk that takes steps forward of size $f$ and
backwards of size $b$ with equal probability.  In other words,
$P(X_{t+1} = X_t + f)$ = $P(X_{t+1} = X_t -b$ = $1/2$.  We are
interested in computing $E(g(X_t)|X_0=x_0)$.  Write an Splus function
called {\tt E} that accepts a function $g$, a time $t$, a starting
position $x_0$, and the forward and backward steps, $f$ and $b$ and
returns the expected value of $E(g(X_t)|X_0=x_0)$.  You should use
recursion and functional programming.  Try your function with $f =
1.1$, $b = -1$, $t = 5$, $x_0 = 100$ and $g(x) = \max(x-100,0)$.  

If you think of $X_t$ as the stock price at time $t$ then you have
just computed the value of a ``at the money'' call option (or is it a
put option?).  Rerun it for various values of $x_0$ and make a plot
the expectation vs the starting point.  Comments?
\end{enumerate}
\end{document}

