\documentstyle[12pt]{article}
\renewcommand{\baselinestretch}{1.4}
\begin{document}
\centerline{\bf Statistical computing}
\centerline{Homework 1 (revised)}

\vspace{2ex}

This is the first of many short homework.  It is due Jan 19th.  

{\bf Grading policy:} My goal is to make sure you understand the
concepts from each week as we go along.  So it is important to do it
immediately.  To encourage you, I will take off 2/10 for each day
late.  So turn it in by the following Friday for any positive credit.
Even if you don't finish it by ``zero points day,'' you will still
need to finish it to complete the course.

{\bf Homework writeups:} Expect to spend about 1/2 of your time
programming and 1/2 of your time writing up what you did.  You should
of course write up your results using \LaTeX.  Edit your code into
your latex file and then use the $\backslash$tt command to make it look
like typewriter output.  (This is the usual convention for presenting
code.)  Your write up should be detailed enough such that if I were to
give it to another student he/she could reproduce your results
exactly.  (Including any mistakes you might have made!)

\begin{enumerate}
\item Code the two functions that compute the absolute value in
class.  Call them {\tt absoSq} and {\tt absoIf}.  (Recall the first
computed the square and then the squareroot.  The second tested if $x$
was greater than 0.)  Check the function on various values and see if
they both work.
\begin{enumerate}
\item If you only checked the values, 3, 4.674 and 10000, do you have
confidence that your functions work?
\item Print out the help for the {\tt unix.time} command (staple it to
the end of your homework.)  Also read the help on {\tt lapply} and
{\tt rnorm}.
\item Run the following command:

\centerline{\tt unix.time(lapply(rnorm(10000),absoSq))}
and

\centerline{\tt unix.time(lapply(rnorm(10000),absoIf))}

What do the above commands do?  Which runs faster, absoSq or absoIf?
Why do you think this is the case?
\item Run {\tt absoSq(-2+3*1i)} and {\tt absoIf(-2+3*1i)}.  What is
happening? 
\end{enumerate}

\vspace{2ex}

\item The Fibonacci are defined as $F(x) = F(x - 1) + F(x-2)$.  This
inductive definition requires that we specify the values for $x=0$ and
$x=1$.  Let them be zero and one respectively.
\begin{enumerate}
\item Program up the Fibonacci numbers.
\item Find the running time for $x=1,2,3,4,5,10,15,20,25,30,35,40,$ etc.
Plot the running time against $x$.  What shape does it follow?
\item Plot the log running time against $x$.  What is the
slope?  How does this relate to the shape in the previous part?
\item Estimate an equation for the running time as a function of $x$.
\end{enumerate}
\end{enumerate}

\end{document}

