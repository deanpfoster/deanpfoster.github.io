\documentstyle[12pt]{article}
\renewcommand{\baselinestretch}{1.2}
\begin{document}
\centerline{\bf Statistical computing: Homework 4}

\vspace{2ex}

This fifth homework is due Monday, Feb 23rd.  You should do the put
the write up on your homepage.  (Actually, you should put it on a
separate page and provide a pointer to it on your homepage.)  Please
turn the name of your homepage on Feb 23rd.

Your goal in this homework is to provide robust functions.  There is
only one new function to write.  But, you will probably want to
rewrite most of your earlier programs to make them be more robust.

The way I will grade it is by cutting and pasting the code from your
homepage to Splus and running it.  You will lose points, if I give it
input that it generates an incorrect answer for.  You won't lose as
much (or possible any at all) if you told me in your documentation
that it wouldn't in fact work.  

I started writing a home page a week ago.  It has a few pointers that
you might find useful if you haven't every written any html.  Point
your broswer at ``www-stat.wharton.upenn.edu/~foster''.

\begin{enumerate}
\item Add the following functions to your homepage:
\begin{enumerate}
\item {\tt d()}, the derivative operator
\item {\tt root(f,lower,upper)} find a root in the interval lower to upper
\item {\tt root(f)}, a root finder 
\item {\tt arg.max(f)} and {\tt max(f)}
\item A definition of a density function
\item {\tt norm(mu,sigma), uniform(theta), exponential(lambda)}, some
density generators
\item {\tt CDF} which accepts a density and returns the CDF
\item {\tt E(density,g)} which computes $E(g(X))$ and $X \sim {\tt
density}$.
\item {\tt MGF(density)}
\item {\tt transform(density,g)} (this is new) This returns the
density of $Y$ where $Y = g(X)$ and $X \sim {\tt density}$.
\end{enumerate}
You should tell me exactly what restrictions you want to place on
inputs.  If you only want to find roots for monotonic functions, tell
me that.  It is better the fewer restrictions you place on your
functions.  Provide documentation for each function similar to the
documentation that Splus provides.  

\item Make a separate page off of your homepage that has testing
functions.  Your goal is to find a test that breaks other peoples
functions but not your own.  So, you might want to look at their pages
as they are being developed!

\end{enumerate}
\end{document}

