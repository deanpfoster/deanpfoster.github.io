\documentstyle[12pt]{article}
\renewcommand{\baselinestretch}{1.2}
\begin{document}
\centerline{\bf Statistical computing: Homework 3}

\vspace{2ex}

This third homework is due Monday, Feb 2th.  As usual, these should be
programmed using a functional programming style and written up
carefully.  Feel free to ``reuse'' your code from last week.  If you
don't end up reusing your code, say what was wrong with it that
required you to rewrite it and how you might avoid this sort of
problem in the future.

\begin{enumerate}
\item Code up the density concepts from last week.  Do densities,
and expectations as we did in class.  
\begin{enumerate}

\item Write functions that will generate normal($\mu,\sigma$),
uniform(0,$\theta$), and exponential($\mu$).  Write a ``wrapper''
function that will convert the built in {\tt pchisq} into the format
we used in class.  

\item Add a default of $\mu = 0$ and $\sigma =1$ to your {\tt normal}
function.  So {\tt normal(3,4)}, has a mean of 3 and a SD of 4, and
{\tt normal(3)} has a mean of 3 and a SD of 1, and {\tt norm()} has a
mean of zero and a SD of 1.

\item Write and test a function, {\tt CDF}, that when given a
density will generate a CDF.  

\item Write and test a function, {\tt MGF}, that will take a density and return
the moment generating function.

\item Use the derivative function from last week and the {\tt MGF}
from this week to find a the first few moments of a non-central
chisquared distrubution.  Check them by using the {\tt E} operator you
wrote.  Which computes the 10th moment quicker?
\end{enumerate}
\newpage
\item Write a fixed point function.  In other words, a function that
will take another function and a value and then keep applying the
function until a fixed point results.  
\begin{enumerate}
\item Newton's method for finding a root of an equation is usually
written as: 
$$x_n = x_{n-1} - \frac{f(x_{n-1}}{f'(x_{n-1})}$$ 
Code up {\tt Newton}'s method using your fixed point algorithm above.
\item Create a function {\tt difficult} which has a root at zero, but
for which {\tt Newton(difficult,1)} doesn't converge.
\item Check that your bisection algorithm converges for your function
{\tt difficult}.
\item Prove or provide a counter example: If Newton's method is
started close enough to a root then it will converge to the root.
\item Compute the squareroot of of the first $n$ numbers using both
bisection and Newton's method.  Which is faster?  By how much?
\end{enumerate}
\item Using the quadratic equation and a two term Taylor series you
should be able to write a root finder that is even faster than
Newton's method.  Program up such an algorithm.
\begin{enumerate}
\item Find a function it fails to find the root for.
\item Write a hybrid algorithm that will use bisection to get close to
a root, and then use your quadratic root finder to find the actual
root. 
\item  How does its speed compare to bisection and Newton's method?  
\end{enumerate}
\end{enumerate}
\end{document}
