\documentstyle[12pt]{article}
\renewcommand{\baselinestretch}{1.2}
\begin{document}
\centerline{\bf Statistical computing: Homework 4}

\vspace{2ex}

This fourth homework is due Wednesday, Feb 11th.  As usual, these
should be programmed using a functional programming style and written
up carefully.

In preparation for following week's homework, Check with the systems
administrator of your department and see what you have to do to create
a home page.  (In statistics that person is Barry Kirzner,
kirzner@stat.  You have to ask him for an account on enkido.)  

\begin{enumerate}
\item There are many possible modes objects can take on in Splus.
Your goal in this problem is to return something of each of the
following modes (or say why it is impossible): 
\begin{enumerate}
\begin{enumerate}
\item  ``numeric'', 
\item ``logical'',
\item ``complex'',
\item  ``character'', 
\item ``null''
\item ``list'', 
\item  ``function'', 
\item ``expression'',
\item  ``comment.expression'',
\item ``name'',
\item  ``call'',
\item  ``$<-$''.
\end{enumerate}
\end{enumerate}
This really is a question of creativity.  In other words, I want you
to come up with a {\em useful} function that returns something of the
desired mode (if it is possible).  So for the first few, you only need
mention code you wrote in previous exercises.  But for some of the
rest, you will have to work at thinking of a reason for doing it!

\item Write up a function {\tt MLE} that takes a density generating
function and a vector and finds the value maximum likelihood
estimator.  Test it on the {\tt car.gals} data using both a normal and
an exponential distribution.  Does it do the right thing?

\item Suppose you have a function {\tt sin} and a grid of points.  You
can evaluate the {\tt sin} on the grid of points and then run a
polynomial fit to the result.  This will generate a polynomial fit to
the {\tt sin} function.  We will think of the resulting polynomial as
a function, {\em not} a set of coefficients.
\begin{enumerate}
\item Write a function {\tt fit(f,lower,upper,k)} which takes an
input function {\tt f} and fits a polynomial of degree {\tt k} to it
over the interval {\tt lower} to {\tt upper}.  Of course it should
return a function which represents the fit.  Test it on various sample
functions.  Find situations where it works well and situations where
it fails.  (Show at least one graph each of fitting well and failing
to fit well.)
\item Write a different function {\tt d.fit(f,lower,upper,k)} which
takes an input function {\tt f} and fits a polynomial of degree {\tt
k} to it over the interval {\tt lower} to {\tt upper}.  It should then
return the derivative of this polynomial as a function.  Test it on
some examples.
\item Why is the above better than merely doing {\tt
d(fit(f,lower,upper,k))}?
\item You now have two ways of computing a derivative, you can use
{\tt d} which you defined in previous classes and {\tt d.fit}.  To
contrast {\tt fit(d(f),lower,upper,k-1)} which returns a polynomial fit
to the derivative and {\tt d.fit(f,lower,upper,k)} which also returns
a polynomial which is the derivative of the fit.  Are these always the
same?  If so prove it, if not, find an example where they differ.  If
they differ, which is more accurate?  Why?
\end{enumerate}
\end{enumerate}
\end{document}

