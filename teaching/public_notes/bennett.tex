\documentclass{article}

\begin{document}

Bennett's bound (1962) is usually stated for a bounded collection of
 $n$ independent random variables $U_1, \ldots, U_n$ with $\sup |U_i|
 < M$, $E\,U_i=0$, and $\sum_i E\,U_i^2 = 1$, and $\tau >0$
\begin{displaymath}
  P(\sum_i U_i \ge \tau) 
   \le  \exp\left( 
      \frac{\tau}{M} -
      \left(\frac{\tau}{M}+\frac{1}{M^2}\right)\log(1+M\tau)
    \right).
\end{displaymath}
We will rewrite it and narrow its focus to $n$ IID random variables
 $X_1,\ldots,X_n$, which are bounded by 1, with Var$(X_i) =
\sigma^2$.   Then
\begin{displaymath}
  P(\overline{X} - EX \ge \gamma) 
   \le  \exp\left(n\gamma -
        n(\gamma + \sigma^2)\log(1+\gamma/\sigma^2)\right) 
\end{displaymath}
Writing it differently:
\begin{displaymath}
  P(\overline{X} - EX \ge k\sigma^2) 
   \le  \exp\left(n \sigma^2(k - (k+1)\log(k+1))
    \right) 
\end{displaymath}
Or in its most traditional form, if $x \ll \sigma\sqrt{n}$, 
expanding the log
\begin{displaymath}
  P\left(\frac{\overline{X} - EX}{\sigma/\sqrt{n}}  \ge x\right)
   \le  \exp\left(-x^2/2 + x^3/(6\sigma\sqrt{n}) + O(x^4/n\sigma^2))
    \right) 
\end{displaymath}
Or even more crudely $x < .3 \sigma\sqrt{n}$, then
\begin{displaymath}
  P\left(\frac{\overline{X} - EX}{\sigma/\sqrt{n}}  \ge x\right)
   \le  \exp\left(-x^2/2(1 - x/\sigma\sqrt{n})\right)
\end{displaymath}


\begin{itemize}
\item Bennett, G. (1962), ``Probability inequalities for the sum of
  independent random variables,'' {\it JASA}, {\bf 57}, 33--45.
\end{itemize}

\end{document}
