\documentclass[letter,12pt]{extarticle}
\renewcommand{\baselinestretch}{1.3}
\usepackage{simplemargins}
\setallmargins{.25in}
\setrightmargin{.5in}
\settopmargin{.5in}
\pagestyle{empty}
\usepackage{color}
\definecolor{mypurple}{rgb}{.3,0,.5}
\newcommand{\gap}{\vspace{1em}}


\newcommand{\answer}[1]{\noindent{\textcolor{mypurple}{\scriptsize{\{\bf answer:} \em #1\}}}}
\renewcommand{\answer}[1]{}

% Redefine gap to space out exam
% \renewcommand{\gap}{\vspace{2.em}}


\begin{document}
\centerline{\bf Probability Final} 

You may use your sheet from the midterm and a new sheet of notes.  No
 calculators, cell phones, PDA's, laptops, or HAL 9000's.  Show your
 reasoning.  Don't just give the answer.

\gap
\begin{enumerate}

\item  A box contains two gold balls and two clay balls.  You
are allowed to choose successively balls from the box at random.  You win 1~dollar
each time you draw a gold ball and lose 1~dollar each time you draw a clay ball. 
After a draw, the ball is not replaced.  
\begin{enumerate}
\item If you draw exactly one ball, what is your expected
earnings?\answer{$2/4 - 2/4 = 0$.}
\item What is the moment generating function for the value of the
first draw?\answer{$(1/2)e^t + (1/2) e^{-t}$}
\item If you draw exactly $k$ balls (for $k = 1,2,3,4$) what is your
expected earnings? \answer{$0$}
\item If you draw until you are ahead by 1~dollar or until there are
 no more gold balls, what is your expected earnings?\answer{GGcc = 1,
GcGc = 1, GccG = 1, cGGc = 1, cGcG = 0, ccGG = 0.  Grand total is 4, each equally
likely, so 4/6, or 2/3.}
\end{enumerate}
\gap

\item Suppose you win 1 dollars when an black card is drawn from a deck of
cards but you lose 1 dollar when a red card is drawn.
(So out of the 52 cars, you win with 26 of them and lose with 26 of
them.)
\begin{enumerate}
\item Let $X_1$ be the amount you win on the first draw, and $X_2$ be the
amount you win on the second draw.  (Assume you don't put the card
back.) What is $Cov(X_1,X_2)$?

\answer{$(26/52)(25/51)-(26/52)(26/51)-(26/52)(26/51)+(26/52)(25/51)$
$ = (1/2)(1/52)(25-26-26+25)$
$ = (1/2)(1/51)(-2) = 1/51$ }

\item What is the mean and variance of $X_1 + X_2$?\answer{.25 - 2/51}
\item What is the mean and variance of $\sum_{i=1}^{52} X_i$? (Hint:
think before you compute.)\answer{0}
\end{enumerate}
\gap


\item Consider a non-negative random varible: $X \ge 0$.  
\begin{enumerate}
\item If $E(X) = 1$, what is a good bound on $P(X \ge 100)$?
\item If $E(X) = 1$, and $V(X) = 1$ what is a good bound for $P(X \ge 100)$?
\item If the generating function $h_X(2) = 4$, (I.e. $E(2^X) = 4$ then what is a
good bound for $P(X \ge 100)$?
\end{enumerate}
\gap


\item Suppose the moment generating function for $X$ is $g(t) = 1 +
 t$.  In other words, $E(e^{tX}) = e^t$.  What can you tell me about
 $X$? \answer{$E(X) = 1$, $E(X^2) = 1$, so $var(X) = 0$ so $P(X=1)=1$.}


\item The law of large numbers tells us alot about a sum of random
 variables.  The CLT tells us even more about sums.  But what about
 products?  Let $X_i$ be a random variable that takes on either $+1$
or $-1$ with equal probability.  Let $P_n = \prod_{i=1}^n X_i$.  Will
$P_n$ converge to some fixed value?  (I.e. law of large numbers?)  If
it converges, what is this value, if it doesn't converge, what does
$P_n$ look like?\answer{Does not converge.  $P(P_n = 1) = P(P_n = -1)
= .5$.}

\gap
\item Statistics is often driven by two things, a prediction and a
residual.  Define the random variable $Z = E(Y|X)$ and the random
variable $W = Y - Z$.  Then $Z$ is the prediction and $W$ is the
residual of the ``regression'' of $Y$ on $X$.  
\begin{enumerate}
\item What is $E(Z)$?\answer{$E(Z) = E(E(Y|X)) = E(Y)$}
\item What is $E(XZ)$\answer{$E(XZ) = E(XE(Y|X)) = E(E(XY|X)) = E(XY)$}
\item Let $h()$ be an arbitary function, show $E(h(X)Z) = E(h(X)Y)$. \answer{$E(g(X)Z) = E(g(X)E(Y|X)) = E(E(g(X)Y|X)) = E(g(X)Y)$}
\item What is $E(WZ)$?\answer{$E(WZ) = E(E(WZ|X)) = E(ZE(W|X)) =
E(ZE(Y - Z|X)) = E(ZE(Y|X) - ZE(Z|X)) = E(Z^2 - Z^2) = 0$}
\end{enumerate}
\gap

\item Let $X_i$ be a random variable with mean $1.01$ and standard
deviation $.2$.  (For example, $X = 1.21$ or $X = .81$ with equal
probability, but that is such an ugly statement, lets pretend I didn't
mention it.)  Let $W = \prod_{i=1}^n X_i$.  Suppose all the $X_i$'s
are independent, so the whole series is IID.
\begin{enumerate}
\item What is $E(W)$?\answer{$E(W) = 1.01^n$}
\item What is the long run growth rate (i.e. $\lim_{n \to \infty}
(\log W_n)/n$)?\answer{$\log W_n)/n \approx E(\log(X)) \approx E(X-1) - V(X)/2
= .01 - .2^2/2 = -.01$}
\item What will $W_{1000}$ look like?\answer{$.99^{1000}=e^{-.01*1000}
= e^{-10} \approx .0001$}
\end{enumerate}
\gap

\item Suppose you put 100 mice on a calorie restriction diet.
Normal mice on a normal diet live 1000 days with a standard deviation
of 150 days.   
\begin{enumerate}
\item If this diet doesn't change the length of life for these mice,
what will be the mean, variance and distribution of $\overline{T}$?
(Where $\overline{T}$ is the average number of days a mouse in the
experiment lives.)\answer{$E(\overline{T}) = 1000$, $V(\overline{T}) =
(15)^2$, it is normal.}
\item Find a good estimate the probability that $\overline{T}$ is
bigger than 1300.\answer{$(\overline{T}-1000)/15 = 20$.  So via
Chebeshev, the probability is less than $1/20^2$, or 1/400.}
\item If your experiment actually yielded an average of 1300, would
you believe that these mice have the same mean as typical
mice?\answer{Nope!  Low feed mice live longer.}
\item (bonus) Using generating functions, provide a better
bound.\answer{If you have a really big table for normals, it might go
out to twenty, and so you know the answer is about 1/googol. (I
mentioned this fact in class--but this is just an asside.)  Here are
your steps: (1) compute the moment generating function to be $g(t) =
e^{t^2/2}$. (2) Now use Markov for $P(e^{tZ} > e^{t20}) <
e^{t^2/2}/e^{t 20}$. (3) Now optimize this over $t$ to see that $t=20$
gets the best bound. (4) the bound is now $e^{-200}$.}
\end{enumerate}

\end{enumerate}
\end{document}
\newpage
\renewcommand{\baselinestretch}{1.0}
\small\normalsize
\centerline{\bf CLASS COMMENTS}

You will be asked by a computer to evaluate the class.  But since
there were a few new things I tried that were new, I wanted to get
your reactions to them.  If you are feeling time stressed--feel free
to wait until after the exam to fill this out, or drop it by my office
at some later time.  (Use the back for more space if you want.)

\gap
{\bf Stories:} I tried to start most classes with an unrelated
 factoid / story.  The educational concept is that these are easier to
 remimber.  So my challenge is three fold: Can you remimber any of
 them?  Is there one that stuck with you?  Did you like them?


\vspace{1.5in}

{\bf Book:} We used an open source book this semester.  I liked the
 content, but I'm not the target audience.
\begin{itemize}
\item Which format did you use the book in:
\begin{itemize}
\item Only pdf
\item only Book
\item Both hard copy and pdf
\end{itemize}
\item Did you like the book?  
\vspace{2em}
\item Was it too hard / too easy?
\vspace{2em}
\item Which chapter did you like the best?
\vspace{2em}
\item Which one the least?
\vspace{2em}
\end{itemize}

{\bf Other:} If you like, feel free to comment on anything
 else in the class (use the other side).  These comments will only go
 to me.  If you want to make your comment public to future
 students--then you have to use the on line system.
\end{document}
