\documentclass[14pt]{extarticle}
\usepackage[margin=0.5in]{geometry}

\renewcommand{\baselinestretch}{1.3}
\pagestyle{empty}
\begin{document}
Dean Foster \hfill {\bf Stat 430: First Midterm Exam} \hfill\today
\vspace{1em}

\begin{enumerate}
\item (10 pts) This problem has you show deMorgan's law:
\begin{itemize}
\item Draw a Venn diagram of $(A \cup B)^c$.
\item Draw a Venn diagram of $A^c \cap B^c$.
\end{itemize}
Are they the same sets?

\vfill

\item (10 points) Take $\Omega = \{1,2,3,4\}$, and suppose we know that $\{1,3\}$ and
$\{3,4\}$ are both in the algebra $\cal A$.  Show that $\{1,4\}$ is in $\cal A$.

\vfill

\item (10 points) Suppose $\Omega = \{1,2,3\}$, Give an example of an
algebra of sets $\cal A$ over $\Omega$ such that $\{1\} \not\in {\cal A}$.

\vfill

\item (10 points) Suppose $\Omega = \{1,2\}$, then clearly it is
impossible to get the set $\{3\}$.  But, argue why it doesn't make
sense to say $P(\{3\}) = 0$.

\vfill

\item (20 points) Suppose Tom either takes a taxi or the bus.  To save
money, he only takes the taxi 1/10 of the time.  When he takes a taxi,
he is late 20\% of the time, but when he takes the bus he is late 80\%
of the time.  One morning, Tom, Dick and Harry are scheduled to have a
meeting.  When, Tom arrives on time Dick says, ``I see you sprung for
a cab this morning!''  But before Tom can reply, Harry jumps in with
``Nah, he's too cheap!  I bet he was just lucky!''  
\begin{enumerate}
\item What is the probability that Tom took a taxi?
\item What amounts should Dick and Harry bet to make this fair?
\item Who is more likely to be right?
\end{enumerate}
\newpage
\item (20 points) Suppose the generating function for the non-negative
 integer valued random variable $X$ is $E(s^X) = .5 + .2s + .3 s^2$.
\begin{enumerate}
\item What is $P(X = 1)$?
\item What is $E(X)$?
\item Suppose $Y = \sum_{i=1}^n X_i$ where each $X_i$ is IID with the
same generating function.  What is the generating function for $Y$,
and from this find what $P(Y=0)$ is?
\end{enumerate}

\vfill

\item (15 points) Suppose we have an asset that returns a random amount each
year.  We will define $R_t$ to be the ratio of the price at the end
of year $t$ to the price at the beginning of the year.  So after $T$
years, the total value when you start with a single dollar is:
\begin{displaymath}
W_T = \prod_{t=1}^T R_t = R_1 \times R_2 \times R_3 \times \cdots
\times R_T
\end{displaymath}
Assume that the returns are independent and identically distributed. 
If $E(R_t) = 1.7$ what is the formula for $E(W_T)$?  What will
$E(W_{20})$ equal?

\vfill

\item (5 points) Let $X_i$ be a Bernulli trial, namely, $P(X_i = 1) =
 p = 1 - P(X_i = 0)$, where $\{X_i\}_{i=1,n}$ are an IID
 sequence.  Let $Y = \sum_{i=1}^n X_i$.  Then $Y$ is a Binomial with
 parameters $p$ and $n$, and hopefully you have written on your cheat
 sheet that $Var(Y) = np(1-p)$.  Let's check that you wrote this down
 correctly:
\begin{enumerate}
\item Warmup: What is the $var(X_i)$?
\item Trivia: What is the $cov(X_i,X_j)$ if $i \ne j$?
\item Real question: What is $var(\sum_{i=1}^n X_i)$?
\item Bonus: Starting from $P(Y = y) = {n \choose y}p^y(1-p)^{n-y}$
compute the variance of $Y$.
\end{enumerate}

\end{enumerate}
\end{document}


