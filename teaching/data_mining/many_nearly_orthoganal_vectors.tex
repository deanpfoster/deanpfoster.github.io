\documentclass{article}

\begin{document}
How many nearly orthoganal vectors can be cram into a $d$ dimensional
space? 

We will say that two vectors are nearly orthognal if the angle between
them is close to $\pi/2$.  In particular, if $|x\top y| \le \epsilon$
for $|x| = 1$ and $|y| = 1$.  We will use the probabilistic method for
our proof.

If $d$ is large enough, we can sample from the surface of the sphere
with radius 1 by taking each direction being an independnent normal
with mean zero and variance $1/d$.  We can then ask what is the
probability that two random normals are nearly orthoganal?
\begin{eqnarray*}
P(\frac{|X\top Y|}{|X||Y|} > \epsilon) & \approx & P(|X\top Y|>\epsilon)\\
& = & P(|X\top \delta_1| > \epsilon)\\
& = & P(|X_1| > \epsilon)\\
& = & P(|Z| > \sqrt{d}\epsilon)\\
& \approx & e^{-d\epsilon^2/2}
\end{eqnarray*}
If we have a total of $N$ random vectors shoved onto the surface of
our sphere, then we have a total of $N \choose 2$ angles we all want
to be close to 90 degrees.  So by the union bound, this will be likely
to happen if 
\begin{eqnarray*}
{N \choose 2} e^{-d\epsilon^2/2} & \approx & 1 \\
N^2& \approx &  e^{d\epsilon^2/2} \\
N& \approx &  e^{d\epsilon^2/4} \\
N& \approx &  (e^{\epsilon^2/4})^d
\end{eqnarray*}
So we want to know what value of $\epsilon$ has the property that
$e^{\epsilon^2/4} = 1.5$ to get the result I have claimed.  Clearly we
could plug in other values instead of 1.5 and get smaller epsilons.
Since the log of 1.5 is about .4, we have $\epsilon^2 \approx 1.6$ or
$\epsilon \approx 1.3$.  Ooops!  THis is a useless bound since we know
perfect correlation is 1.0. 

So we want to know what value of $\epsilon$ has the property that
 $e^{\epsilon^2/4} = 1.1$ which will still be exponential, but not as
 extreme as previously claimed.  Since the log of 1.1 is about .1, we
 have $\epsilon^2 \approx .4$ or $\epsilon \approx .6$.


\end{document}
