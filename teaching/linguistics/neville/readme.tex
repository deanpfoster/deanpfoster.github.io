\documentclass[12pt]{article}


\title{Corpora used in speech activity detection}
\author{Neville Ryant}
\date{}

\begin{document}
\maketitle

\section{Overview}
This data sets contains 10,000 instances from each of 19 speech domains. For each domain, 62 acoustic features were extracted from the signal every 10 ms and for each frame, a speech/nonspeech label assigned based on a human annotation. From the resulting frames, 10,000 were sampled without replacement according to a uniform distribution. 

\section{File Formats}
The data is provided in two formats:
    \begin{enumerate}
        \item {\bf data.txt.gz} \\
              This is a gzipped plaintext file. After unzipping, the resulting plaintext file can be read into R via
              \begin{quotation}
                load.table(`data.txt', header=TRUE, sep=` ');
              \end{quotation}
              which will provide a 190,000 row data frame whose first column lists the domain from which the frame was taken, whose second column is a binary variable indicating whether the frame corresponds to a speech segment, and whose remaining columns contain acoustic features.

         \item {\bf data.rda} \\
               This binary file contains the R dataframe encoded in `data.txt.' To load it into R as the dataframe object `sad.df' execute the following:
               \begin{quotation}
                   load(`data.rda');
               \end{quotation}
    \end{enumerate}



\section{Features}
The following 62 acoustic features are provided:
\begin{enumerate}
    \item zcr$\_$wl0.005s \\
      Zero-crossing rate of signal extracted using a 5 ms analysis window.

    \item zcr$\_$wl0.035s \\
      Zero-crossing rate of signal extracted using a 35 ms analysis window.

    \item le$\_$0-8000Hz$\_$wl0.005s \\
      Log-energy of signal in band from 0-8000 Hz calculated using a 5 ms analysis window.

    \item le$\_$0-4000Hz$\_$wl0.005s \\
      Log-energy of signal in band from 0-4000 Hz calculated using a 5 ms analysis window.

    \item le$\_$4000-8000Hz$\_$wl0.005s \\
      Log-energy of signal in band from 4000-8000 Hz calculated using a 5 ms analysis window.

    \item le$\_$0-2000Hz$\_$wl0.005s \\
      Log-energy of signal in band from 0-2000 Hz calculated using a 5 ms analysis window.

    \item le$\_$2000-4000Hz$\_$wl0.005s \\
      Log-energy of signal in band from 2000-4000 Hz calculated using a 5 ms analysis window.

    \item le$\_$4000-6000Hz$\_$wl0.005s \\
      Log-energy of signal in band from 4000-6000 Hz calculated using a 5 ms analysis window.

    \item le$\_$6000-8000Hz$\_$wl0.005s \\
      Log-energy of signal in band from 6000-8000 Hz calculated using a 5 ms analysis window.

    \item le$\_$0-1000Hz$\_$wl0.005s \\
      Log-energy of signal in band from 0-1000 Hz calculated using a 5 ms analysis window.

    \item le$\_$1000-2000Hz$\_$wl0.005s \\
      Log-energy of signal in band from 1000-2000 Hz calculated using a 5 ms analysis window.

    \item le$\_$2000-3000Hz$\_$wl0.005s \\
      Log-energy of signal in band from 2000-3000 Hz calculated using a 5 ms analysis window.

    \item le$\_$3000-4000Hz$\_$wl0.005s \\
      Log-energy of signal in band from 3000-4000 Hz calculated using a 5 ms analysis window.

    \item le$\_$4000-5000Hz$\_$wl0.005s \\
      Log-energy of signal in band from 4000-5000 Hz calculated using a 5 ms analysis window.

    \item le$\_$5000-6000Hz$\_$wl0.005s \\
      Log-energy of signal in band from 5000-6000 Hz calculated using a 5 ms analysis window.

    \item le$\_$6000-7000Hz$\_$wl0.005s \\
      Log-energy of signal in band from 6000-7000 Hz calculated using a 5 ms analysis window.

    \item le$\_$7000-8000Hz$\_$wl0.005s \\
      Log-energy of signal in band from 7000-8000 Hz calculated using a 5 ms analysis window.

    \item le$\_$0-8000Hz$\_$wl0.035s \\
      Log-energy of signal in band from 0-8000 Hz calculated using a 35 ms analysis window.

    \item le$\_$0-4000Hz$\_$wl0.035s \\
      Log-energy of signal in band from 0-4000 Hz calculated using a 35 ms analysis window.

    \item le$\_$4000-8000Hz$\_$wl0.035s \\
      Log-energy of signal in band from 4000-8000 Hz calculated using a 35 ms analysis window.

    \item le$\_$0-2000Hz$\_$wl0.035s \\
      Log-energy of signal in band from 0-2000 Hz calculated using a 35 ms analysis window.

    \item le$\_$2000-4000Hz$\_$wl0.035s \\
      Log-energy of signal in band from 2000-4000 Hz calculated using a 35 ms analysis window.

    \item le$\_$4000-6000Hz$\_$wl0.035s \\
      Log-energy of signal in band from 4000-6000 Hz calculated using a 35 ms analysis window.

    \item le$\_$6000-8000Hz$\_$wl0.035s \\
      Log-energy of signal in band from 6000-8000 Hz calculated using a 35 ms analysis window.

    \item le$\_$0-1000Hz$\_$wl0.035s \\
      Log-energy of signal in band from 0-1000 Hz calculated using a 35 ms analysis window.

    \item le$\_$1000-2000Hz$\_$wl0.035s \\
      Log-energy of signal in band from 1000-2000 Hz calculated using a 35 ms analysis window.

    \item le$\_$2000-3000Hz$\_$wl0.035s \\
      Log-energy of signal in band from 2000-3000 Hz calculated using a 35 ms analysis window.

    \item le$\_$3000-4000Hz$\_$wl0.035s \\
      Log-energy of signal in band from 3000-4000 Hz calculated using a 35 ms analysis window.

    \item le$\_$4000-5000Hz$\_$wl0.035s \\
      Log-energy of signal in band from 4000-5000 Hz calculated using a 35 ms analysis window.

    \item le$\_$5000-6000Hz$\_$wl0.035s \\
      Log-energy of signal in band from 5000-6000 Hz calculated using a 35 ms analysis window.

    \item le$\_$6000-7000Hz$\_$wl0.035s \\
      Log-energy of signal in band from 6000-7000 Hz calculated using a 35 ms analysis window.

    \item le$\_$7000-8000Hz$\_$wl0.035s \\
      Log-energy of signal in band from 7000-8000 Hz calculated using a 35 ms analysis window.

    \item se$\_$0-8000Hz$\_$wl0.005s \\
      Shannon entropy of power spectrum of signal in band from 0-8000 Hz (normalized to produce a density) calculated using a 5 ms analysis window. 

    \item se$\_$0-4000Hz$\_$wl0.005s \\

    \item se$\_$4000-8000Hz$\_$wl0.005s \\

    \item se$\_$0-2000Hz$\_$wl0.005s \\

    \item se$\_$2000-4000Hz$\_$wl0.005s \\

    \item se$\_$4000-6000Hz$\_$wl0.005s \\

    \item se$\_$6000-8000Hz$\_$wl0.005s \\

    \item se$\_$0-1000Hz$\_$wl0.005s \\

    \item se$\_$1000-2000Hz$\_$wl0.005s \\

    \item se$\_$2000-3000Hz$\_$wl0.005s \\

    \item se$\_$3000-4000Hz$\_$wl0.005s \\

    \item se$\_$4000-5000Hz$\_$wl0.005s \\

    \item se$\_$5000-6000Hz$\_$wl0.005s \\

    \item se$\_$6000-7000Hz$\_$wl0.005s \\

    \item se$\_$7000-8000Hz$\_$wl0.005s \\

    \item se$\_$0-8000Hz$\_$wl0.035s \\

    \item se$\_$0-4000Hz$\_$wl0.035s \\

    \item se$\_$4000-8000Hz$\_$wl0.035s \\

    \item se$\_$0-2000Hz$\_$wl0.035s \\

    \item se$\_$2000-4000Hz$\_$wl0.035s \\

    \item se$\_$4000-6000Hz$\_$wl0.035s \\

    \item se$\_$6000-8000Hz$\_$wl0.035s \\

    \item se$\_$0-1000Hz$\_$wl0.035s \\

    \item se$\_$1000-2000Hz$\_$wl0.035s \\

    \item se$\_$2000-3000Hz$\_$wl0.035s \\

    \item se$\_$3000-4000Hz$\_$wl0.035s \\

    \item se$\_$4000-5000Hz$\_$wl0.035s \\

    \item se$\_$5000-6000Hz$\_$wl0.035s \\

    \item se$\_$6000-7000Hz$\_$wl0.035s \\

    \item se$\_$7000-8000Hz$\_$wl0.035s \\
\end{enumerate}


\section{Domains selected}
\begin{enumerate}
  \item Buckeye corpus
  \item SCOTUS 2001 term
  \item TIMIT
  \item FFMTIMIT
  \item NTIMIT
  \item STCTIMIT
  \item CTIMIT
  \item WTIMIT
  \item AMI recordings from NIST RT-05S
  \item CMU recordings from NIST RT-05S
  \item ICSI recordings from NIST RT-05S
  \item NIST recordings from NIST RT-05S
  \item VT recordings from NIST RT-05S
  \item Quiet recordings from SPINE1
  \item Office recordings from SPINE1
  \item Aircras Carrier CIC recordings from SPINE1
  \item HUMVEE recordings from SPINE1
  \item AWACS recordings from SPINE1
  \item MCE recordings from SPINE 1
\end{enumerate}


\section{Corpora used}
\begin{enumerate}
  \item {\bf BUCKEYE} \\
    \begin{itemize}
        \item Duration: $\approx$27 hours
        \item Freq range: 0-8000 Hz
        \item Summary: \\
          Buckeye is a corpus of conversational speech collected from 40 native English speakers resident in Columbus, OH. All recordings were conducted using the same head-mounted microphone in a quiet room.
    \end{itemize}

  \item {\bf SCOTUS 2001 term} \\
    \begin{itemize}
        \item Duration: $\approx$74.5 hours
        \item Freq range: 0-8000 Hz
        \item Summary: \\
          These represent the entirety of the recorded output of the 2001 term of the Supreme Court of the United States, segmented into individual speaker turns. 
    \end{itemize}


  \item {\bf TIMIT} \\
    \begin{itemize}
        \item Duration: $\approx$5.5 hours
        \item Freq range: 0-8000 Hz
        \item Summary:\\
          TIMIT is a corpus of read speech recorded at TI. Each of 630 native speakers representing 8 major dialects of American English each read 10 sentences. All recording was conducted in sound-treated booth using a close-talking noise-cancelling head-mounted microphone\footnote{Although every attempt was made to keep equipment and recording conditions constant across sessions, I do recall it being mentioned that the location of the recording studio was moved to a different building partway through collection.}.
    \end{itemize}

  \item {\bf FFMTIMIT} \\
    \begin{itemize}
        \item Duration: $\approx$5.5 hours
        \item Freq range: 0-8000 Hz
        \item Summary:\\
          FFMTIMIT contains recordings from the secondary microphone employed during TIMIT data collection. Unlike the primary microphone, the secondary microphone was not head-mounted, but a Breul and Kjaer 0.5'' free-field microphone. Consequently, these recordings contain low-frequency noise not present in TIMIT (primarily from the HVAC system and mechanical vibration trasmitted through the floor of the sounnd booth).
    \end{itemize}

  \item {\bf NTIMIT} \\
    \begin{itemize}
        \item Duration: $\approx$5.5 hours
        \item Freq range: 0-4000 Hz
        \item Summary: \\
          NTIMIT was derived from TIMIT by transmitting the entirety of the corpus through a telephone handset over various channels in the NYNEX network. 
    \end{itemize}

  \item {\bf STCIMIT} \\
    \begin{itemize}
        \item Duration: $\approx$5.5 hours
        \item Freq range: 0-4000 Hz
        \item Summary: \\
          Like NTIMIT, STCTIMIT was derived from TIMIT by retransmission across the telephone network. Unlike NTIMIT, STCTIMIT was procued by passing the entire TIMIT corpus across a single channel in a single call with the result that a single type of channel distortion and noise is present across all files.
    \end{itemize}

  \item {\bf CTIMIT} \\
    \begin{itemize}
        \item Duration: $\approx$3 hours
        \item Freq range: 0-4000 Hz
        \item Summary: \\
          CTIMIT was derived from TIMIT by transmitting 3367 (out of 6300) utterances over cellular telephone channels. Transmission occured in a specially equipped van during a variety of driving conditions across New Hampshire and Massacusetts.
    \end{itemize}

  \item {\bf WTIMIT} \\
    \begin{itemize}
        \item Duration: $\approx$5.5 hours
        \item Freq range: 0-8000 Hz
        \item Summary:\\
          WTIMIT was derived from TIMIT by retransmitting the entire corpus over a 3G AMR-WB network. Consequently, while both CTIMIT and WTIMIT are mobile telephony derivatives, WTIMIT contains wideband speech.
    \end{itemize}

  \item {\bf 2005 Spring NIST Rich Transcription Eval (RT-05S} \\
    \begin{itemize}
        \item Duration: $\approx$29 hours
        \item Freq range: 0-8000 Hz
        \item Summary: \\
          The 2005 Spring NIST Rich Transcription Conference Meeting Evaluation Set consists of portions of meeting speech collected between 2001 and 2005 at 5 sites. From each of five sites, two 12-minute meeting portions were included. Each meeting was recorded from a variety of mikes included head-mounted mikes on each participant, lapel mike, and a variety of farifield microphones and microphone arrays.
    \end{itemize}

  \item {\bf SPINE1} \\
    \begin{itemize}
        \item Duration: $\approx$40.5 hours
        \item Freq range: 0-8000 Hz
        \item Summary: \\
          The Speech in Noisy Environments (SPINE) corpus consists of recordings of military personnel working on a collaborative, Battleship-like task in which they seek and attempt to sink targets. Participants were seated separately in sound-proofed chambers in which previously recorded military environments were reproduced. The resulting corpus represents a variety of handset, vocoder, and transmission channel combinations and 6 noise conditions:
          \begin{itemize}
            \item quiet
            \item office
            \item aircras carrier CIC
            \item humvee
            \item ESA AWACS
            \item MCE
          \end{itemize}
    \end{itemize}


\end{enumerate}
\end{document}
