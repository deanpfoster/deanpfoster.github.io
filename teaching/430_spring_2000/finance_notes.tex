\documentclass{article}
\renewcommand{\baselinestretch}{1.3}

\begin{document}
\section*{Notes on covariance and finance}

\subsection*{Long run growth rates}

Here are some of the equations that we have been talking about over
the past few days.

Let $R_t$ be the return on a stock at time $t$.  We will take it to be
a number bigger than or equal to zero.  Typically, $R_t \approx 1$
since you have about the same amount at the end of the time period as
at the beginning.  We want to study returns based on two properties,
E$(R)$ and Var$(R)$.  To motivate this, let's look at the long run
proformance of series of returns.

Consider the sequence of independent and identically distributed
random variables, $R_1, R_2, \ldots, R_t$.  If you put some money in
the investment at the begining of the time period and let it ride
until the end, then the final wealth divided by your orginal wealth
is 
$$R_1 \times R_2 \times \cdots \times R_t.$$
We know a lot about how to add up random variables (montra =
``expectation is linear'') so let's convert this to a sum.
$$R_1 \times R_2 \times \cdots \times R_t = e^{\ln R_1 + \ln R_2 +
\cdots + \ln R_t}$$
But the sum will be approximately $t E(\ln R)$ by the weak law of
large numbers.  So,
$$R_1 \times R_2 \times \cdots \times R_t \approx e^{t E(\ln R)}$$

Rather than compute $E(\ln R)$ we will approximate it by a Taylor
series: 
\begin{eqnarray*}
E(\ln R) &= &E\left((R-1) - (R-1)^2/2 + (R-1)^3/3 - \cdots \right)\\
&\approx& E(1 + (R-1) - (R-1)^2/2) =\\
&& = 1 + E(R-1) - E(R-1)^2/2 \\
&& = E(R) - E(R- E(R) + E(R) - 1)^2/2 \\
&& = E(R) - E(R- E(R))^2/2  + (E(R) - 1)^2/2 \\
&& = E(R) - \hbox{Var}(R)/2  + (E(R) - 1)^2/2 \\
&& \approx E(R) - 1  - \hbox{Var}(R)/2
\end{eqnarray*}
Where the first approximation comes from throwing away the ``small''
terms in the taylor series, and the last approximation comes from
throwing away the $ (E(R) - 1)^2$ term which should hopefully be small
since $E(R) \approx 1$.  So our key equation now becomes:
$$R_1 \times R_2 \times \cdots \times R_t \approx e^{t E(R) -1 -
\hbox{\scriptsize Var}(R)/2}$$ 
Note that $E(R) - 1$ is what is usually quoted as the return
rate--namely a number like 7\%, or 10\%.

Thus to evaluate the growth rate of a stock we can use 
$$\hbox{VCGR = volitility corrected growth rate} = E(R) - 1  - \hbox{Var}(R)/2$$
The nice thing about this equation is that it is in terms of only
means and variances, so we can use all that we are studying about
mean, variances and covariances.

\section*{Optimization}

Once we can measure the long run growth rate, the obvious question is
how to maximize it.  To do that we need to compute the long run growth
rate of a combination of returns.  Suppose we have two investments,
$a$ and $m$.  Let the returns on these investments be $R_{a,t}$ and
$R_{m,t}$.  We will assume that the returns are independent over time,
so we can drop the dependence on $t$ and just write $R_a$ and $R_m$.

Suppose that we put a fraction of $\alpha$ in $a$ and the rest ($1 -
\alpha$) in $m$.  Then our VCGR is:
$$\hbox{VCGR} = E(\alpha R_a + (1 - \alpha)R_m) - \hbox{Var}(\alpha
R_a + (1 - \alpha)R_m)/2$$
In general this works out to be:
$$\hbox{VCGR} = \alpha E(R_a) + (1 - \alpha)E(R_m) - \alpha^2
\hbox{Var}(R_a)/2  + (1 - \alpha)^2\hbox{Var}(R_m)/2 + \alpha(1-\alpha)\hbox{Cov}(R_a,R_m)$$
This equation is a lot easier to work with if we assume that
Var$(R_m)$ is zero.  Then the covariance also has to be zero. (Why?)
So the equation reduces to:
$$\hbox{VCGR} = \alpha E(R_a) + (1 - \alpha)E(R_m) - \alpha^2\hbox{Var}(R_a)/2$$
$$\hbox{VCGR} = E(R_m) + \alpha (E(R_a) - E(R_m)) - \alpha^2\hbox{Var}(R_a)/2$$
This is optimized at:
$$\alpha_{\hbox{\scriptsize optimum}} = \frac{E(R_a) - E(R_m)}{\hbox{Var}(R_a)}$$

But how can we deal with it if $m$ doesn't have zero variance?  If $m$
doesn't have zero variance, then lets assume that we have ``cash''
which does have zero variance!  Then our problem is to put $\alpha_a$
in $a$, and $\alpha_m$ in $m$ and the remaining amount $1 - \alpha_a -
\alpha_m$ in a risk free investment which has zero variance.  For
simplicity we will assume that the risk free rate returns zero percent
per year.  This is fairly accurate if we use inflation corrected
dollars.  We want to optimize the following then:
$$\hbox{VCGR} = E(\alpha_a R_a + \alpha_m R_m + 1 - \alpha_a - \alpha_m) - \hbox{Var}(\alpha_a
R_a + \alpha_mR_m + 1 - \alpha_a - \alpha_m)/2$$
$$\hbox{VCGR} = E(1 + \alpha_a (R_a - 1) + \alpha_m (R_m - 1)) - \hbox{Var}(\alpha_a
R_a + \alpha_m R_m)/2$$
If we try to optimize this, we still have the troublesome covariance
term.  So Let's find a $Z$ such that Cov$(Z,R_m)$ = 0, and $Z = R_a -
\beta R_m$.  Then we will invest in $Z$ and $m$ instead of investing
in $a$ and $m$.  But, if we are supposed to invest 5 in $Z$ and 8 in
$m$, this is equivalent to investing 5 in $a$ and $8 - 5\beta$ in
$m$.  So once we know how to invest in $Z$ we can figure out how to
invest in $m$.

\subsection*{Regression}

Let's find a $Z$ such that Cov$(Z,R_m)$ = 0, and $Z = R_a - \beta R_m$:
\begin{eqnarray*}
\hbox{Cov}(Z,R_m) &=& \hbox{Cov}(R_a - \beta R_m,R_m)  \\
                  &=& \hbox{Cov}(R_a,R_m)  - \beta \hbox{Cov}(R_m,R_m) \\
                  &=& \hbox{Cov}(R_a,R_m)  - \beta \hbox{Var}(R_m) 
\end{eqnarray*}
If we then set $\hbox{Cov}(Z,R_m)$ to be zero, we must have 
$$\beta = \frac{\hbox{Cov}(R_a,R_m)}{\hbox{Var}(R_m)}.$$
We can easilly compute the mean and variance of $Z$ by linearity of
expectation.  In particular:
\begin{eqnarray*}
E(Z) &=& E(R_a - \beta R_m) \\
     &=& E(R_a) - \frac{\hbox{Cov}(R_a,R_m)E(R_m)}{\hbox{Var}(R_m)}
\end{eqnarray*}
and
\begin{eqnarray*}
\hbox{Var}(Z) &=& \hbox{Var}(R_a - \beta R_m) \\
     &=& \hbox{Var}(R_a) - 2 \beta \hbox{Cov}(R_a,R_m) + \beta^2 \hbox{Var}(R_m)\\
     &=& \hbox{Var}(R_a) - 2
\left(\frac{\hbox{Cov}(R_a,R_m)}{\hbox{Var}(R_m)}\right) \hbox{cov}(R_a,R_m) +
\left(\frac{\hbox{Cov}(R_a,R_m)}{\hbox{Var}(R_m)}\right)^2 \hbox{var}(R_m)\\
     &=& \hbox{Var}(R_a) - \frac{(\hbox{Cov}(R_a,R_m))^2}{\hbox{Var}(R_m)} 
\end{eqnarray*}
\subsection*{Optimzation with a covariance of zero}

Our problem is now to optmize
$$\hbox{VCGR} = E(1 + \alpha_z (Z - 1) + \alpha_m (R_m - 1)) - \hbox{Var}(\alpha_z
Z + \alpha_m R_m)/2$$
But we know that $Z$ and $R_m$ have zero covariance.  So
$$\hbox{VCGR} = 1 + \alpha_z E(Z - 1) + \alpha_m E(R_m - 1)) -
\alpha_z^2\hbox{Var}(Z)/2 -\alpha_m^2 \hbox{Var}(R_m)/2$$
But we can rewrite this as:
$$\hbox{VCGR} = 1 + \left(\alpha_z E(Z - 1) -
\alpha_z^2\hbox{Var}(Z)/2\right)  + \left(\alpha_m E(R_m - 1)) -\alpha_m^2 \hbox{Var}(R_m)/2\right)$$
Now we can optimize $\alpha_z$ and $\alpha_m$ seperately.  So the
optimum is at:
\begin{eqnarray*}
\alpha_{z,\hbox{\scriptsize optimum}} &=& \displaystyle \frac{E(Z)}{\hbox{Var}(Z)} \\
\alpha_{m,\hbox{\scriptsize optimum}} &=& \displaystyle
\frac{E(R_m)}{\hbox{Var}(R_m)}
\end{eqnarray*}

\end{document}

