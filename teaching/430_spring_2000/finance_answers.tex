\documentclass{article}
\renewcommand{\baselinestretch}{1.3}

\begin{document}
\section*{Answers to finance homework (Homework 7)}

\subsection*{Helping grandma}

Method 1: Using ``real dollars,'' we change the growth rate of the
market from .07 to .04.  This is the inflation adjusted growth rate.
Now we compute our optimal $\alpha$ by $.04/(.2)^2$.  So she should
put 100\% of her money in the stock market.

Method 2:  Her VCGR is as follows (where R = return on the market):
\begin{eqnarray*}
\hbox{VCGR} & = & \alpha E(R) + (1-\alpha) 1.03 -
\hbox{var}(\alpha R + (1-alpha)1.03)/2 \\
& = & \alpha 1.07 + (1-\alpha) 1.03 - \alpha^2 \hbox{var}(R)/2 \\
& = & 1 + \alpha .04 - \alpha^2 * .04/2 
\end{eqnarray*}
Which is optimized at $\alpha = 1$.

\subsection*{Red, Green, White}

We are assuming that white is cash, so its return is always exactly 1
(or zero depending on which return you are using).  Let $R_r$ be the
random variable describing the return on red, and $R_g$ be the random
variable describing the return on green:
\begin{eqnarray*}
\hbox{VCGR} & = & \alpha_r E(R_r) + \alpha_g E(R_g) + (1-\alpha_r
- \alpha_g) 1 - \hbox{var}(\alpha_r R_r + \alpha_g R_g + (1-\alpha_r - \alpha_g)1)/2 \\
& = & \alpha_r 1.71 + \alpha_g 1.07 + (1-\alpha_r
- \alpha_g) 1 - \alpha_r^2 \hbox{var}(R_r)/2 - \alpha_g^2 \hbox{var}(R_g)/2 \\
& = & 1 + \alpha_r .71 + \alpha_g .075 - \alpha_r^2 1.30^2/2 - \alpha_g^2 .20^2/2 \\
& = & 1 + \left(\alpha_r .71 - \alpha_r^2 1.30^2/2\right) + \left(\alpha_g .075 - \alpha_g^2 .20^2/2 \right)
\end{eqnarray*}
Now each of the two pieces can be seperately optimized.
$\alpha_{r,\hbox{optimum}} = .71/1.69 = .42$ and
$\alpha_{g,\hbox{optimum}} = .075/.04 = 1.875$, so the fraction put in
cash must be -1.295.  By plugging these back in to the equation we can
get the VCGR.

\subsection*{Ford: ``Any color you want as long as its black.''}

I'll just give the answers here.  See the handout for some idea of
what is going on.  The $\beta$ is $.03/.04 = .75$.  So the $Z$ we want
is: $Z = \hbox{Ford} - .75 \hbox{Market}$.  The $E(Z) = .09 - .75 *
.07 = .0375$ (notice this is the ``little r'' returns).  The variance
of $Z$ is: 

$$\hbox{var}{Z} = \hbox{Var}(R_f) - \frac{(\hbox{Cov}(R_f,R_m))^2}{\hbox{Var}(R_m)} $$
$$  = .06 - .03^2/.04 $$
$$  = .06 - .03^2/.04 = .0375$$
(How's that for an amazing coincidence!)  So the optimal amount to put
into $Z$ is 100\%. The amount we put in the market is .07/.04 = 1.75 =
175\%. 

Unfortunately, this has solved the wrong problem.  We want to invest
in ford and the market not $Z$ and the market.  Since  $Z =
\hbox{Ford} - .75 \hbox{Market}$ if we want $(Z = 1, M = 1.75)$ we can
generate this using $(F = 1, M = 1)$.  So we put 100\% into both.
This leads to -100\% in cash.

The easiest way to comput the VCGR is to use the Z,M decomposition.
Then: 
$$\hbox{VCGR} = 1 E(Z) + 1 E(R_m) -1 - .0375/2 - .04/2$$
$$\hbox{VCGR} = 1.0375 + 1.07 -1 - .0375/2 - .04/2$$
$$\hbox{VCGR} = 1.068$$
so our best growth rate is about 6.8\%.  

\hbox{amazon}

The key problem here is that you don't have any values to work with.
So you have to guess.  One guess for beta from Yahoo is 2.87.  Other
people have found beta's of about 1.6.  YMMV.  Using the last 250 days
you can create an annual variance estimate of .945.  Or you can
eye-ball it off the graph and get anything from .5 up to 2.  If you
are outside this range, you might need new glasses.

The most difficult one to find out is the growth rate.  If you used
the growth rate since amazon was created you'll have a very large
number.  If you used the last year, you might get 5 percent growth, or
possibly -5\% growth--depending on when your year begins and ends.
If you believe in the efficent market hypothesis, then Z should have
neither a positive return nor a negative return.  So $E(Z) = 0$.  This
means that $E(R_{\hbox{Amazon}} = \beta R_M$.  The idea is that both
the beta and the return on the market are more accurately estimated
than the return on Amazon is.  The nice thing about using this return
is that the amount you put into $Z$ is exactly zero!  So it boils down
to just finding the correct amount to put in the market.

\end{document}
