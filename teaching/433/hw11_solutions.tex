\documentclass[10pt,a4paper]{article}

\usepackage{color}

\begin{document}

\begin{flushleft}
Course No. Stat 433 \\
\today
\end{flushleft}

\begin{center}
{\Large{\bf  Homework 11 Solution}}
\end{center}

\textcolor[rgb]{0.98,0.00,0.00}{Comments from the grader:}
\begin{itemize}

    \item \textcolor[rgb]{0.98,0.00,0.00}{These are only partial solutions.  We selected
    questions which were problematic to most of the class or are of particular interest.}
    \item \textcolor[rgb]{0.98,0.00,0.00}{The maximum grade for this homework assignment is 10.}
    \item \textcolor[rgb]{0.98,0.00,0.00}{Your solution should contain explanations and not only
    final answers. Points will be deducted if partial solutions
    are submitted.}
    \item \textcolor[rgb]{0.98,0.00,0.00}{Please save a copy of your work and submit the original.
    Write your name and email on top of the first page.}
    \item \textcolor[rgb]{0.98,0.00,0.00}{if you notice a typo in the solution file or have a problem with the homework
    grading please email: sivana@wharton.upenn.edu
}
\end{itemize}


\begin{flushleft}

\begin{eqnarray*}
\\
\end{eqnarray*}


\textbf{Question 4.9}

Notice that given $N(t)$,$W_1,\ldots,W_{N(t)}$ follow a uniform
distribution. Hence,
\begin{eqnarray*}
E(W_1,\ldots,W_n|N(t)=n)&=&(\frac{t}{2})^n\\
&\Rightarrow& E(W_1,\ldots,W_n)=E((\frac{t}{2})^{N(t)})\\
&=&\sum_{n=0}^\infty (\frac{t}{2})^n \frac{(\lambda t)^n
e^{-\lambda
t}}{n!}\\
&=&\sum_{n=0}^\infty \frac{(\frac{t^2}{2} \lambda )^n
e^{-\lambda t}}{n!}\\
&=&e^{-\lambda t} \sum_{n=0}^\infty \frac{(\frac{t^2}{2} \lambda
)^n}{n!}\\
&=&e^{-\lambda t} e^{(\frac{t^2}{2} \lambda)}\\
&=&e^{-\lambda t (1- \frac{t}{2})}\\
\end{eqnarray*}

\begin{eqnarray*}
\\
\end{eqnarray*}



\textbf{Question 1.2}\\

We know that $\lambda_k=\alpha+k\beta$ and that $X(0)=0$. Also
from the book we know that $P_n(t) = \lambda_0 \cdots
\lambda_{n-1} [\sum_{i=0}^{n} B_{i,n} e^{-\lambda_i t}]$.

Now we need to figure out the $B$ coefficients.

\begin{eqnarray*}
P_0(t)&=&e^{-\alpha t}\\
P_1(t) &=& \lambda_0 [B_{0,1} e^{-\alpha t} + B_{1,1}
e^{-(\alpha+\beta) t}\\
&\Rightarrow&
B_{0,1}=\frac{1}{\lambda_1-\lambda_0}=\frac{1}{\beta}\\
&\Rightarrow&
B_{1,1}=\frac{1}{-\lambda_1+\lambda_0}=-\frac{1}{\beta}\\
\end{eqnarray*}

After playing around with the equations (just work out a few more
$B$ coefficients) we can see that
\begin{eqnarray*}
B_{k,n}&=&\frac{1}{(\lambda_0-\lambda_k)(\lambda_1-\lambda_k)
\cdots (\lambda_{k-1}-\lambda_k)(\lambda_{k+1}-\lambda_k) \cdots
(\lambda_{n}-\lambda_k)}\\
&\Rightarrow& B_{k,n} = \frac{(-1)^k}{\beta^n \cdot k!(n-k)!}
\end{eqnarray*}

Hence,
\begin{eqnarray*}
P_n(t) &=&\prod_{k=0}^{n-1} (\alpha+k\beta) \sum_{k=0}^n
\frac{(-1)^k}{\beta^n \cdot k!(n-k)!} e^{-(\alpha+\beta k )t}\\
&=&(\prod_{k=0}^{n-1} (\alpha+k\beta)) \cdot \frac{e^{- \alpha t
}}{\beta^n \cdot n!} \sum_{k=0}^n \left ( \begin{array}{c}
 n\\
 k \end{array} \right ) (-e^{-\beta t} )^k \\
 &=&(\prod_{k=0}^{n-1} (\alpha+k\beta)) \cdot \frac{e^{- \alpha t
}}{\beta^n \cdot n!} (1-e^{-\beta t} )^n \\
\end{eqnarray*}

\begin{eqnarray*}
\\
\end{eqnarray*}


\textbf{Question 1.3}\\

\begin{eqnarray*}
\lambda_i &=&\lim_{n\rightarrow0} \frac{P(X(t+h)-X(t)=1|X(t)=x)}{h}\\
&=&\lim_{n\rightarrow0} \frac{ \left ( \begin{array}{c}
 i(n-i)\\
 1 \end{array} \right ) (\alpha h +o(h))(1-\alpha h + o(h))^{i(n-i)-1}
}{h}\\
&=&  i (n-i) \cdot \alpha
\end{eqnarray*}

Intuitively, there are $n-i$ susceptible people and i infected
people. Also, $\lambda_0=0$ since no one can infect.


\begin{eqnarray*}
\\
\end{eqnarray*}



\textbf{Problem 2.1}\\
All the sojourns and T are independent and exponentially
distributed with parameters $\mu_1,\ldots,\mu_N,\theta$. We will
use this fact to find the desired probability.
\begin{eqnarray*}
P(X(t)=0)&=& P(T>\sum_{n=1}^N S_n)\\
&=&\int_0^\infty \ldots \int_0^\infty \int_{s_1+\cdots+s_N}^\infty
\mu_1 e^{-\mu_1 s_1} \cdots \mu_N e^{-\mu_N s_N}\theta e^{-\theta
t} dt \cdot ds_N \cdots ds_1\\
&=&\int_0^\infty \ldots \int_0^\infty \mu_1 e^{-\mu_1 s_1} \cdots
\mu_N e^{-\mu_N s_N} (\int_{s_1+\cdots+s_N}^\infty \theta
e^{-\theta t} dt )\cdot ds_N \cdots ds_1\\
&=&\int_0^\infty \ldots \int_0^\infty \mu_1 e^{-\mu_1 s_1} \cdots
\mu_N e^{-\mu_N s_N} (e^{-\theta (s_1+\cdots+s_N)}) \cdot ds_N \cdots ds_1\\
&=&\int_0^\infty \ldots \int_0^\infty \mu_1 e^{-(\mu_1+\theta)
s_1} \cdots \mu_N e^{-(\mu_N+\theta) s_N} ds_N \cdots ds_1\\
&=&\prod_{j=1}^N \frac{\mu_j}{\mu_j+\theta} \int_0^\infty \ldots
\int_0^\infty (\mu_1+\theta) e^{-(\mu_1+\theta)
s_1} \cdots (\mu_N+\theta) e^{-(\mu_N+\theta) s_N} ds_N \cdots ds_1\\
&=&\prod_{j=1}^N \frac{\mu_j}{\mu_j+\theta}
\end{eqnarray*}

The last step is true since all of the variables are independent
and we can integrate each one individually. Also, we integrate
each variable over the (adjusted) exponential density which of
course yields a result of value 1.

Some students only explained why intuitively this results should
hold. They received full marks for that but here we wanted to also
show you how you can prove this mathematically.

\begin{eqnarray*}
\\
\end{eqnarray*}


\textbf{Question 2.3}\\

\begin{eqnarray*}
E(W_1,\ldots,W_N)&=& \sum_{i=1}^N E(W_i)\\
\end{eqnarray*}

Now notice (and if you can't do that look at graph in the book
which describes the pure-death process) that

\begin{eqnarray*}
E(W_1)&=& E(S_N)=\frac{1}{\mu_N}\\
E(W_2)&=& E(S_N+S_{N-1})=\frac{1}{\mu_N}+\frac{1}{\mu_{N-1}}\\
\vdots\\
E(W_i)&=& \sum_{i=N-i+1}^N \frac{1}{\mu_i} \\
\vdots\\
E(W_N)&=& \sum_{i=1}^N \frac{1}{\mu_i}\\
\end{eqnarray*}

Hence,

\begin{eqnarray*}
E(W_1,\ldots,W_N)&=& \sum_{i=1}^N E(W_i)\\
&=& \sum_{i=1}^N i \frac{1}{\mu_i} \\
\end{eqnarray*}


Some people mistakenly wrote that $E(W_i)= \sum_{i=1}^{N-i+1}
\frac{1}{\mu_i}$. This is not true and the graph of the pure-death
process should clarify why (in case the math didn't).
\begin{eqnarray*}
\\
\end{eqnarray*}

\end{flushleft}
\end{document}
