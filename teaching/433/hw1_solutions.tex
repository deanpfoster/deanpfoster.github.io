\documentclass[10pt,a4paper]{article}

\usepackage{color}

\begin{document}

\begin{flushleft}
Course No. Stat 433 \\
\today
\end{flushleft}

\begin{center}
{\Large{\bf  Homework 1 Solution}}
\end{center}

\textcolor[rgb]{0.98,0.00,0.00}{Comments from the grader:}
\begin{itemize}

    \item \textcolor[rgb]{0.98,0.00,0.00}{These are only partial solutions.  We selected
    questions which were problematic to most of the class.}
    \item \textcolor[rgb]{0.98,0.00,0.00}{The maximum grade for this homework assignment is 10.}
    \item \textcolor[rgb]{0.98,0.00,0.00}{Your solution should contain explanations and not only
    final answers. Points will be deducted if partial solutions
    are submitted.}
    \item \textcolor[rgb]{0.98,0.00,0.00}{Please save a copy of your work and submit the original.
    Write your name and email on top of the first page.}
    \item \textcolor[rgb]{0.98,0.00,0.00}{if you notice a typo in the solution file or have a problem with the homework
    grading please email: sivana@wharton.upenn.edu
}
\end{itemize}


\begin{flushleft}


\textbf{Page 23 Question 12}
\begin{eqnarray*}
Cov(X,Y) &=& Cov(U+W,V-W) \\
&=& Cov(U,V)-Cov(U,W)+Cov(W,V)-Cov(W,W)\\
&=&=-\sigma^2
\end{eqnarray*}

Since U,W and V are all independent their covariances are all
zeros. Since $ Cov(W,W)=Var(W)$ this yields the above result.

\begin{eqnarray*}
\\
\end{eqnarray*}

\textbf{Page 61 Question 2}\\
 Let X be a random variable which counts the number of heads we get after tossing
 four nickels.
 Let Y be a random variable which counts the number of heads we
 get by tossing six dimes.
 Assuming that the coins are fair coins, X and Y have both a
 binomial distribution with probability of success which equals
 $0.5$. $X\sim Bin(4,0.5)$ and $Y\sim Bin(6,0.5)$.  It follows
 that $X+Y~Bin(10,0.5)$. Hence,
 \begin{eqnarray*}
P(X=2|X+Y=4)&=&\frac{P(X=2,Y=2)}{P(X+Y=4)}\\
&=& \frac{P(X=2)P(Y=2)}{P(X+Y=4)}\\
&=& \frac{3}{7}
 \end{eqnarray*}

\newpage
\textbf{Page 62 Question 2 Part c}\\
Since everyone correctly solved the first two parts of this
question we will only concentrate on the third part.

Unfortunately, this problem does lend itself to easy analysis.  In
other words, it is hard to write it out as conditional probabilities.
Instead, we need to go back to our definition of probability--namely
count of events divided by total counts.

So if we think of there being alot of families, say $n$ of them, we
 can compute the number of sons altogether as $n E(X)$ where $X$ is a
 random variable counting the number of boys in a family.  Now if we
let $Z$ be the expected number of sons in a family who have a sister,
the number of sons in the population who have a sister will be close
to $n E(Z)$.  So our answer then is approximately $n E(Z)/(n E(X))$
which will be close to $E(Z)/E(X)$ if there are a lot of families.  So
this is the answer we are looking for.

\begin{eqnarray*}
E(X) &=& \sum_{i=0}^3 i P(X=i)\\
&=& 0 \cdot \frac{1}{2}+ 1 \cdot \frac{1}{4}+ 2 \cdot \frac{1}{8} + 3 \cdot \frac{1}{8} \\
&=& \frac{7}{8}
\end{eqnarray*}

\begin{eqnarray*}
E(Z) &=&  0 \cdot \frac{1}{2}+ 1 \cdot \frac{1}{4}+ 2 \cdot \frac{1}{8} + 0 \cdot \frac{1}{8} \\
&=& \frac{1}{2}
\end{eqnarray*}
So the probability of having a sister for a random boy is $4/7$.  This
means the probability of not having a sister is $3/7$.  This describes
the distribution of sisters for a random boy.  Notice that using a
similar argument, the probability that a random girl having a sister
is $\frac{0}{(7/8)}$ which is zero.

We can use the same analysis to find the number of brothers.  But it
takes a bit more work since there is the possibility of having zero,
one or two brothers.  Let $W_0$ be the number of sons who
have no brothers, $W_1$ be the number who have one brother and $W_2$
be the number who have 2 brothers.  Then our probability distribution
we are after is $P(\hbox{number of brothers} = i) = E(W_i)/E(X)$.  

\begin{eqnarray*}
E(W_0) &=&  0 \cdot \frac{1}{2}+ 1 \cdot \frac{1}{4}+ 0 \cdot \frac{1}{8} + 0 \cdot \frac{1}{8} \\
&=& \frac{2}{8}\\
E(W_1) &=&  0 \cdot \frac{1}{2}+ 0 \cdot \frac{1}{4}+ 2 \cdot \frac{1}{8} + 0 \cdot \frac{1}{8} \\
&=& \frac{2}{8}\\
E(W_2) &=&  0 \cdot \frac{1}{2}+ 0 \cdot \frac{1}{4}+ 0 \cdot \frac{1}{8} + 3 \cdot \frac{1}{8} \\
&=& \frac{3}{8}\\
\end{eqnarray*}
So our distribution is $P(0) = 2/7$, $P(1) = 2/7$ and $P(2) = 3/7$.

Please notice that since these are both distribution functions the
probabilities add up to one. This was a common mistake with some
students. Also there is no relevance as to which brother you are
sampling, i.e. the first, the second or the third.

\end{flushleft}
\end{document}
