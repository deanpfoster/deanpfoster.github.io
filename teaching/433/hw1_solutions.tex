\documentclass[10pt,a4paper]{article}

\usepackage{color}
\usepackage{amssymb}
\usepackage{amsmath,amsfonts,bm,rotating}

\begin{document}

\begin{flushleft}
Course No. Stat 433 \\
\today
\end{flushleft}

\begin{center}
{\Large{\bf  Homework 1 Solution}}
\end{center}

\textcolor[rgb]{0.98,0.00,0.00}{Comments from the grader:}
\begin{itemize}

    \item \textcolor[rgb]{0.98,0.00,0.00}{These are only partial solutions.  We selected
    questions which were the most problematic for the class.}
    \item \textcolor[rgb]{0.98,0.00,0.00}{The maximum grade for this homework assignment is 10.}
    \item \textcolor[rgb]{0.98,0.00,0.00}{Your solution should contain explanations, and not just
    final answers. Points will be deducted if partial solutions
    are submitted.}
    \item \textcolor[rgb]{0.98,0.00,0.00}{Make sure that your work is readable/understandable.  If necessary, skip every other line.  In addition, please {\bf staple} your homework.}
    \item \textcolor[rgb]{0.98,0.00,0.00}{If you notice a typo in the solution file or have a problem with the homework
    grading, please come by my office hours, or email me (entine4@wharton.upenn.edu)}

\end{itemize}


\begin{flushleft}


\textbf{Page 23 Question 12}
\begin{eqnarray*}
Cov(X,Y) &=& Cov(U+W,V-W) \\
&=& Cov(U,V)-Cov(U,W)+Cov(W,V)-Cov(W,W)\\
&=&=-\sigma^2
\end{eqnarray*}

Since U,W and V are all independent their covariances are all
zeros. Since $ Cov(W,W)=Var(W)$ this yields the above result.
\bigskip

\textbf{Page 61 Question 2}\\
 Let X be a random variable which counts the number of heads we get after tossing
 four nickels.
 Let Y be a random variable which counts the number of heads we
 get by tossing six dimes.
 Assuming that the coins are fair coins, X and Y have both a
 binomial distribution with probability of success which equals
 $0.5$. $X\sim Bin(4,0.5)$ and $Y\sim Bin(6,0.5)$.  It follows
 that $X+Y~Bin(10,0.5)$. Hence,
 \begin{eqnarray*}
P(X=2|X+Y=4)&=&\frac{P(X=2,Y=2)}{P(X+Y=4)}\\
&=& \frac{P(X=2)P(Y=2)}{P(X+Y=4)}\\
&=& \frac{3}{7}
 \end{eqnarray*}

\newpage

\textbf{Page 63 Question 6}\\
First note that 
\[P(X=x,N=n)=P(N=n)P(X=x|N=n)=\frac{1}{2}(1-\frac{1}{2})^{n-1}\binom{n}{x}\left(\frac{1}{2}\right)^n\]
Thus:
\[P(X=0)=P(X=0,N=1)+P(X=0,N=2)+\ldots\]
\[= \sum^{\infty}_{n=1}(\frac{1}{2})^{2n}\binom{n}{0}=\sum^{\infty}_{n=1}(\frac{1}{4})^n\]
This is a simple geometric series, so
\[P(X=0)=\frac{\frac{1}{4}}{1-\frac{1}{4}}=\frac{3}{4}\]

$P(X=1)$ is computed similarly:
\[P(X=1)=\sum^{\infty}_{n=1}n(\frac{1}{4})^n\]
There are a few ways to compute this sum, here is one of the niftier ones.  Let $f(x)=\sum^{\infty}_{n=0}x^n=\frac{1}{1-x}$.  Differentiation both sides, we get
\[\sum^{\infty}_{n=0}nx^{n-1}=\frac{1}{(1-x)^2}\]
Multiplying both sides by x, we get:
\[\sum nx^n=\frac{x}{(1-x)^2}\]
Letting $x=\frac{1}{4}$, we get $P(X=1)=\frac{\frac{1}{4}}{(1-\frac{1}{4})^2}=\frac{4}{9}$.
\bigskip



\textbf{Page 64 Question 10 Part c}\\
Since almost everyone correctly answered the first two parts of this
question, we will only concentrate on the third part.

This one is tricky, but here's a good way of going about it.  By selecting a boy, you are conditioning on his family having at least one boy.  Thus, 
\[P(G)=0, P(BG)=\frac{\frac{1}{4}}{\frac{1}{2}}=\frac{1}{2}, P(BBG)=\frac{1}{4}, P(BBB)=\frac{1}{4}\]
Now consider a small population (of families with boys) which contains these types of families in the proportions given above.  The simplest example is:
\[\{BG\}, \{BG\}, \{BBG\},\{BBB\}\]
The chance of selecting at random any particular boy is $\frac{1}{7}$.  Since four boys are in families containing 1 sister, P(1 sister)$=\frac{4}{7}$, and thus P(0 sisters)$=\frac{3}{7}$

The exact same logic holds for finding the probability distribution function for the number of brothers a particular boy has.  P(0 brothers)=P(selecting boy from BG)=$\frac{2}{7}$, P(1 brother)=P(boy from BBG)=$\frac{2}{7}$, and P(2 brothers)=$\frac{3}{7}$.

\end{flushleft}
\end{document}
