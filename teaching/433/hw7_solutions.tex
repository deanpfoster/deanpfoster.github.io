\documentclass[10pt,a4paper]{article}

\usepackage{color}

\begin{document}

\begin{flushleft}
Course No. Stat 433 \\
\today
\end{flushleft}

\begin{center}
{\Large{\bf  Homework 7 Solution}}
\end{center}

\textcolor[rgb]{0.98,0.00,0.00}{Comments from the grader:}
\begin{itemize}

    \item \textcolor[rgb]{0.98,0.00,0.00}{These are only partial solutions.  We selected
    questions which were problematic to most of the class.}
    \item \textcolor[rgb]{0.98,0.00,0.00}{The maximum grade for this homework assignment is 10.}
    \item \textcolor[rgb]{0.98,0.00,0.00}{Your solution should contain explanations and not only
    final answers. Points will be deducted if partial solutions
    are submitted.}
    \item \textcolor[rgb]{0.98,0.00,0.00}{Please save a copy of your work and submit the original.
    Write your name and email on top of the first page.}
    \item \textcolor[rgb]{0.98,0.00,0.00}{if you notice a typo in the solution file or have a problem with the homework
    grading please email: sivana@wharton.upenn.edu
}
\end{itemize}


\begin{flushleft}

\begin{eqnarray*}
\\
\end{eqnarray*}


\textbf{Page 211 Question 1}

Let $X$ indicate the number of balls in A.

The appropriate transition matrix is:
\[ P = \left ( \begin{array}{cccccc}
 0.5 & 0.5 & 0 & 0 & 0 & 0 \\
 0.5 & 0 & 0.5 & 0 & 0 & 0  \\
 0 & 0.5 & 0 & 0.5 & 0 & 0  \\
 0 & 0 & 0.5 & 0 & 0.5 & 0  \\
 0 & 0 & 0 & 0.5 & 0 & 0.5  \\
 0 & 0 & 0 & 0 & 0.5 & 0.5  \\
\end{array} \right) \]

First check and see that $P$ is regular (i.e., $P^8$ is a matrix
where all the entries are positive). Since the rows and columns of
$P$ both add up to 1 it means that $P$ is doubly stochastic.
Hence, the limit distribution is
$\pi=(\frac{1}{6},\ldots,\frac{1}{6})^{'}$. This means that the
markov chain spends $\frac{1}{6}$ of its time in state 0 (or any
other state).


\begin{eqnarray*}
\\
\end{eqnarray*}



\textbf{Page 211 Question 6}\\
\begin{enumerate}
    \item $lim_{n\longrightarrow\infty}P(X_{n+1}=j|X_0=i)=\pi_j$
    \item $lim_{n\longrightarrow\infty}P(X_{n}=k,X_{n+1}=j|X_0=i)=\pi_k \cdot p_{kj}$
    \item $lim_{n\longrightarrow\infty}P(X_{n-1}=k,X_{n}=j|X_0=i)=\pi_k \cdot p_{kj}$
\end{enumerate}
\begin{eqnarray*}
\\
\end{eqnarray*}

\textbf{Page 211 Question 10}\\
$P$ is regular and has $N+1$ states (from 0 to $N$). $P$ is also
doubly stochastic and hence its limit distribution is
$\pi=(\frac{1}{N+1},\ldots,\frac{1}{N+1})^{'}$



\begin{eqnarray*}
\\
\end{eqnarray*}

\textbf{Page 211 Question 12}\\

a.Since $\Pi$ is the stationary distribution matrix and $P =
Q+\Pi$ we know that:
\begin{itemize}
    \item $\Pi P = \Pi $
    \item $ \Pi^2= \Pi $
    \item $\Pi Q = \Pi P -\Pi^2 = 0$
\end{itemize}

Hence since $\Pi Q = 0$ and $\Pi^n=Pi$
\begin{eqnarray*}
P^n &=& (Q+\Pi)^n\\
 &=& Q^n+\Pi^n\\
 &=& Q^n+\Pi
\end{eqnarray*}

b. First check and see that $P$ is regular (since $P^2$ has
positive entries). Hence by solving $\pi P = \pi$ we get the limit
distribution which is $\pi = (0.25,0.5,0.25)$.

Which means that

\[ \Pi = \left ( \begin{array}{ccc}
 0.25 & 0.5 & 0.25 \\
 0.25 & 0.5 & 0.25 \\
 0.25 & 0.5 & 0.25 \\
\end{array} \right) \]


\[ Q = \left ( \begin{array}{ccc}
 0.25 & 0 & -0.25 \\
 0 & 0 & 0 \\
 -0.25 & 0 & 0.25 \\
\end{array} \right) \]

Hence,
\[ Q^n = \frac{1}{2^{n+1}}\left ( \begin{array}{ccc}
 1 & 0 & -1 \\
 0 & 0 & 0 \\
 -1 & 0 & 1 \\
\end{array} \right) \]

\[ P^n = \left ( \begin{array}{ccc}
 0.25+\frac{1}{2^{n+1}} & 0.5 & 0.25-\frac{1}{2^{n+1}} \\
 0.25 & 0.5 & 0.25 \\
 -\frac{1}{2^{n+1}}+0.25 & 0.5 & \frac{1}{2^{n+1}}+0.25 \\
\end{array} \right) \]


\begin{eqnarray*}
\\
\end{eqnarray*}

\end{flushleft}
\end{document}
