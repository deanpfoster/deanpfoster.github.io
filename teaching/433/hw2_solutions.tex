\documentclass[10pt,a4paper]{article}

\usepackage{color}

\begin{document}

\begin{flushleft}
Course No. Stat 433 \\
\today
\end{flushleft}

\begin{center}
{\Large{\bf Solution Homework 2}}
\end{center}

\textcolor[rgb]{0.98,0.00,0.00}{Comments from the grader:}
\begin{itemize}

    \item \textcolor[rgb]{0.98,0.00,0.00}{These are only partial solutions.  We selected
    questions which were problematic to most of the class.}
    \item \textcolor[rgb]{0.98,0.00,0.00}{The maximum grade for this homework assignment is 10.}
    \item \textcolor[rgb]{0.98,0.00,0.00}{Your solution should contain explanations and not only
    final answers. Points will be deducted if partial solutions
    are submitted.}
    \item \textcolor[rgb]{0.98,0.00,0.00}{Please save a copy of your work and submit the original.
    Write your name and email on top of the first page.}
    \item \textcolor[rgb]{0.98,0.00,0.00}{if you notice a typo or have a problem with the homework
    grading please email: sivana@wharton.upenn.edu
}
\end{itemize}

\begin{flushleft}
Comments regarding this assignment: The definition of a martingale
consists of two parts (see page 87 in the text book). Most of you
did not prove the first part, i.e. $E|X_n|<\infty$. If you did not
prove this throughout the assignment you lost 1 point.



\begin{eqnarray*}
\\
\end{eqnarray*}

\textbf{Page 94 Question 1}
\begin{itemize}
    \item Let $X=X_{n+2}$, $Y=X_{n+1}$ and $Z={X_0,\ldots,X_n}$.
    From these definitions the first identity is proven.
    \item Hence if $X_n$ is a martingale then:
    \begin{eqnarray*}
    E[X_{n+2}|X_0,\ldots,X_n]\\
    &=&E[E[X_{n+2}|X_0,\ldots,X_{n+1}]|X_0,\ldots,X_n]\\
    &=&E[X_{n+1}|X_0,\ldots,X_n]= X_n
    \end{eqnarray*}

    \end{itemize}



\begin{eqnarray*}
\\
\end{eqnarray*}

\textbf{Page 94 Question 2} We need to verify that
\begin{enumerate}
    \item $E|X_n|<\infty$
    \item $E[X_{n+1}|X_0,\ldots,X_n]=X_n$.

\end{enumerate}
For the first part notice that $X_n$ has non-negative values.
Hence, $E|X_n|=E(X_n)$.

\begin{eqnarray*}
E|X_n|&=&E[X_n]\\
&=& E[2^{n}U_1\ldots U_{n}]\\
&=& 2^n \prod_{i=1}^n E[U_i]\\
&=& 1<\infty
\end{eqnarray*}

The second part prove is as follows:
\begin{eqnarray*}
E[X_{n+1}|X_0,\ldots,X_n]&=& E[2^{n+1}U_1\ldots
U_{n+1}|1,2U_1,\ldots,2^n U_1 \ldots U_n]\\
&=& E[2^{n+1}U_1\ldots
U_{n+1}|2^n U_1 \ldots U_n]\\
&=& 2^{n}U_1 \ldots U_n \cdot E[2U_{n+1}|2^n U_1 \ldots U_n]\\
&=& 2^{n} U_1 \ldots U_n \cdot E[2U_{n+1}] = 2^n U_1 \ldots U_n =
X_n
\end{eqnarray*}



\begin{eqnarray*}
\\
\end{eqnarray*}

\textbf{Page 94 Question 4} Again we need to prove the two parts
of the martingale definition:
\begin{itemize}
    \item First $E[|X_n|]=E[X_n]$ since $X_n$ can only have
    non-negative values. Hence,
    \begin{eqnarray*}
    E[X_n]&=&E[p^{-n} \cdot \xi_1 \cdots \xi_n]\\
    &=& p^{-n} \cdot \prod_{i=1}^n E[\xi_i]\\
    &=& p^{-n} \cdot p^n \\
    &=& 1 < \infty
    \end{eqnarray*}
    \item The following is the proof for the second part of the
    martingale definition:
    \begin{eqnarray*}
    E[X_{n+1}|X_0,\ldots,X_n]\\
    &=&E[p^{-(n+1)} \cdot \xi_1 \cdots \xi_{n+1}|X_0,\ldots,X_n]\\
    &=&E[p^{-1} \cdot X_n \cdot \xi_{n+1}|X_0,\ldots,X_n]\\
    &=& p^{-1} \cdot X_n \cdot p \\
    &=& X_n
    \end{eqnarray*}
\end{itemize}

A lot of the students ignored the second part of this question
about the convergence of $X_n$ (and lost points as a result).
Notice that for any $\epsilon>0$ the following holds:
\begin{eqnarray*}
P(|X_n-0|>\epsilon)&=&P(X_n>\epsilon)\\
&=&P(\xi_1 \cdots \xi_{n}\neq 0 )\\
&=&P(\xi_1 \cdots \xi_{n}= 1 )\\
&=& p^{-n}
\end{eqnarray*}

By letting $n$ go to infinity we know that
$P(|X_n-0|>\epsilon)\rightarrow 0$ for any $\epsilon>0$. This
means that $X_n$ converges in probability to 0.



\begin{eqnarray*}
\\
\end{eqnarray*}

\textbf{Page 99 Question 1} The matrix has the following
structure:
\[ \left( \begin{array}{cccccc}
1 & 0 & 0 & 0 & 0 & 0  \\
0 & * & * & 0 & 0 & 0  \\
0 & 0 & * & * & 0 & 0  \\
0 & 0 & 0 & * & * & 0  \\
0 & 0 & 0 & 0 & * & *  \\
0 & 0 & 0 & 0 & 0 & 1  \\
\end{array} \right)\]

Consider the second row probabilities.  Given that the current
period has a single diseased person, the probability that there
are 2 diseased people in the next period is $P(X_{n+1}=2|X_n=1) =
\alpha \cdot P(\textrm{the diseased person interacts})$. Which
means $P(X_{n+1}=2|X_n=1) = 0.1 \cdot (1-0.6) = 0.04$. Hence the
complimentary probability is 0.96. Following the same logic yields
the following matrix:
\[ \left( \begin{array}{cccccc}
1 & 0 & 0 & 0 & 0 & 0  \\
0 & 0.96 &  0.04 & 0 & 0 & 0  \\
0 & 0 & 0.96  & 0.04 & 0 & 0  \\
0 & 0 & 0 & 0.96 & 0.04 & 0  \\
0 & 0 & 0 & 0 & 0.96 & 0.04  \\
0 & 0 & 0 & 0 & 0 & 1  \\
\end{array} \right)\]
\end{flushleft}

\end{document}
