\documentclass[12pt]{article}
\usepackage{simplemargins}
\setallmargins{.5in}
\settopmargin{1.2in}
\renewcommand{\baselinestretch}{1.3}
\pagestyle{empty}
\begin{document}
Dean Foster \hfill {\bf Stat 433: Final Exam} \hfill\today
\vspace{1em}

You are allowed both your midterm cheat-sheet and a new one if you made
it up.  No calculators!

\begin{enumerate}

% % % % % % % % % % % % % % % % % % % % % % % % % % % % % % % % % % % % % %

\item Suppose we have a 5 state Markov chain represented by the
following matrix: 
\begin{displaymath}
\left[\begin{array}{ccccc}
        0 & 0 & 1 & 0 & 0 \\
        0 & .9& 0 & .1& 0 \\
        0 & 0 & 0 & 0 & 1 \\
        0 & .1& 0 & .9& 0 \\
        0 & 0 & 1 & 0 & 0
\end{array}
\right]
\end{displaymath}
where we label the states 1-5.
\begin{enumerate}
\item Which states are transient?
\item Which states are periodic?
\item What are the communication classes for this process?
\item If the process started in second state, what will the
distribution be in the limit?
\end{enumerate}

% % % % % % % % % % % % % % % % % % % % % % % % % % % % % % % % % % % % % %

\item Consider 3 standard Brownian motions, $X(t), Y(t),$ and $Z(t)$.
Define the norm process as
\begin{displaymath}
N(t) = X(t)^2 + Y(t)^2 + Z(t)^2.
\end{displaymath}
Define $M(t) = N(t) - f(t)$ for some deterministic function $f(t)$. 
\begin{enumerate}
\item What would the function $f(t)$ have to be to make $E(M(t)) = 0$
for all $t$? 
\item For this $f()$, is the fact that  $E(M(t)) = 0$ enough to show
that $M(t)$ is a martingale?
\end{enumerate}

% % % % % % % % % % % % % % % % % % % % % % % % % % % % % % % % % % % % % %

\item Find the conditional probability that a standard Brownian motion
is not zero in the interval $(a,b]$ given that it is not zero in the
interval $(b,c]$. 
\begin{enumerate}
\item What is the limit as $c \to b$?  Does this make sense?
\item What is the limit as $c \to \infty$?  Does this make sense?
\end{enumerate}
(NOTE: in case you forgot to add the arc-tan rule to
you cheat sheet it says the probability of a standard Brownian motion
having at least one zero between $t$ and $t+s$ is $\frac{2}{\pi}
\arctan\sqrt{s/t}$.)

\item I have a scheme for buying and selling ``Bush'' futures
contracts.  (Recall the graph I passed out in class.)  Currently they
are trading for 60 points.  My scheme will make me money if over the
next 4 months the price goes over 70 points and then ends up at less
than 60 points at the end of the 4 months.  If this doesn't occur, I
will losing money.  Assume that the standard deviation over this time
period period is 20 points.  What is the probability of my scheme
making money?  (Give me a formula for the answer AND a crude guess as
to the numeric value of the probability.)


\item Let $\{N(t); t\ge 0\}$ be a Poisson process of rate $\lambda$,
representing the arrival process of customers entering a slow food
joint.  Each person takes exactly one hour to get their food, eat it
and leave.  Let $X(t)$ denote the number of customers remaining in the
store at time $t$ and $Y(t)$ be the number that have come and departed
by time $t$.
\begin{enumerate}
\item Find the joint distribution of $X(t)$ and $Y(t)$.
\item Is $Y(t)$ a Poisson process?  Justify your answer.
\end{enumerate}

\item What is the covariance function for a Poisson process?  In other
words, what is $Cov(N(t),N(s))$ where $N(t)$ is a Poisson process with
rate $\lambda$.

\item Consider a mink that is put in a cage that has one central room
and 6 side rooms.  The main room is called M, and the side rooms are
called A, B, C, D, E and F.  None of the side rooms are connected to
each other.  In other words, when the mink leaves a side room, he must
move back into the main room.

In each period the mink either stays put with probability p or moves
to one of the adjacent rooms with probability 1-p.  
\begin{enumerate}
\item Write down two different probabilistic models for this problem.
(One of the should have 7 states, and the other should have only two
states). 
\item Find the probability that the mink is in the main room.
\item Find the probability that the mink is in room A.
\item Suppose rooms A and B have interesting toys in them.  (Say room
A has a swimming pool--a favorite of minks, and room B has a traffic
cone--which is much less popular.)  Hence once the mink enters either
room A or room B, there is a much higher probability of staying than
in the other side rooms.  If possible, modify your models to
accommodate this change.
\end{enumerate}

\end{enumerate}
BONUS. Prove or disprove that $M(t)$ in problem 2 is a martingale.


\end{document}


