\documentclass[10pt,a4paper]{article}

\usepackage{color}

\begin{document}

\begin{flushleft}
Course No. Stat 433 \\
\today
\end{flushleft}

\begin{center}
{\Large{\bf  Homework 3 Solution}}
\end{center}

\textcolor[rgb]{0.98,0.00,0.00}{Comments from the grader:}
\begin{itemize}

    \item \textcolor[rgb]{0.98,0.00,0.00}{These are only partial solutions.  We selected
    questions which were problematic to most of the class.}
    \item \textcolor[rgb]{0.98,0.00,0.00}{The maximum grade for this homework assignment is 10. The points are NOT evenly distributed among the questions.}
    \item \textcolor[rgb]{0.98,0.00,0.00}{Your solution should contain explanations and not only
    final answers. Points will be deducted if partial solutions
    are submitted.}
    \item \textcolor[rgb]{0.98,0.00,0.00}{Please save a copy of your work and submit the original.
    Write your name and email on top of the first page.}
    \item \textcolor[rgb]{0.98,0.00,0.00}{if you notice a typo in the solution file or have a problem with the homework
    grading please email: sivana@wharton.upenn.edu
}
\end{itemize}


\begin{flushleft}


\textbf{Page 104 Question 2.1}

First notice that:

\[ \left( \begin{array}{cccc} 0.25 & 0.25 & 0.25 & 0.25
\end{array} \right)
 \left( \begin{array}{cccc}
0.4 & 0.3 & 0.2 & 0.1  \\
0.1 & 0.4 & 0.3 & 0.2  \\
0.3 & 0.2 & 0.1 & 0.4  \\
0.2 & 0.1 & 0.4 & 0.3  \\
\end{array} \right) =
 \left( \begin{array}{cccc} 0.25 & 0.25 & 0.25 & 0.25
\end{array} \right)\]

As a results of this fact it turns out that:

\[ \left( \begin{array}{cccc} 0.25 & 0.25 & 0.25 & 0.25
\end{array} \right)
 \left( \begin{array}{cccc}
0.4 & 0.3 & 0.2 & 0.1  \\
0.1 & 0.4 & 0.3 & 0.2  \\
0.3 & 0.2 & 0.1 & 0.4  \\
0.2 & 0.1 & 0.4 & 0.3  \\
\end{array} \right)^n =
 \left( \begin{array}{cccc} 0.25 & 0.25 & 0.25 & 0.25
 \end{array} \right)\]

for any natural $n\geq1$. This result holds since the transition
matrix is doubly stochastic. A doubly stochastic matrix is a
matrix that both its columns and rows sum up to 1. If the initial
distribution is the uniform distribution and the transition matrix
is  doubly stochastic then $P(X_n=k)=\frac{1}{m}$ where $m$ is the
number of states (for any state $k$) associated with the Markov
chain $X_n$.

\begin{eqnarray*}
\\
\end{eqnarray*}

\textbf{Page 104 Question 2.2}\\

We will solve this question using the same method explained in the
text book on pages 136-137. Since

\[ P = \left ( \begin{array}{cc}
 1-\alpha & \alpha \\
\alpha & 1-\alpha   \\
\end{array} \right) \]

it follows that

\[ P^n = \frac{1}{2 \cdot \alpha} \left ( \begin{array}{cc}
 \alpha & \alpha \\
\alpha & \alpha   \\
\end{array} \right)  + \frac{(1-2 \cdot \alpha)^n }{2 \cdot \alpha}
\left ( \begin{array}{cc}
 \alpha & -\alpha \\
-\alpha & \alpha   \\
\end{array} \right)\]


Hence, $P(X_5=0|X_0=0)=\frac{1+(1-2 \cdot \alpha)^5}{2}$.


\end{flushleft}
\end{document}
