\documentclass[12pt]{article}
%\usepackage{simplemargins}
\usepackage{enumerate}

\usepackage[usenames]{color}
%\newcommand{\answer}[1]{\noindent{\textcolor{blue}{\{{\bf answer:} \em #1\}}}}

\newcommand{\answer}[1]{

\noindent{{\{{\bf answer:} \em #1\}}}}
\setallmargins{.5in}
%\setallmargins{.5in}
%\settopmargin{1.2in}
\renewcommand{\baselinestretch}{1.2}
\pagestyle{empty}
\begin{document}
Dean Foster \hfill {\bf Stat 433: Midterm Exam} \hfill Feb 2008
\vspace{1em}

Instructions:
\begin{itemize}
\item Be sure to show your work.  It is great if you can guess the right
answer.  But to call it mathematics, you need to be able to show why
your guess is correct.  Hence provide justifications!
\item No calculators.
\item You may use a single page of notes (one side).
\end{itemize}

\begin{enumerate}
\item (25 pts) Recall the Poisson distribution, $p_\lambda(x) =
 e^{-\lambda} \lambda^x/x!$.  In case you have totally forgotten 430,
 here are some trivia, E$(X) = \lambda$ and Var$(X) = \lambda$.  If
 $X$ and $Y$ are Poisson random variables with parameters $\lambda$
 and $\mu$ respectively, then $X+Y$ is Poisson with parameter $\lambda
 + \mu$.  See, I told you I would remind you of the Poisson
 distribution!

At the CDC there are two epidemiologists, called Happy T. Harry, and
 Sally O. Sad.  Each of them are researching the impending bird flu
 epidemic.  They both agree that during the time that a person has the
 bird flu, and before that person dies, they will infect a Poisson
 number of other people.  They agree the correct model to use is a
 branching process with the number of births being a Poisson random
 variable with mean $\lambda$.
\begin{enumerate}
\item If we view this as a Markov chain, how many states does it have?
\answer{infinite}
\item Write down the first three rows of this chain.
\answer{
\begin{displaymath}
P = \left[\begin{array}{cccccc}
               1  & 0  & 0  &  0 & 0 & \ldots   \\
               e^{-\lambda}
               & e^{-\lambda}\lambda
               & e^{-\lambda}\lambda^2/2
               & e^{-\lambda}\lambda/3!
               & e^{-\lambda}\lambda/4!& \ldots      \\
               e^{-2\lambda}
               & e^{-2\lambda}2\lambda
               & e^{-2\lambda}(2\lambda)^2/2
               & e^{-2\lambda}(2\lambda)/3!
               & e^{-2\lambda}(2\lambda)/4!& \ldots      \\
               e^{-3\lambda}
               & e^{-3\lambda}3\lambda
               & e^{-3\lambda}(3\lambda)^2/2
               & e^{-3\lambda}(3\lambda)/3!
               & e^{-3\lambda}(3\lambda)/4!& \ldots      \\
         \end{array}
     \right]
\end{displaymath}
}
\end{enumerate}
Happy believe that the Poisson parameter is $\lambda = .95$, whereas
 Sally believes that the Poisson parameter is $\lambda = 1.01$.
\begin{enumerate}
\setcounter{enumii}{2}
\item If Happy is correct, what is the probability of the bird flu
going extinct?\answer{1}
\item What can we say if Sally is correct?  Describe as carefully as
you can.\answer{if $X_t$ is the number of bird flu's at time $t$, then
$P(\lim X_t = \infty) = P(lim X_t \ne 0) > 0$.  }
\item (Bonus) What does the ``T.'' stand for? \answer{Time!}
\end{enumerate}

\item (20 pts) Consider a sequence of coin tosses.
\begin{enumerate}
\item On average how long will it take to get the pattern H,T,H,T,H,T?
\answer{Consider a gambling team with one member arriving per letter,
each betting on the pattern in a doubling fashion.  Then they will
have exactly $2^2 + 2^4 + 2^6$ dollars when they end.  Since it is a
martingale, they must have placed $2^2 + 2^4 + 2^6$ bets as they went
along.  See Michael Steele's lecture for details.}
\item Suppose that the chance of getting a heads, is 2/3 and the
 chance of getting a tails is 1/3, then how long will it take to get
 H,H,T,H,H?
\answer{Again consider a gambling team.  If they are betting on heads,
they will win 3/2X and if they are betting on tails they will win 3X.
So they end up with $3/2 + (3/2)^2 + (3/2)^43$}
\end{enumerate}

\item (25 pts) Suppose an urn contains one red and one green ball.
  Each time a ball is removed, a coin is tossed and if it lands heads
 a blue ball is added.  So if the coin lands tails, the total number
 of balls decreases by one, and if it lands heads, the number of balls
 stays the same.  The process is stopped when we are out of balls.
\begin{enumerate}
\item (5 pts) List the possible states the system could be in if you
keep track of the colors of the balls.
\answer{RG, RB, GB, BB, R,G,B,empty}
\item (5 pts) Write down the transition matrix.

\answer{
\begin{displaymath}
P = \left[\begin{array}{cccccccccc}
   &  RG &  RB &  GB &  BB &   R  &   G &   B &   \emptyset\\
RG &   0 & .25 & .25 &  0  &  .25 & .25 &   0 &   0  \\
RB &   0 & .25 &   0 & .25 &  .25 &   0 & .25 &   0  \\
GB &   0 &   0 & .25 & .25 &  0   &   .25 & .25 &   0  \\
BB &   0 &   0 &  0  &  .5 &   0  &   0 &   .5&   0  \\
R  &   0 &   0 &  0  &  0  &   0  & 0  &   .5 &  .5  \\
G  &   0 &   0 &  0  &  0  &   0  & 0  &   .5 &  .5  \\
B  &   0 &   0 &  0  &  0  &   0  & 0  &   .5 &  .5  \\
\emptyset
   &   0 &   0 &  0  &   0   &   0 & 0 &   0 &   1
\end{array}
     \right]
\end{displaymath}
}
\item (0 pts) List the possible states the system if we just keep
track of the number of balls.
\answer{3}
\item (5 pts) Is the process Markov if the state is just the number of
 balls?  If so, write down the transition matrix.

\answer{True.
\begin{displaymath}
P = \left[\begin{array}{cccc}
\hbox{state} &  2 &  1 &   \emptyset \\
2        &  .5 &  .5 &   0 \\
1        &  0 &  .5  &  .5 \\
\emptyset&  0 &  0  &   1
\end{array}
     \right]
\end{displaymath}}
\item (5 pts) If we just keep track of whether the urn is empty or
not, is the process Markov?  If so, write down the transition matrix.
\answer{2 states, and it is not Markovian.}
\item (5 pts) How many rounds do we expect this process to run for
until the urn is empty?
\answer{Solving it backwards (using the first equations):
\begin{eqnarray*}
u_\emptyset &= &0 \\
u_B   &=& 1 + .5u_B + .5 u_\emptyset = 2 \\
u_G   &=& 1 + .5 u_B + .5 u_\emptyset = 2 \\
u_R   &=& 1 + .5 u_B + .5 u_\emptyset = 2 \\
u_{BB}&=& 1 + .5 u_{BB} + .5 u_B = 4 \\
u_{GB}&=& 1 + .25 u_{GB} + .25 u_{BB} + .5 u_B = 1 + .25 u_{GB}  +1 + 1 = 4\\
u_{RB}&=& 1 + .25 u_{RB} + .25 u_{BB} + .5 u_B = 1 + .25 u_{RB}  +1 + 1 = 4\\
u_{RG}&=& 1 + .25 u_{RB} + .25 u_{BB} + .5 u_R + .5 u_G = 1 + .25
u_{RB}  +1 + 1/2 + 1/2 = 4
\end{eqnarray*}
Using the numbered states is shorter:
\begin{eqnarray*}
u_\emptyset &= &0 \\
u_1   &=& 1 + .5u_1 + .5 u_0 = 2 \\
u_2   &=& 1 + .5 u_2 + .5 u_1 = 4
\end{eqnarray*}
}
\end{enumerate}
\item (20 pts) You consulting for company ABC.  You have a new scheme
for keeping their chip manufacturing plant running.  But, it doesn't
always work.  So the plant is in one of two states, either {\bf w}orking, or
{\bf b}roken.  You can also find your self {\bf u}nemployed if they
don't like your performance.  You model the three states as a Markov
chain (states are in the order {\bf w}, {\bf b}, {\bf u}):
\begin{displaymath}
P = \left[\begin{array}{ccc}
               .9  &  .1  &  0     \\
               .6  &  .2  &  .2     \\
                0   &  0   &  1
         \end{array}
     \right]
\end{displaymath}
To give you incentive, you earn 1 every time you are in {\bf w}
state, and lose 3 every time you are in state {\bf b}.
\begin{enumerate}
\item Which states are transient?  Which are absorbing?
\answer{{\bf b} and {\bf w} are transient, and {\bf u} is
absorbing.} \item Write down first step equations for the expected
time until you
 reach ${\bf u}$.
\answer{First step
\begin{eqnarray*}
v_w & = & 1 + .9 v_w + .1 v_b \\
v_b & = & 1 + .6 v_w + .2 v_b + .2 v_u \\
v_u & = & 0
\end{eqnarray*}
}
\item If the machine starts in the working state, what is the expected
amount that you earn?
\end{enumerate}
\answer{Matrixes would be easier here, but without them:
\begin{eqnarray*}
0 & = & 1 - .1 v_w + .1 v_b \\
0 & = & -3 + .6 v_w - .8 v_b \\
v_u & = & 0
\end{eqnarray*}

Solving the above equations leads to
\begin{eqnarray*}
v_w & = & 25\\
v_b & = & 15 \\
v_u & = & 0
\end{eqnarray*}
}

\item (10 pts) After you become unemployed consulting for ABC, you
pick up a consulting job for company XYZ.  But this time you decide to
make sure you don't ever get unemployed.  So due to careful
negotiations, you make sure that you will always be employed, hence
the following two state Markov chain with states {\bf w} and {\bf b}:
\begin{displaymath}
P = \left[\begin{array}{cc}
               .9  &  .1       \\
               .75  &  .25
         \end{array}
     \right]
\end{displaymath}
But the amount you are paid is discounted.  So if on day $i$ the
 machine is working, you get $s^{i}$ dollars, and if the machine is
 broken you lose $ -3 (s^{i})$.  The first day you work is day zero.
  So for example if the machine were in the states {\bf b},{\bf
 b},{\bf w},{\bf w},{\bf b},... then your total payment would be $-3
 -3s + s^2 + s^3 - 3s^4 + \ldots$.  The machine starts out in the
 broken state.  What is your expected total earnings?
\answer{First step
\begin{eqnarray*}
v_w & = & 1 + s(.9 v_w + .1 v_b) \\
v_b & = & -3 + s(3/4 v_w + 1/4 v_b)
\end{eqnarray*}
Now we solve it...
}
\end{enumerate}
\end{document}
