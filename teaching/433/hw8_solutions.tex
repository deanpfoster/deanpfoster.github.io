\documentclass[10pt,a4paper]{article}

\usepackage{color}

\begin{document}

\begin{flushleft}
Course No. Stat 433 \\
\today
\end{flushleft}

\begin{center}
{\Large{\bf  Homework 8 Solution}}
\end{center}

\textcolor[rgb]{0.98,0.00,0.00}{Comments from the grader:}
\begin{itemize}

    \item \textcolor[rgb]{0.98,0.00,0.00}{These are only partial solutions.  We selected
    questions which were problematic to most of the class or are of particular interest.}
    \item \textcolor[rgb]{0.98,0.00,0.00}{The maximum grade for this homework assignment is 10.}
    \item \textcolor[rgb]{0.98,0.00,0.00}{Your solution should contain explanations and not only
    final answers. Points will be deducted if partial solutions
    are submitted.}
    \item \textcolor[rgb]{0.98,0.00,0.00}{Please save a copy of your work and submit the original.
    Write your name and email on top of the first page.}
    \item \textcolor[rgb]{0.98,0.00,0.00}{if you notice a typo in the solution file or have a problem with the homework
    grading please email: sivana@wharton.upenn.edu
}
\end{itemize}


\begin{flushleft}

\begin{eqnarray*}
\\
\end{eqnarray*}


\textbf{Page 231 Question 2.2}

a.Let $X$ count the number of operating components and $Y$ an
indicator which takes on a value one when a component is repaired
for 1 day. The markov chain states are as follows:
\begin{enumerate}
    \item  State 1: $X=2,Y=0$
    \item  State 2: $X=1,Y=0$
    \item  State 3: $X=1,Y=1$
    \item  State 4: $X=0,Y=1$
    \item  State 5: $X=0,Y=0$
\end{enumerate}

The appropriate transition matrix is:
\[ P =   \left ( \begin{array}{ccccc}
 (1-\alpha)^2 & 2\alpha(1-\alpha) & 0 & 0 & \alpha^2 \\
 0 & 0 & 1-\beta & \beta & 0  \\
  1-\beta &  \beta& 0 & 0 & 0 \\
 0 & 0 & 1 & 0 & 0  \\
 0 & 0 & 0 & 1 & 0  \\
\end{array} \right) \]

b. Solve the following equations:
\begin{eqnarray*}
(1-\alpha)^2 \pi_0+ (1-\beta) \pi_2&=&\pi_0\\
2\alpha(1-\alpha) \pi_0+\beta \pi_2 + \pi_3 &=& \pi_1\\
(1-\beta) \pi_1 &=& \pi_2\\
\beta \pi_1+\pi_4&=&\pi_3\\
\alpha^2 \pi_0 &=& \pi_4 \\
\pi_0 + \pi_1 + \pi_2 + \pi_3 + \pi_4 &=&1
\end{eqnarray*}

The fraction of time the system is operating is
$\pi_0+\pi_1+\pi_2\approx0.95$.
\begin{eqnarray*}
\\
\end{eqnarray*}



\textbf{Page 233 Question 2.6}\\
We begin by proving the $X_n$ defines a Markov chain.

Consider the following 4 situations:
\begin{itemize}
    \item $P(X_{n+1}=0|X_0,\ldots,X_n=1)=P(X_{n+1}=0|X_n=1)=p$ from the
problem definition (''on a given day the computer fails with
probability p")
    \item $P(X_{n+1}=1|X_0,\ldots,X_n=1)=P(X_{n+1}=1|X_n=1)=q$
    using the same logic as before.
    \item $P(X_{n+1}=0|X_0,\ldots,X_n=0)=P(N>k+1|N>k)$ for some time k . Since N
    has a geometric distribution we know that
    $P(N>k+1|N>k)=P(N>1)=(1-\beta)=\alpha$. Since this is
    independent of k we arrive to the conclusion that
    $P(X_{n+1}=0|X_0,\ldots,X_n=0)=P(X_{n+1}=0|X_n=0)=\alpha$.
    \item
    $P(X_{n+1}=1|X_0,\ldots,X_n=0)=P(X_{n+1}=1|X_n=0)=\beta$ using the
    same logic as before.
\end{itemize}

Therefore, $X_n$ is a Markov Chain with $P$, the transition
matrix, which is indicated in the question. Since $\beta\in(0,1)$
$P$ must be regular and thus by solving $\pi P= \pi$ we can obtain
the limit distribution
$\pi=[\frac{p}{p+\beta},\frac{\beta}{p+\beta}]^{'}$.

\begin{eqnarray*}
\\
\end{eqnarray*}

\textbf{Page 243 Question 3.3}\\

a. The classes are:
\begin{enumerate}
    \item $\{0,2\}$ since $p_{0,2}>0$ and $p_{2,0}>0$ .
    \item $\{4,5\}$ since $p_{4,5}>0$ and $p_{5,4}>0$.
    \item $\{1,3\}$ since $p_{1,3}>0$ and $p_{3,1}>0$.
\end{enumerate}

b.The classes are:
\begin{enumerate}
    \item $\{1,2\}$ since $p_{1,2}>0$ and $p_{2,1}>0$ .
    \item $\{4,3\}$ since $p_{4,3}>0$ and $p_{3,4}>0$.
    \item $\{0\}$ and $\{5\}$ since they are both absorbing states.
\end{enumerate}


\begin{eqnarray*}
\\
\end{eqnarray*}


\textbf{Page 245 Question 3.1}\\

a.  \begin{eqnarray*}
f_{00}^{(0)}&=&0\\
f_{00}^{(1)}&=&1-a\\
f_{00}^{(n)}&=&a(1-b)^{n-2}b \textrm{ for $n>1$}\\
\end{eqnarray*}

b. \begin{eqnarray*} \sum_{k=1}^n f_{00}^{(k)} p_{00}^{(n-k)} &=&
(1-a) \frac{b+a(1-a-b)^{n-1}}{a+b} + \sum_{k=2}^n a(1-b)^{k-2}b
\cdot \frac{b+a(1-a-b)^{n-k}}{a+b}\\
&=& \frac{b}{a+b} + \frac{a(1-a-b)^{n}}{a+b}
\end{eqnarray*}

The last transition requires tedious algebra operations (using
definition of geometric sums) but you were expected to perform
them and get the last result.


\begin{eqnarray*}
\\
\end{eqnarray*}

\end{flushleft}
\end{document}
