\documentclass[10pt,a4paper]{article}

\usepackage{color}

\begin{document}

\begin{flushleft}
Course No. Stat 433 \\
\today
\end{flushleft}

\begin{center}
{\Large{\bf  Homework 9 Solution}}
\end{center}

\textcolor[rgb]{0.98,0.00,0.00}{Comments from the grader:}
\begin{itemize}

    \item \textcolor[rgb]{0.98,0.00,0.00}{These are only partial solutions.  We selected
    questions which were problematic to most of the class or are of particular interest.}
    \item \textcolor[rgb]{0.98,0.00,0.00}{The maximum grade for this homework assignment is 10.}
    \item \textcolor[rgb]{0.98,0.00,0.00}{Your solution should contain explanations and not only
    final answers. Points will be deducted if partial solutions
    are submitted.}
    \item \textcolor[rgb]{0.98,0.00,0.00}{Please save a copy of your work and submit the original.
    Write your name and email on top of the first page.}
    \item \textcolor[rgb]{0.98,0.00,0.00}{if you notice a typo in the solution file or have a problem with the homework
    grading please email: sivana@wharton.upenn.edu
}
\end{itemize}


\begin{flushleft}

\begin{eqnarray*}
\\
\end{eqnarray*}


\textbf{Page 256 Question 4.3}

\begin{eqnarray*}
\pi_0&=&q_1 \pi_1 \\
\pi_1&=&\pi_0+q_2 \pi_2\\
\pi_2&=&p_1\pi_1+q_3\pi_3\\
\end{eqnarray*}

From these three questions we can see that the following pattern
is created:

\begin{eqnarray*}
\pi_1&=&\frac{1}{q_1} \pi_0 \\
\pi_2&=&\frac{p_1}{q_1q_2} \pi_0\\
\pi_3&=&\frac{p_1p_2}{q_1q_2q_3} \pi_0\\
\end{eqnarray*}

Generalizing these equations yields the following conclusion


\begin{eqnarray*}
\pi_i&=&\frac{\pi_0}{p_i} \prod_{j=1}^{i} \frac{p_j}{q_j}
\end{eqnarray*}

Integrating the fact that $\sum_{i=0}^N \pi_i =1$ we can figure
out $\pi_0$.

\begin{eqnarray*}
\\
\end{eqnarray*}



\textbf{Question 4.8}\\
The transition matrix is as follows:

\[ P = \left ( \begin{array}{ccccc}
 \frac{1}{2} &\frac{1}{2} & 0 & 0 & \ldots  \\
 \frac{1}{3} & \frac{1}{3} & \frac{1}{3}& 0  & \ldots \\
 \frac{1}{4 }& \frac{1}{4 } & \frac{1}{4 } & \frac{1}{4 } & 0\\
 \vdots &  \vdots &  \vdots &  \vdots &  \vdots
\end{array} \right) \]

Using the first few equations we conclude that the following
holds:

\begin{eqnarray*}
\pi_1&=& \pi_0 \\
\pi_2&=&\pi_1-\frac{1}{2}\pi_0 = \frac{1}{2}\pi_0\\
\pi_3&=&\pi_2-\frac{1}{3}\pi_1 = \frac{1}{6}\pi_0\\
\pi_4&=&\pi_3-\frac{1}{4}\pi_2 = \frac{1}{24}\pi_0\\
\end{eqnarray*}

Generally we can prove that $\pi_i=\frac{1}{i!}\pi_0$ using the
induction method.  Using the fact that the sum over all $\pi_i$
must be one and the Taylor series expansion for $e$ we get

\begin{eqnarray*}
\sum_{i=0}^{\infty} \pi_{i} &=& 1 \\
\pi_0 \sum_{i=0}^{\infty} \frac{1}{i!} &=& 1 \\
\pi_0 \cdot e &=& 1\\
\pi_0 &=&\frac{1}{e}
\end{eqnarray*}


\begin{eqnarray*}
\\
\end{eqnarray*}


\textbf{Question 1.1}\\
\begin{eqnarray*}
P(X=k)&=&\int_0^1 f(X=k|\xi_1+\ldots+\xi_k=x)f(\xi_1+\ldots+\xi_k=x)dx \\
&=&\int_0^1 \frac{f(X=k,\xi_1+\ldots+\xi_k=x)}{f(\xi_1+\ldots+\xi_k=x)}f(\xi_1+\ldots+\xi_k=x)dx \\
&=&\int_0^1 \frac{f(\xi_1+\ldots+\xi_k=x,\xi_1+\ldots+\xi_{k+1}>1)}{f(\xi_1+\ldots+\xi_k=x)}f(\xi_1+\ldots+\xi_k=x)dx \\
&=&\int_0^1 \frac{f(\xi_1+\ldots+\xi_k=x,\xi_{k+1}>1-x)}{f(\xi_1+\ldots+\xi_k=x)}f(\xi_1+\ldots+\xi_k=x)dx \\
&=&\int_0^1 \frac{f(\xi_1+\ldots+\xi_k=x)P(\xi_{k+1}>1-x)}{f(\xi_1+\ldots+\xi_k=x)}f(\xi_1+\ldots+\xi_k=x)dx \\
&=&\int_0^1 P(\xi_{k+1}>1-x)f(\xi_1+\ldots+\xi_k=x)dx \\
&=&\int_0^1 (1-F(1-x))f(\xi_1+\ldots+\xi_k=x)dx \\
&=&\int_0^1 (e^{-\lambda(1-x)})\frac{\lambda^k x^{k-1} e^{-\lambda x}}{(k-1)!}dx \\
&=&\frac{\lambda^k e^{-\lambda}}{(k-1)!} \int_0^1  x^{k-1} dx \\
&=&\frac{\lambda^k e^{-\lambda}}{k!}\\
\end{eqnarray*}



\begin{eqnarray*}
\\
\end{eqnarray*}

\textbf{Question 1.7}\\
Let $X_t$ be a random variable counting the number of shocks up to
time t.  Then it follows that the probability that the system is
surviving at time t is:

\begin{eqnarray*}
&=&\sum_{i=0}^{\infty} \alpha^i P(X_t=i)\\
&=&\sum_{i=0}^{\infty} \alpha^i \frac{(\lambda t)^i e^{-\lambda t}
}{i!}\\
&=& e^{-\lambda t} \sum_{i=0}^{\infty} \frac{(\alpha\lambda t)^i}
{i!}\\
&=&e^{-\lambda t} e^{\alpha\lambda t} \\
&=&e^{-\lambda t (1- \alpha)} \\
\end{eqnarray*}

\begin{eqnarray*}
\\
\end{eqnarray*}

\end{flushleft}
\end{document}
