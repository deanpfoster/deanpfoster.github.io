\documentclass[10pt,a4paper]{article}

\usepackage{color}

\begin{document}

\begin{flushleft}
Course No. Stat 433 \\
\today
\end{flushleft}

\begin{center}
{\Large{\bf  Homework 9 Solution}}
\end{center}

\textcolor[rgb]{0.98,0.00,0.00}{Comments from the grader:}
\begin{itemize}
    \item \textcolor[rgb]{0.98,0.00,0.00}{These are only partial solutions.  We selected
    questions which were the most problematic for the class.}
    \item \textcolor[rgb]{0.98,0.00,0.00}{The maximum grade for this homework assignment is 10.}
    \item \textcolor[rgb]{0.98,0.00,0.00}{Your solution should contain explanations, and not just
    final answers. Points will be deducted if partial solutions.
    are submitted.}
    \item \textcolor[rgb]{0.98,0.00,0.00}{Make sure that your work is readable/understandable.  If necessary, skip every other line.  Clearly circle your answers.  In addition, please {\bf staple} your homework.}
    \item \textcolor[rgb]{0.98,0.00,0.00}{If you notice a typo in the solution file or have a problem with the homework
    grading, please come by my office hours, or email me (entine4@wharton.upenn.edu)}

\end{itemize}


\begin{flushleft}

\begin{eqnarray*}
\\
\end{eqnarray*}




\textbf{Question 1.1}\\
\begin{eqnarray*}
P(X=k)&=&\int_0^1 f(X=k|\xi_1+\ldots+\xi_k=x)f(\xi_1+\ldots+\xi_k=x)dx \\
&=&\int_0^1 \frac{f(X=k,\xi_1+\ldots+\xi_k=x)}{f(\xi_1+\ldots+\xi_k=x)}f(\xi_1+\ldots+\xi_k=x)dx \\
&=&\int_0^1 \frac{f(\xi_1+\ldots+\xi_k=x,\xi_1+\ldots+\xi_{k+1}>1)}{f(\xi_1+\ldots+\xi_k=x)}f(\xi_1+\ldots+\xi_k=x)dx \\
&=&\int_0^1 \frac{f(\xi_1+\ldots+\xi_k=x,\xi_{k+1}>1-x)}{f(\xi_1+\ldots+\xi_k=x)}f(\xi_1+\ldots+\xi_k=x)dx \\
&=&\int_0^1 \frac{f(\xi_1+\ldots+\xi_k=x)P(\xi_{k+1}>1-x)}{f(\xi_1+\ldots+\xi_k=x)}f(\xi_1+\ldots+\xi_k=x)dx \\
&=&\int_0^1 P(\xi_{k+1}>1-x)f(\xi_1+\ldots+\xi_k=x)dx \\
&=&\int_0^1 (1-F(1-x))f(\xi_1+\ldots+\xi_k=x)dx \\
&=&\int_0^1 (e^{-\lambda(1-x)})\frac{\lambda^k x^{k-1} e^{-\lambda x}}{(k-1)!}dx \\
&=&\frac{\lambda^k e^{-\lambda}}{(k-1)!} \int_0^1  x^{k-1} dx \\
&=&\frac{\lambda^k e^{-\lambda}}{k!}\\
\end{eqnarray*}



\begin{eqnarray*}
\\
\end{eqnarray*}

\textbf{Question 1.7}\\
Let $X_t$ be a random variable counting the number of shocks up to
time t.  Then it follows that the probability that the system is
surviving at time t is:

\begin{eqnarray*}
&=&\sum_{i=0}^{\infty} \alpha^i P(X_t=i)\\
&=&\sum_{i=0}^{\infty} \alpha^i \frac{(\lambda t)^i e^{-\lambda t}
}{i!}\\
&=& e^{-\lambda t} \sum_{i=0}^{\infty} \frac{(\alpha\lambda t)^i}
{i!}\\
&=&e^{-\lambda t} e^{\alpha\lambda t} \\
&=&e^{-\lambda t (1- \alpha)} \\
\end{eqnarray*}

\begin{eqnarray*}
\\
\end{eqnarray*}

\end{flushleft}
\end{document}
