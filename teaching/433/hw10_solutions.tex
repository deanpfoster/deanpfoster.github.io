\documentclass[10pt,a4paper]{article}

\usepackage{color}

\begin{document}

\begin{flushleft}
Course No. Stat 433 \\
\today
\end{flushleft}

\begin{center}
{\Large{\bf  Homework 10 Solution}}
\end{center}

\textcolor[rgb]{0.98,0.00,0.00}{Comments from the grader:}
\begin{itemize}
    \item \textcolor[rgb]{0.98,0.00,0.00}{These are only partial solutions.  We selected
    questions which were the most problematic for the class.}
    \item \textcolor[rgb]{0.98,0.00,0.00}{The maximum grade for this homework assignment is 10.}
    \item \textcolor[rgb]{0.98,0.00,0.00}{Your solution should contain explanations, and not just
    final answers. Points will be deducted if partial solutions.
    are submitted.}
    \item \textcolor[rgb]{0.98,0.00,0.00}{Make sure that your work is readable/understandable.  If necessary, skip every other line.  Clearly circle your answers.  In addition, please {\bf staple} your homework.}
    \item \textcolor[rgb]{0.98,0.00,0.00}{If you notice a typo in the solution file or have a problem with the homework
    grading, please come by my office hours, or email me (entine4@wharton.upenn.edu)}

\end{itemize}



\begin{flushleft}

\begin{eqnarray*}
\\
\end{eqnarray*}


\textbf{Question 2.4}

Each point falls in the interval [0,1) with probability $1/N$ and
we have $N$ points. The number of points in the interval is
distributed binomially with parameters ($1/N$,$N$). Using the law
of rare events we can conclude that as $N \rightarrow \infty$  the
number of points in the interval $S_N$ follows a Poisson
distribution with $\lambda=\frac{1}{N}\cdot N=1$.


\begin{eqnarray*}
\\
\end{eqnarray*}

\textbf{Question 2.10}\\


We know that

\begin{eqnarray*}
P(E(p)=X(p))\geq 1-p^2  &\Rightarrow&  P(E(p)\neq X(p))\leq p^2
\end{eqnarray*}

Using equation 2.8 on page 285 in the book we know that for all
$k\in I$ the following is true

\begin{eqnarray*}
|P(S_n=k)-P(X(\mu)=k)|\leq \sum_{k=1}^n P(E(p_k)\neq X(p_k))\leq
\sum_{k=1}^n p_k^2
\end{eqnarray*}

Since this holds for all $k\in I$ the desired result holds.


\begin{eqnarray*}
\\
\end{eqnarray*}

\textbf{Question 2.11}\\


First notice that $\{X \in B\}=\{X \in B\ \cap Y \in B\}\cup \{X
\in B\ \cap Y \not \in B\}$. Using the same logic we know that
$\{Y \in B\}=\{Y \in B\ \cap X \in B\}\cup \{Y \in B\ \cap X \not
\in B\}$. Hence,


\begin{eqnarray*}
|P(X \in B)-P(Y \in B)|&&\\
&=& | P(X \in B) + P(Y \in B) - P(X\in B)-  P(Y \not \in B)|\\
&=& |P(X \in B\ \cap Y \in B) + P(X \in B\ \cap Y \not \in B)\\
&-&P(Y \in B\ \cap X \in B) - P(Y \in B\ \cap X \not \in B)\\
&=& |P(X \in B\ \cap Y \not \in B)- P(Y \in B\ \cap X \not \in B)\\
\end{eqnarray*}

Since $\{X \in B\ \cap Y \not \in B\} \subseteq  \{X \neq Y\}$ the
desired result follows.


\begin{eqnarray*}
\\
\end{eqnarray*}


\textbf{Question 3.3}\\
The two dimensional transformation of variables formula is
\begin{eqnarray*}
f_{S_0,S_1}(s_0,s_1)=f_{W_0,W_1}(w_0,w_1)det(J).
\end{eqnarray*}

where J is the Jacobain matrix of $(S_0,S_1)$ as a function of
$(W_0,W_1)$. In this example the determinant of the Jacobain is 1
and as a result the density is $\lambda^2 e^{-\lambda
(s_0+s_1)}=\lambda e^{-\lambda s_0} \cdot \lambda e^{-\lambda
s_1}$. Hence we see that it is just the joint distribution of two
independent exponential random variables.



\begin{eqnarray*}
\\
\end{eqnarray*}

\textbf{Question 3.8}\\

\begin{eqnarray*}
P(W_r=x|X(t)=n)&=& \frac{\frac{\lambda^r x^{r-1} e^{-\lambda
x}}{(r-1)!} \cdot \frac{(\lambda(t-x))^{n-r}
e^{-\lambda(t-x)}}{(n-r)!}}{ \frac{(\lambda t)^{n} e^{-\lambda
t}}{n!}}\\
&=& \left ( \begin{array}{c}
 n \\
 r
\end{array} \right) \cdot \frac{r}{t} (\frac{x}{t})^{r-1}
(1-\frac{x}{t})^{n-r}
\end{eqnarray*}

A different that arrives to the same solution is to start from the
conditional cumulative distribution function and take the
derivative with respect to x.


\begin{eqnarray*}
\\
\end{eqnarray*}

\textbf{Question 4.6c}\\
\begin{eqnarray*}
P(W_2<w)=1-P(\mbox{All Ui's } >w)-P(\mbox{exactly 1 Ui } > w)\\
=1 - \frac{(1-w)^5}{t^5} - \frac{5w(1-w)^4}{t^5}\\
\end{eqnarray*}
Differentiating with respect to w gives us the density function
\[f(w)=\frac{20w(1-w)^4}{t^5}\]


\textbf{Question 4.8}\\

Using the iterated expectation rule we know that
$E(Z(t))=E(E(Z(t)|N(t)))$. First we will find $E(Z(t)|N(t))$.

\begin{eqnarray*}
E(Z(t)|N(t))&=& E(\sum_{k=1}^{N(t)} \theta_k (t)) \\
&=& \sum_{k=1}^{N(t)} E(\theta_k (t)) \\
&=& \sum_{k=1}^{N(t)} E(\xi_k e^{-\alpha (t-w_k)}) \\
&=& E(\xi_1) \sum_{k=1}^{N(t)} E(e^{-\alpha (t-w_k)}) \\
&=& E(\xi_1) \frac{N(t)}{\alpha t} \cdot (1-e^{-\alpha t}) \\
\end{eqnarray*}

Hence, $E(Z(t))=E( E(\xi_1) \frac{N(t)}{\alpha t} \cdot
(1-e^{-\alpha t}) )=\frac{\lambda  E(\xi_1)}{\alpha }
(1-e^{-\alpha t}) $

\begin{eqnarray*}
\\
\end{eqnarray*}

\end{flushleft}
\end{document}
