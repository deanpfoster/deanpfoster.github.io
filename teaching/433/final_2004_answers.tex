% ----------------------------------------------------------------
% AMS-LaTeX Paper ************************************************
% **** -----------------------------------------------------------
\documentclass{amsart}
\usepackage{graphicx}
% ----------------------------------------------------------------
\vfuzz2pt % Don't report over-full v-boxes if over-edge is small
\hfuzz2pt % Don't report over-full h-boxes if over-edge is small
% THEOREMS -------------------------------------------------------
\newtheorem{thm}{Theorem}[section]
\newtheorem{cor}[thm]{Corollary}
\newtheorem{lem}[thm]{Lemma}
\newtheorem{prop}[thm]{Proposition}
\theoremstyle{definition}
\newtheorem{defn}[thm]{Definition}
\theoremstyle{remark}
\newtheorem{rem}[thm]{Remark}
\numberwithin{equation}{section}
% MATH -----------------------------------------------------------
\newcommand{\norm}[1]{\left\Vert#1\right\Vert}
\newcommand{\abs}[1]{\left\vert#1\right\vert}
\newcommand{\set}[1]{\left\{#1\right\}}
\newcommand{\Real}{\mathbb R}
\newcommand{\eps}{\varepsilon}
\newcommand{\To}{\longrightarrow}
\newcommand{\BX}{\mathbf{B}(X)}
\newcommand{\A}{\mathcal{A}}
% ----------------------------------------------------------------
\begin{document}

\title{STAT 433: Final Exam Solutions}

\date{ \today}




% ----------------------------------------------------------------

\maketitle
% ----------------------------------------------------------------
\flushleft
{\bf QUESTION 1}
\newline
{\bf (a)}
Transient states $\left\{1\right\}$ \\
{\bf (b)}
Periodic states $\left\{3,5\right\}$ \\
{\bf (c)} Communicating classes $\left\{2,4\right\}$ and
$\left\{3,5\right\}$ \\
 {\bf (d)}
 $\pi_2=\pi_4=1/2$
\bigskip

{\bf QUESTION 2}
\newline
{\bf (a)}
\newline
\begin{eqnarray}
\nonumber E[M(t)]&=&E[N(t)-f(t)] \\
\nonumber 0&=&E[X(t)^2+Y(t)^2+Z(t)^2]-E[f(t)] \\
\nonumber E[f(t)]&=&t+t+t=3t \\
\nonumber &\Downarrow& \\
\nonumber f(t)&=&3t
\end{eqnarray}
 {\bf (b)}
\newline
$E[M(t)]=0$ is not enough to show that M(t) is a martingale. In
order to show it is a martingale, we also need to show
\begin{eqnarray}
\nonumber E[M(t)|M(t-1),M(t-2),\ldots,M(0)]&=&M(t-1)
\end{eqnarray}
\bigskip

{\bf QUESTION 3}
\newline
\begin{eqnarray}
\nonumber P\left\{\underbrace{B(t)\neq 0, a<t\leq b}_{A}
|\underbrace{ B(t)\neq 0, b<t\leq c}_{B}\right\} &=&
\frac{P\left\{\overbrace{B(t)\neq 0, a<t\leq b}^{A} \cap
\overbrace{B(t)\neq 0,
b<t\leq c}^{B}\right\}}{P\left\{\underbrace{B(t)\neq 0, b<t\leq c}_{B} \right\}} \\
\nonumber &=& \frac{1-2/\pi \arctan\sqrt{\frac{c-a}{a}}}{1-2/\pi
\arctan\sqrt{\frac{c-b}{b}}}
\end{eqnarray}
 {\bf (a)}
\newline
\begin{eqnarray}
\nonumber \lim_{c\rightarrow b}\frac{1-2/\pi
\arctan\sqrt{\frac{c-a}{a}}}{1-2/\pi
\arctan\sqrt{\frac{c-b}{b}}}&=&1-2/\pi \arctan\sqrt{\frac{b-a}{a}}
\end{eqnarray}
since $1-2/\pi \arctan(0) = 1$
\newline
{\bf (b)}
\begin{eqnarray}
\nonumber \lim_{c\rightarrow \infty}\frac{1-2/\pi
\arctan\sqrt{\frac{c-a}{a}}}{1-2/\pi
\arctan\sqrt{\frac{c-b}{b}}}&=&\sqrt{\frac{a}{b}}
\end{eqnarray}
using l'Hopitals rule. \newline \newline
\bigskip
{\bf QUESTION 4}
\newline
{\bf (a)}
\newline
In general,
\begin{eqnarray}
\nonumber P(\max_{0<s<t}B(s)\geq z, B(t) \leq x)&=&P(B(t)\geq
2z-x)\\
\nonumber &=&1-\Phi\left(\frac{2z-x}{\sqrt{t}}\right) \\
\nonumber z=\frac{70-60}{20}=.5 &,&x=\frac{60-60}{20}=0,t=1 \text{ time unit defined as 4 months}\\
\nonumber P(\text{Make
money})&=&1-\Phi\left(2(.5)-0\right)=1-\Phi\left(1\right)
\end{eqnarray}
\newline
Alternatively, consider our path of interest as the reflected
Brownian motion at 70. This unreflected path would be above 80 to
be like our "above 70 and then back below 60" path. So  P(Make
money)=1-$\Phi(\frac{80-60}{20})=1-\Phi(1)$ which is the same as
before. \newline \newline
\bigskip
{\bf QUESTION 5}
\newline
{\bf (a)}
\begin{eqnarray}
\nonumber && \text{Case 1: } t \leq 1 \\
\nonumber && X(t)=N(t) \\
\nonumber && Y(t)=0 \\
\nonumber && \text{Case 2: } t \geq 1 \\
\nonumber && X(t)=N(t)-N(t-1) \sim \text{Pois}(\lambda) \\
\nonumber && Y(t)=N(t-1) \sim \text{Pois}((t-1)\lambda)
\end{eqnarray}
\newline
$X(t)$ and $Y(t)$ are independent. \newline
\newline
{\bf (b)}
\newline
This was a typo. The intended question was is X(t) a Poisson
variable. The answer is no. Whether or not Y(t) is a Poisson
variable is hard to tell so any answer is correct. \newline
\newline
\bigskip
{\bf QUESTION 6}
\newline
Assuming $0\leq s \leq t$ \newline
\begin{eqnarray}
\nonumber \text{Cov}[N(t),N(s)]&=&E[N(t)N(s)]-\lambda t \lambda
s\\
\nonumber &=&E[N(t)N(s)]-\lambda^2ts \\
\nonumber \\
\nonumber
E[N(t)N(s)]&=&E[{N(t)-N(s)+N(s)}N(s)]=E[{N(t)-N(s)}N(s)]+E[N(s)^2]
\\
\nonumber E[N(s)^2]&=&Var[N(s)]+(E[N(s)])^2= \lambda s + (\lambda
s)^2 \\
\nonumber
E[{N(t)-N(s)}N(s)]&=&E[N(t)-N(s)]E[N(s)]=\lambda(t-s)\lambda s
\end{eqnarray}
combining all terms, we have
\begin{eqnarray}
 \nonumber
\text{Cov}[N(t),N(s)]&=&\lambda(t-s)\lambda s + \lambda s +
\lambda^2 s^2 -\lambda t \lambda s =\lambda s
\end{eqnarray}
\newline
Similarly for $0\leq t \leq s$ we have
$\text{Cov}[N(t),N(s)]=\lambda t$, hence
\begin{eqnarray}
 \nonumber
\text{Cov}[N(t),N(s)]&=&\lambda \min(t,s)
\end{eqnarray}
\newline

\bigskip
{\bf QUESTION 7}
\newline
{\bf (a)}
\begin{eqnarray}
\nonumber \text{Model 1} &&\left[
\begin{array}[4]{ccccccc}
p & \frac{1-p}{6} & \frac{1-p}{6} & \frac{1-p}{6} & \frac{1-p}{6} & \frac{1-p}{6} & \frac{1-p}{6}  \\
1-p & p & 0 & 0 & 0 & 0 & 0 \\
1-p & 0 & p & 0 & 0 & 0 & 0 \\
1-p & 0 & 0 & p & 0 & 0 & 0 \\
1-p & 0 & 0 & 0 & p & 0 & 0 \\
1-p & 0 & 0 & 0 & 0 & p & 0 \\
1-p & 0 & 0 & 0 & 0 & 0 & p \\
\end{array}
\right]
\end{eqnarray}
\begin{eqnarray}
\nonumber \text{Model 2} &&\left[
\begin{array}[4]{cc}
p & 1-p  \\
1-p & p \\
\end{array}
\right]
\end{eqnarray}
\newline
{\bf (b)} Inspection of Model 2 tells us $\pi_{\text{main}}=1/2$
\newline
{\bf (c)} Probability of not being in the main room is 1/2 and all
six rooms are exchangeable so $1/2 \times 1/6 = 1/12$=P(mink in
room A). \newline
{\bf(d)}
\begin{eqnarray}
\nonumber \text{Model 1} &&\left[
\begin{array}[4]{ccccccc}
p & \frac{1-p}{6} & \frac{1-p}{6} & \frac{1-p}{6} & \frac{1-p}{6} & \frac{1-p}{6} & \frac{1-p}{6}  \\
1-p-\epsilon_1 & p+\epsilon_1 & 0 & 0 & 0 & 0 & 0 \\
1-p-\epsilon_2 & 0 & p+\epsilon_2 & 0 & 0 & 0 & 0 \\
1-p & 0 & 0 & p & 0 & 0 & 0 \\
1-p & 0 & 0 & 0 & p & 0 & 0 \\
1-p & 0 & 0 & 0 & 0 & p & 0 \\
1-p & 0 & 0 & 0 & 0 & 0 & p \\
\end{array}
\right]
\end{eqnarray}
\end{document}
