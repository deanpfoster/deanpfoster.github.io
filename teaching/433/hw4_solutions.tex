\documentclass[10pt,a4paper]{article}

\usepackage{color}

\begin{document}

\begin{flushleft}
Course No. Stat 433 \\
\today
\end{flushleft}

\begin{center}
{\Large{\bf  Homework 4 Solution}}
\end{center}

\textcolor[rgb]{0.98,0.00,0.00}{Comments from the grader:}
\begin{itemize}
    \item \textcolor[rgb]{0.98,0.00,0.00}{These are only partial solutions.  We selected
    questions which were the most problematic for the class.}
    \item \textcolor[rgb]{0.98,0.00,0.00}{The maximum grade for this homework assignment is 10.}
    \item \textcolor[rgb]{0.98,0.00,0.00}{Your solution should contain explanations, and not just
    final answers. Points will be deducted if partial solutions.
    are submitted.}
    \item \textcolor[rgb]{0.98,0.00,0.00}{Make sure that your work is readable/understandable.  If necessary, skip every other line.  Clearly circle your answers.  In addition, please {\bf staple} your homework.}
    \item \textcolor[rgb]{0.98,0.00,0.00}{If you notice a typo in the solution file or have a problem with the homework
    grading, please come by my office hours, or email me (entine4@wharton.upenn.edu)}

\end{itemize}



\begin{flushleft}

\begin{eqnarray*}
\\
\end{eqnarray*}


\textbf{Page 130 Question 4.1}

Our Markov Chain consists of 4 states:
\begin{enumerate}
    \item State 0: All the flips up till now cannot possibly lead to the pattern
    HHT. If the next coin flip is a H we progress to state 1 else
    we stay in this state.
    \item State 1: The current flip is H. If the next flip is H we
    continue to state 2. If the next flip is T we go back to state
    0.
    \item State 2: The current flip is H and the flip before was
    H. If the next flip is T we continue to state 2. If the next flip
    is H we stay in this state.
    \item State 3: The current flip is T and the last two flips
    were H. Once it reaches state 3 the Markov Chain remains in this state.
\end{enumerate}
The appropriate transition matrix is:
\[ P = \left ( \begin{array}{cccc}
 0.5 & 0.5 & 0 & 0  \\
 0.5 & 0 & 0.5 & 0  \\
 0 & 0 & 0.5 & 0.5  \\
 0.5 & 0.5 & 0 & 1  \\
\end{array} \right) \]

Let $T=min\{n \geq 0 ; X_n=3\}$ and $u_i=E(T|X_0=i)$. The first
step analysis yields the following equations:

\begin{eqnarray*}
v_0=1+0.5v_0+0.5v_1 \\
v_1=1+0.5v_0+0.5v_2 \\
v_2=1+0.5v_2+0.5v_3 \\
\end{eqnarray*}

If we start at state 3 then the mean time until we first get these
is 0, hence $v_3=0$. Solving the above system of equations reveals
that $v_0=8$.


A similar Markov Chain can be defined for the second part of the
question in the following manner.

\begin{enumerate}
    \item State 0: All the flips up till now cannot possibly lead to the pattern
    HTH. If the next coin flip is a H we progress to state 1 else
    we stay in this state.
    \item State 1: The current flip is H. If the next flip is T we
    continue to state 2. If the next flip is H we stay in state
    1.
    \item State 2: The current flip is T and the flip before was
    H. If the next flip is H we continue to state 2. If the next flip
    is T we go back to state 0.
    \item State 3: The current flip is H and the last two flips
    were HT. Once it reaches state 3 the Markov Chain remains in this state.
\end{enumerate}
The appropriate transition matrix is:
\[ P = \left ( \begin{array}{cccc}
 0.5 & 0.5 & 0 & 0  \\
 0 & 0.5 & 0.5 & 0  \\
 0.5 & 0 & 0 & 0.5  \\
 0.5 & 0.5 & 0 & 1  \\
\end{array} \right) \]

Let $T=min\{n \geq 0 ; X_n=3\}$ and $u_i=E(T|X_0=i)$. The first
step analysis yields the following equations:

\begin{eqnarray*}
v_0=1+0.5v_0+0.5v_1 \\
v_1=1+0.5v_1+0.5v_2 \\
v_2=1+0.5v_0+0.5v_3 \\
\end{eqnarray*}

If we start at state 3 then the mean time until we first get these
is 0, hence $v_3=0$. Solving the above system of equations reveals
that $v_0=10$.

From this analysis one can see that the average time till we get a
pattern of HHT is shorter than the average time till we reach HTH.
This makes sense because once the first Markov Chain reaches state
2 it either stays there or advances while the second Markov Chain
can return to state 0 even when it reaches state 2.
\begin{eqnarray*}
\\
\end{eqnarray*}

\textbf{Page 132 Question 4.7}\\
 \begin{eqnarray*}
h_i &=& \sum_{n=0}^{\infty} E[\beta^n c(X_n) |X_0=i]\\
&=& c(i) + \sum_{n=1}^{\infty} E[\beta^n c(X_n) |X_0=i]\\
&=& c(i) + \sum_{n=1}^{\infty} \sum_j E[\beta^n c(X_n) |X_0=i, X_1=j] P(X_1=j|X_0=i)\\
&=& c(i) + \sum_{n=1}^{\infty} \sum_j E[\beta^n c(X_n) |X_1=j] P_{ij}\\
&=& c(i) + \sum_j  \beta \sum_{n=1}^{\infty}  E[\beta^{n-1} c(X_n) |X_1=j] p_{ij}\\
&=& c(i) + \sum_j  \beta \sum_{m=0}^{\infty}  E[\beta^{m} c(X_m) |X_0=j] p_{ij}\\
&=& c(i) + \sum_j  \beta \sum_{m=0}^{\infty}  h_j p_{ij}\\
 \end{eqnarray*}

where $m=n-1$ and we just use $m$ as a dummy index.



\begin{eqnarray*}
\\
\end{eqnarray*}

\textbf{Page 134 Question 4.17}\\
Define $u_i=E(s^T|X_0=i)$ then we need to find $u_0$. Using first
step analysis we have the following equations:
\begin{eqnarray*}
u_0 &=& 0.7 s u_0 + 0.3 s u_1\\
u_1 &=& 0.6 s u_1 + 0.4 s u_2\\
\end{eqnarray*}

Notice that $u_2=1$ since $E(s^T|X_0=2)=s^0=1$.

By solving the above set of equations we can conclude that
\begin{eqnarray*}
u_0 = \frac{0.12 s^2}{(1-0.7s)(1-0.6s)}
\end{eqnarray*}

Some students wrote that $E(g(X))=g(E(X))$ but this is not
generally true unless the function $g(X)$ is a linear function. In
our case $g(X)=s^X$ which is not a linear function.

\end{flushleft}
\end{document}
