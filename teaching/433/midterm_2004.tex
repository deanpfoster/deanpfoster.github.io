\documentclass[12pt]{article}
\usepackage{simplemargins}
\setallmargins{.5in}
\settopmargin{1.2in}
\renewcommand{\baselinestretch}{1.3}
\pagestyle{empty}
\begin{document}
Dean Foster \hfill {\bf Stat 433: Midterm Exam} \hfill\today
\vspace{1em}

Be sure to show your work.  It is great if you can guess the right
answer.  But to call it mathematics, you need to be able to show why
your guess is correct.  Hence provide justifications!

\begin{enumerate}
\item (20 pts) We have discussed many ``Markov-martingale'': e.g.
random walks, exponential random walks, branching processes with $\mu
= 1$.  One that we haven't discussed is called the ``square
martingale.''  Let's discuss it now.

Suppose that $M_t = \sum_{i=1}^t \eta_i$ where the $\eta_i$'s are all
independent and identically distributed with mean 0 and variance
$\sigma^2$.  Take $M_0 = 0$.  Let $Q_t = M_t^2 - t\sigma^2$.
Obviously $Q_0 = 0$.
\begin{enumerate}
\item Derive $E(Q_t)$?  (i.e. Derive means ``show your reasoning.'')
\item Derive $E(Q_t|Q_{t-1})$? 
\item Derive $E(Q_t|Q_{t-1},Q_{t-1},\ldots,Q_0)$?
\item Comment on whether the above 3 parts show that $M_t$ is
  Markovian?  Do they show that $M_t$ is a martingale?
\end{enumerate}

\item (20 pts) Consider the following transition matrix between states
  1, 2, 3 and 4:
\begin{displaymath}
P = \left[\begin{array}{cccc}
             .1 & .5 & .1 & .3 \\
             .2 & .4 & .1 & .3 \\
             0  & .4 & .4 & .3 \\
             0  & .2 & .4 & .1
         \end{array}
     \right]
\end{displaymath}
Suppose the process will actually be stopped when it arrives in either
of the last two states.  
\begin{enumerate}
\item Write down the transition matrix for this modified processes.
\item If it reaches state 3 first, we ``lose''.  If it reaches state 4
  first, we ``win.''  Write down the first step equations for the
  probability of winning given we start in state $i$.
\end{enumerate}
\item (20 pts) Consider a flu epidemic that ``doubles'' every week.
  In other words, we will model the number of people with the flu as a
  branching process (called $X_t$) with mean family size $\mu = 2$.
  As usual, take $X_0 = 1$.
\begin{enumerate}
\item Show that $Z_n = X_n/2^n$ is a non-negative martingale.  
\item Suppose the CDC (Center for Disease Control) predicts that in
  week $n$ there will be fewer than $100*2^n$ flu cases in the US.
  What is the probability that at some point during the flu season,
  this prediction will be wrong?  
\end{enumerate}
\newpage

\item (20 pts) Suppose an urn contains two red and one green ball.
Each time a ball is removed, it is replaced with a blue ball.  The
process is stopped the first time a blue ball is drawn.
\begin{enumerate}
\item (5 pts) List the possible states the system could be in.
\item (10 pts) Write down the transition matrix.
\item (5 pts) How many rounds do we expect this process to run for?
\end{enumerate}


\item (20 pts) Consider the following transition matrix:
\begin{displaymath}
P = \left[\begin{array}{ccc}
             a & b & 0 \\
             0 & c & d \\
             0 & 0 & 1
         \end{array}
     \right]
\end{displaymath}
where $a > 0, b > 0, c > 0$, and  $d > 0$.
\begin{enumerate}
\item If the states are called state 1, state 2, and state 3, which of
these three states are absorbing?
\item Write down the first step equation that solves the probability
of state 2 eventually going to state 3.  What is the solution?
\item What is the probability that starting from state 1, the process
will eventually be in state 3?
\item Approximately, what will $P^T$ look like for large $T$?
\item (bonus = 5 pt) Give a non-trivial upper bound on $P^T$.
\end{enumerate}

\item (For fun only.  Or 5 pts) Modern computers are supposed to be
  very, very fast.  So they should be able to simulate the problem of
  ``a million monkeys will eventually type out all the works of
  Shakespeare.''  As a simple starting point, lets simulate random
  draws from the 27 symbols, ``A'', ``B'', ``C'', \ldots, ``Z'', ``
  ''.
\begin{enumerate}
\item How long will we have to wait for the 18 character phrase ``TO
BE OR NOT TO BE''?
\item What comes next?  (In the Shakespeare, not the monkeys!)
\end{enumerate}

\end{enumerate}
\end{document}


