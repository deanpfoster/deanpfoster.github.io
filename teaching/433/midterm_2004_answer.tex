% ----------------------------------------------------------------
% AMS-LaTeX Paper ************************************************
% **** -----------------------------------------------------------
\documentclass{amsart}
\usepackage{graphicx}
% ----------------------------------------------------------------
\vfuzz2pt % Don't report over-full v-boxes if over-edge is small
\hfuzz2pt % Don't report over-full h-boxes if over-edge is small
% THEOREMS -------------------------------------------------------
\newtheorem{thm}{Theorem}[section]
\newtheorem{cor}[thm]{Corollary}
\newtheorem{lem}[thm]{Lemma}
\newtheorem{prop}[thm]{Proposition}
\theoremstyle{definition}
\newtheorem{defn}[thm]{Definition}
\theoremstyle{remark}
\newtheorem{rem}[thm]{Remark}
\numberwithin{equation}{section}
% MATH -----------------------------------------------------------
\newcommand{\norm}[1]{\left\Vert#1\right\Vert}
\newcommand{\abs}[1]{\left\vert#1\right\vert}
\newcommand{\set}[1]{\left\{#1\right\}}
\newcommand{\Real}{\mathbb R}
\newcommand{\eps}{\varepsilon}
\newcommand{\To}{\longrightarrow}
\newcommand{\BX}{\mathbf{B}(X)}
\newcommand{\A}{\mathcal{A}}
% ----------------------------------------------------------------
\begin{document}

\title{STAT 433: Midterm Solutions}

\date{ \today}




% ----------------------------------------------------------------

\maketitle
% ----------------------------------------------------------------
\flushleft
{\bf QUESTION 1}
\newline
{\bf (a)}
\begin{eqnarray}
\nonumber E(Q_t)&=&E\left(M_t^2-t\sigma^2\right) \\
\nonumber &=&E\left(\left(\sum_{i=1}^{t} \eta_i\right)^2-t\sigma^2\right) \\
\nonumber &=&E\left(\sum_{i=1}^{t} \eta_i^2 + \sum \sum_{i \neq j} \eta_i \eta_j \right) - t\sigma^2 \\
\nonumber &=&t\sigma^2-t\sigma^2
\end{eqnarray}
\newline
{\bf (b)}
\begin{eqnarray}
\nonumber E(Q_t | Q_{t-1})&=&E(M_t^2-t\sigma^2 | Q_{t-1}) \\
\nonumber &=& E\left(\left[M_{t-1}+\eta_t\right]^2-(t-1)\sigma^2-\sigma^2 | Q_{t-1} \right)\\
\nonumber &=& E\left(M_{t-1}^2-(t-1)\sigma^2 | Q_{t-1}\right)+E\left(\eta_t^2-\sigma^2 | Q_{t-1}\right) \\
\nonumber &=& M_{t-1}^2-(t-1)\sigma^2=Q_{t-1}
\end{eqnarray}
\newline
{\bf (c)}
\begin{eqnarray}
\nonumber E(Q_t | Q_{t-1}) &=& E\left(\left[M_{t-1}+\eta_t\right]^2-t\sigma^2 | Q_{t-1},Q_{t-2},\ldots,Q_0 \right)\\
\nonumber &=& E\left(M_{t-1}^2+2M_{t-1}\eta_t+\eta_t^2 - (t-1)\sigma^2 - \sigma^2 | Q_{t-1},Q_{t-2},\ldots,Q_0 \right) \\
\nonumber &=& E\left(Q_{t-1}+2M_{t-1}\eta_t+\eta_t^2 - \sigma^2 | Q_{t-1},Q_{t-2},\ldots,Q_0 \right) \\
\nonumber &=& Q_{t-1}+2E\left(E\left(M_{t-1}\eta_t+\eta_t^2 - \sigma^2|M_{t-1},Q_{t-1},Q_{t-2},\ldots,Q_0\right) | Q_{t-1},Q_{t-2},\ldots,Q_0\right) \\
\nonumber &=& Q_{t-1}+2E\left(M_{t-1}E\left(\eta_t+\eta_t^2 - \sigma^2|M_{t-1},Q_{t-1},Q_{t-2},\ldots,Q_0\right) | Q_{t-1},Q_{t-2},\ldots,Q_0\right) \\
\nonumber &=& Q_{t-1}+2E\left(M_{t-1}\cdot 0 | Q_{t-1},Q_{t-2},\ldots,Q_0\right) \\
\nonumber &=& Q_{t-1}
\end{eqnarray}
\newline
{\bf (d)}
In order to demonstrate that this is Markovian, we need to make a statement about the joint probability distribution. I.e. show $P(Q_t|Q_{t-1}=P(Q_{t-j}|Q_{t-j-1})$ We have made no such statement and hence do not know whether or not this process is Markovian.

However, since $E(|Q_t|)< \infty$ and from (b) we know it is a martingale
\bigskip

{\bf QUESTION 2}
\newline
{\bf (a)}
\newline
\begin{eqnarray}
\nonumber P&=&\left[
\begin{array}[4]{llll}
.1 & .5 & .1 & .3  \\
.2 & .4 & .1 & .3  \\
0 & 0 & 1 & 0 \\
0 & 0 & 0& 1 \\
\end{array}
\right]
\end{eqnarray}
\newline
{\bf (b)}
\newline
If we let $\tau=\text{min}\left\{ n \ge 0; X_n=3 \text{ or } X_n=4 \right\}$ and $u_i=P(X_T=4 | X_1=i)$ then the first step equations are
\begin{eqnarray}
\nonumber u_1&=&0.1u_1+.5u_2+.3 \\
\nonumber u_2&=&0.2u_1+.4u_2+.3 \\
\nonumber u_3&=&0 \\
\nonumber u_4&=&1
\end{eqnarray}
\bigskip

{\bf QUESTION 3}
\newline
{\bf (a)}
\newline
\begin{eqnarray}
\nonumber && \text{Step 1} \\
\nonumber && E\left(Z_n\right)= E\left(X_n \cdot 2^{-n}\right)=2^{-n}E\left(X_n\right)=2^{-n} \mu^n=2^{-n}2^n=1 < \infty \\
\nonumber && \text{Step 2} \\
\nonumber && E\left(Z_n | Z_{n-1},\ldots,Z_0\right)=2^{-n}E\left(X_n |X_{n-1},\ldots,X_0\right)=2^{n-1}2X_{n-1}=Z_{n-1}
\end{eqnarray}
\newline
{\bf (b)} From the maximal inequality we have
\begin{eqnarray}
\nonumber P\left(\max_{0\le i \le n} X_i \ge 100\cdot 2^n \right) &=&P\left(\max_{0\le i \le n} Z_i \ge 100 \right) \\
\nonumber &\le& \frac{E(Z_0)}{100}=\frac{1}{100} 
\end{eqnarray}
\bigskip
{\bf QUESTION 4}
\newline
{\bf (a)}
\newline
The states are as follows
\begin{table}[h]
\centering
\begin{tabular}{|c|ccc|}
\hline
State  & & Number of & \\
& Red balls & Green balls & Blue balls \\
\hline
0 & 2 & 1 & 0 \\
1 & 2 & 0 & 1 \\
2 & 1 & 1 & 1 \\
3 & 1 & 0 & 2 \\
4 & 0 & 1 & 2 \\
5 & 0 & 0 & 3 \\
6 & 1 blue & ball is & drawn \\
\hline
\end{tabular}
\end{table}
\newline
{\bf (b)}
The transition matrix would be
\begin{eqnarray}
\nonumber P&=&\left[
\begin{array}[6]{ccccccc}
0 & \frac{1}{3} & \frac{2}{3} & 0 & 0 & 0 & 0  \\
0 & 0 & 0 & \frac{2}{3} & 0 & 0 & \frac{1}{3}  \\
0 & 0 & 0 & \frac{1}{3} & \frac{1}{3} & 0 & \frac{1}{3}  \\
0 & 0 & 0 & 0 & 0 & \frac{1}{3} & \frac{2}{3}  \\
0 & 0 & 0 & 0 & 0 & \frac{1}{3} & \frac{2}{3}  \\
0 & 0 & 0 & 0 & 0 & 0 & 1  \\
0 & 0 & 0 & 0 & 0 & 0 & 1  \\
\end{array}
\right]
\end{eqnarray}
\newline
{\bf (c)}
We seek a solution to the following system of equations
\begin{eqnarray}
\nonumber v_0&=&1/3 v_1 + 2/3 v_2 + 1 \\
\nonumber v_1&=&2/3 v_3 + 1\\
\nonumber v_2&=&1/3 v_3 + 1/3 v_4 + 1 \\
\nonumber v_3&=&1/3 v_5 + 1 \\
\nonumber v_4&=&1/3 v_5 + 1 \\
\nonumber v_5&=& 1 \\
\nonumber v_6&=&0
\end{eqnarray}
which leads to $v_0=\frac{26}{9}$
\newline
\bigskip
{\bf QUESTION 5}
\newline
{\bf (a)}
State 3 is the absorbing state 
\newline
{\bf (b)}
\begin{eqnarray}
\nonumber u_i&=&P(X_T=3|X_0=i) \\
\nonumber u_1&=&au_1+bu_2 \\
\nonumber u_2&=&cU_2+d \\
\nonumber u_3&=&1 \\
\nonumber & \Downarrow & \\
\nonumber u_2-cu_2&=&d \Rightarrow u_2=\frac{d}{1-c}=P(X_T=3|X_0=2)
\end{eqnarray}
{\bf (c)}
\begin{eqnarray}
\nonumber u_1&=&au_1+bu_2 \\
\nonumber (1-a)u_1&=&\frac{bd}{1-c} \\
\nonumber u_1&=& \frac{bd}{(1-a)(1-c)}=\underbrace{\frac{bd}{(1-a)(1-c)}}_{b=1-a,d=1-c}=\frac{bd}{bd}=1
\end{eqnarray}
{\bf (d) (e)}
Remember from part (a) that state 3 is the absorbing state so eventually ......
\begin{eqnarray}
\nonumber P^T&\approx&\left[
\begin{array}[4]{lll}
0 & 0 & 1  \\
0 & 0 & 1  \\
0 & 0 & 1  \\
\end{array}
\right]
\end{eqnarray}
and an appropriate bound would be
\begin{eqnarray}
\nonumber P^T&\approx&\left[
\begin{array}[4]{lll}
a^T & (1-bd)^{T/2} & 1  \\
0 & c^T & 1  \\
0 & 0 & 1  \\
\end{array}
\right]
\end{eqnarray}
\bigskip
{\bf QUESTION 6}
\newline
If a team of gamblers come into a casino everyday to gamble on the expression ``TO BE OR NOT TO BE", with the casino laying fair odds 27 to 1, the expected winnings of the casino is zero. 
\begin{eqnarray}
\nonumber 0&=&E[\tau]-\underbrace{27^{18}}_{\text{"TO BE OR NOT TO BE" winner}}-\underbrace{27^5}_{\text{"TO BE `` winner}} \\
\nonumber E[\tau]&=&27^{18}+27^5
\end{eqnarray}
\end{document}
