\documentclass[20pt]{extarticle}
\usepackage{hyperref}
\usepackage[usenames]{color}\definecolor{mypurple}{rgb}{.6,.0,.5}\newcommand{\note}[1]{\noindent{\textcolor{mypurple}{\{{\bf note:} \em #1\}}}}
\newcommand{\tech}[1]{\noindent{\textcolor{red}{\{{\bf technical note:} \em #1\}}}}
\usepackage{xypic}


\begin{document}
\title{Stat 433: Chapter V.1: The Poisson distribution}

\section{Admistrivia}
\begin{itemize}
\item  HW 7 due early next week, 8 du later next week.
\item Read p 267 - 274.
\item Mike loved teaching you all's again. Thanks for being such a
great class.
\end{itemize}

\newpage

\section{How to do continuous time?}

For deterministic continuous functions, we have the intermediate value
 theorem.  It allows us to gradually build up a function by successive
approximations. 

This doesn't help for the discrete setting.

But in probability we have a new magic trick available to us:
 \cite{http://en.wikipedia.org/wiki/Infinite_divisibility_(probability)}{infinitely
 divisable distributions.}

\section{Infinitely divisible distributions}

Example which almost works.
\begin{itemize}
\item Let $X$ = number of Penn sudents winning the lottery this semester
\item $X = M+W$ where $M$ = num Men, $W$ = num Women winners
\item $W = S+N$ where $S$ = num women taking statistics who win
\item $S = A + B$ where $A$ = num women taking statistics 433 who win
\item $S = I + J$ where $I$ = 1 if Sophie wins
\item $I = $ oops... can't divide further
\end{itemize}

\newpage

Alternative division:
\begin{itemize}
\item Let $X$ = number of Penn sudents winning the lottery this semester
\item $X = B+A$ where $B$ = num before spring break
\item $A = F+G$ where $F$ = num won in feburary
\item $\cdots$
\item where $K$ = number won between 12:37.05 and 12:37.06 on Feb 3rd.
\end{itemize}
\newpage

Both are divisable.  
\begin{itemize}
\item The first, the binomial, is only finitely divisable.
\item The second, the Poisson, is infinitely divisable.
\end{itemize}

Three main examples:
\begin{itemize}
\item Normal (leads to brownian motion)
\item Cauchy (leads to Levy processes)
\item Poisson (leads to poisson process)
\end{itemize}

Definition:  A family of distributions $F_\mu$ is infinitely divisable
if for any random variable $Y \sim F_\mu$ we can find random variables
$A$, and $B$ which are independent and $Y=A+B$ and $A$ and $B$ both
have distribution $F_{\mu/2}$.

\tech{Sometimes you need other paramaterizations, so you end up with
something besides $F_{\mu/2}$, it might be $F_{\sqrt{\mu}}$ instead.}

Draw some pictures of how to construct these.

\newpage

\section{Poisson distribution review}

\begin{itemize}
\item Probability function:
\begin{displaymath}
P(X = x) = \frac{e^{-\lambda}\lambda^x}{x!}
\end{displaymath}
\item Mean
\begin{displaymath}
E(X)  =  \lambda 
\end{displaymath}
\item Variance
\begin{displaymath}
Var(X)  =  \lambda
\end{displaymath}
\item If $X \sim $Poisson$(\lambda)$ and $Y \sim $Poisson$(\mu)$, then
 $X+Y$ is Poisson$(\lambda + \mu)$.
\end{itemize}

\newpage
\section{The Poisson process}

\begin{itemize}
\item Index family of random variables by time
\item $X_t$, so an uncountable infinite collection of random variables
\tech{Actually, only countably infinite by ID.  Just keep dividing it
in two and take limits.}
\item $X_t - X_s$ is Poisson($\lambda(t-s)$).
\item $X_0 = 0$.
\item Nice story, but does it exist? (Yes, but ID.)
\end{itemize}

\section{Examples}
\begin{itemize}
\item  Defects along DNA.  
\begin{itemize}
\item SNPs
\item \$1000 genome soon!
\end{itemize}
\item Good model of counts upto time $t$ for many things:
\begin{itemize}
\item gamma particles
\item people at bank
\item accidents
\item insurance
\end{itemize}
\item Bad model in some setting:
\begin{itemize}
\item Good model for bikers in triathelone.
\item Bad model for bikers in tour de france.
\item Why this difference? (Drafting)
\end{itemize}
\end{itemize}
\newpage

\section{Nonhomogeneous process}
\begin{itemize}
\item $\lambda(t)$ instead of constant lambda
\item $E(X_{t+h}-X_t) = lambda(t)h$
\item $P(X_{t+h}-X_t = 1) = lambda(t)h$
\item Example: Call board.
\end{itemize}
\end{document}
