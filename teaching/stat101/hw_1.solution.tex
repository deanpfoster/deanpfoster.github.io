\documentclass[10pt]{article}
\usepackage[dvips]{graphicx}
\setlength{\textheight}{8in}
\setlength{\textwidth}{6in}
\setlength{\headsep}{0.4in}
\setlength{\headheight}{0.1in}
\setlength{\topmargin}{0in}
\setlength{\oddsidemargin}{0in}
\renewcommand{\baselinestretch}{1.5}
%\pagestyle{empty}

\begin{document}
\begin{center}
{\bf \large Statistics 101: Homework 1}
\end{center}


{\bf Exercise 2.2} 
\begin{itemize}
\item[\empty]The task here is to interpret the 
histogram rather to construct it.
\item[\bf a.] The average value falls in the middle of the histogram, where 
the histogram balances. This particular histogram is close to 
symmetric; with nearly equal left and right tails. It appears to us 
that the histogram would balance somewhere in the interval between 
842 and 847. Therefore the average value should be about 845.

Variability is a matter of how spread out the histogram is. In this 
case, there is a value in the 812 to 817 interval, and, on the other 
side, a value in the 867 to 872 interval. (Note in both intervals, 
the frequency is shown as 1, so there is only one value.) We'd say 
there was virtually no skewness.
\item[\bf b.] The smoothed component is shown as a curve through the 
histogram. Skewness is indicated by a nonsymmetric curve. In this 
case, the curve appears almost perfectly symmetric around the middle 
peak. There is a little or no skewness.
\end{itemize}

{\bf Exercise 2.3}

\begin{itemize}
\item[\empty] Recall that a stem-and-leaf display, like the one shown, 
groups the data according to the values in the stem. The first value 
shown must be 812 (as opposed to 81.2 or 8120), from a look at the 
data.

The stem-and-leaf display gives interval width of 10, in contrast to 
the width of 5 in the histogram. In effect, the stem-and-leaf display 
centers the intervals at 815, 825, etc.; the centers for the histogram 
also differ. The two pictures aren't identical.

We see basically the same pattern in both displays. The average value 
is somewhere in the 840's, there is modest variability, and there is 
very little skewness.
\end{itemize}

{\bf Exercise 2.11}
\begin{itemize}
\item[\bf a.] The mode is defined to be the most common value, and 
is most often used to describe qualitative data. Here, the data are 
quantitative. The mode is not very useful for such data.
\item[\bf b.] The median is defined as the $(n+1)/2$th value, when 
the data are arranged from lowest to highest. There are $n=26$ data 
values; the median is the $(26+1)/2=13.5$th value, the average of the 
13th and 14th values. Arrange the data in order from low to high.
\begin{center}
\begin{tabular}{ccccccccccccc}
547&625&630&656&664&667&667&667&679&688&688&688&688\\
688&691&694&697&699&700&701&702&703&703&703&708&711
\end{tabular}
\end{center}
\item[\bf c.] The mean is the average of all 26 numbers. We would 
regard the data as a sample from the ongoing production process, 
so we would call the mean $\bar{y}$. Of course, it doesn't matter 
if we had the original number, or the sorted numbers.
$$\bar{y}=\frac{546+625+\cdots+711}{26}=679.4$$
\item[\bf d.] Remember that the mean is pulled in the direction of 
skewness, as compared to the median. Here, the mean is less than the 
median, indicating that there is a ``tail'' of data toward the left (
smaller) values, pulling the mean down relative to the median.
\end{itemize}

{\bf Exercise 2.12}
\begin{itemize}
\item[\empty] Recall that the first digit(s) of each data value are 
recorded in the the left-hand, stem part of the diagram, and the
next digit in the right hand, leaf part. It's easiest to use the 
sorted data, but either way gives the same picture. The 547 value 
is far lower than any other, so we might indicate it separately.
\begin{center}
\begin{tabular}{r|l}
54&7\\
\vdots&\\
63&5\\
63&0\\
64&\\
65&6\\
66&4 7 7\\
67&7 9\\
68&8 8 8 8 8\\
69&1 4 7 9\\
70&0 1 2 3 3 3 8\\
71&1
\end{tabular}
\end{center}
The data are clearly left skewed, and there is one outlier on the low 
(left) side. Even if we to ignore the outlier, there is still skewness.
\end{itemize}

{\bf Exercise 2.17}
\begin{itemize}
\item[\bf a.] The mean is shown as 794.23; the standard 
deviation (STDEV)is shown as 34.25. Therefore, mean minus one standard 
deviation is $794.23-34.25=759.98$ and mean plus one standard deviation 
is $794.23+34.25=828.48$.  The actual data are whole numbers, not 
decimals; values between 760 and 828 will fall in this interval.
\item[\bf b.] 51 out of 60 is 85$\%$; $51/60=0.85$. According to the 
Empirical Rule, the percentage theoretically should be only $68\%$. 
The one standard deviation interval is too wide in this case. It seems 
likely that skewness or outliers have inflated the standard deviation. 
This will make the interval ``too wide'' and capture ``too many'' of 
the data values.
\end{itemize}

{\bf Exercise 2.18}
\begin{itemize}
\item[\empty] Recall that outliers are shown in a boxplot as points 
beyond the ``whiskers'' of the plot. The boxplot shows several 
outliers, including one very serious one. These outliers will inflate 
the standard deviation, making the ``one standard deviation interval'' 
wide and causing the Empirical Rule to fail.
\end{itemize}

{\bf Exercise 2.70}
\begin{itemize}
\item[\bf b.] A histogram is shown here. The data 
appear basically mound-shaped, with a modest right skew.

\centerline{\includegraphics{histogram.eps}}

\item[\bf c.] The mode is at 65, which is a decent first guess for 
the mean. But there are more values higher than 65 than lower. These 
values will pull up the mean a little. The mean should be a bit above 
65, say about 66.
\item[\bf d.] The range $66\pm 5$ just about includes all the values. 
Therefore, two standard deviations be about 5, so one standard 
deviation should be about 2.5.
\item[\bf e.] JMP yielded 
\begin{verbatim}
             FOOD
Minimum              61.22
Maximum              70.74
Mean                 66.02377 
Median               65.81
Standard Deviation        2.114983                
\end{verbatim}                     
\end{itemize}

{\bf Exercise 2.71}
\begin{itemize}
\item[\bf a.] A stem-and-leaf display of the data is as follows:
\begin{verbatim}
Decimal point is at the colon
   12 : 4
   12 : 99
   13 : 00111222444
   13 : 55556667778889999
   14 : 0011112244
   14 : 5678999
   15 : 0
   15 : 555
   16 : 1
\end{verbatim}
We would call that more or less bell-shaped, with some right skew.      
\item[\bf e.] JMP yielded 
\begin{verbatim}
             NONFOOD
Minimum              12.38
Maximum              16.15
Mean                 13.94585
Median               13.84
Standard Deviation        0.7710269  
\end{verbatim}                     
\end{itemize}

{\bf Exercise 2.72}
\begin{itemize}
\item[\bf a.] Here is a boxplot.

\centerline{\includegraphics{boxplot.ps}}
 
\item[\bf b.] There is one candidate outlier on the low side, at about 
0.805 or so. There are no outliers shown above the upper whisker.
\item[\bf c.] JMP calculated summary statistics as shown here.
\begin{verbatim}
             RATIO
Minimum              0.8060
Maximum              0.8434
Mean                 0.8256
Median               0.8254
Standard Deviation       0.007367554
\end{verbatim}  
Is it true that $0.8256=66.02377/(66.02377+13.94585)$? By hand, the 
fraction comes out to 0.8256, all right. In fact, it isn't true in 
general that the mean of a ratio is the ratio of means. 
\end{itemize}

\end{document}




