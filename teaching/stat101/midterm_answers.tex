\documentclass[10pt]{article}
\usepackage[dvips]{graphicx}
\setlength{\textheight}{8in}
\setlength{\textwidth}{6in}
\setlength{\headsep}{0.4in}
\setlength{\headheight}{0.1in}
\setlength{\topmargin}{0in}
\setlength{\oddsidemargin}{0in}
\renewcommand{\baselinestretch}{1.5}
%\Pragestyle{empty}

\def\Pr{{\bf P}}
\def\E{{\bf E}}

\begin{document}
\begin{center}
{\bf \large Statistics 101: Midterm Exam 1}
\end{center}


{\bf Problem I}
\begin{itemize}
\item[\bf A.] Since the distance from the median to lower quartile
$Q_1$ is less than the distance from the median to the upper
quartile $Q_3$ this indicates the number of hits is skewed to the
right. Also the lower whisker is smaller than the upper whisker.
\item[\bf B.] Visually 36 is not above the upper whisker.
Also $IQR=Q_3-Q_1=19-2=17$ and the upper inner fence is
$Q_3+1.5*IQR=44.5$. Since 36 is smaller it is not a candidate
outlier.
\item[\bf C.] The median is higher for recreational than for
business. Since the box is bigger for recreational it also has
higher variance. Business and recreational are both skewed to the
right.
\item[\bf D.] Since 36 is well above the box for business it would
be candidate outlier. Visually $Q_3\sim 12$ and $IQR\sim 12$. So
limit is $12+1.5*12=30$.
\item[\bf E.]
\item[\bf (i)] The right tail would be pulled in (higher values
less extreme). This would make the data look symmetric.
\item[\bf (ii)] Since logs preserve order, median of the
$\ln (\mbox{ hits }) =\ln(7.5)=2.0149$.
\end{itemize}

{\bf Problem II}

\begin{itemize}
\item[\bf A.] Because  $B$ and $C$ are independent, it follows
that
$$\Pr(A\cup C)=\Pr(B)+\Pr(C)-\Pr(B\cap C)=\Pr(B)+\Pr(C)- \Pr(B)*\Pr(C)
              =.5+.4-.5*.4=.7.$$
\item[\bf B.] Since $B\cap A$ is a subset of $B\cup C$ we get that
$$\Pr(B\cap C|B\cup C)=\frac{\Pr(B\cap C)}{\Pr(B\cup
C)}=.2/.7=2/7.$$
\item[\bf C.]
\item[\bf (i)] Since $\Pr(A\cap B)=\Pr(B|A)\Pr(A)=.6*.6$ we find that
$$\Pr(\bar{A}\cap\bar{B})=1-\Pr(A\cup B)=1-[\Pr(A)+\Pr(B)-\Pr(A\cap B)]=
1-(.6+.5-.6*.6)=.26.$$

The different interpretation of this question is possible, so the
full credit is granted also for the following solution:
$$\Pr(\overline{A\cap B})=1-\Pr(A\cap B)=1-.6*.6=.64.$$
\item[\bf (ii)] Since the probability that C occurs is independent
of whether A or B occurs  we have that
$$\Pr(\bar{C}|\bar{A}\cup\bar{B})=\Pr(\bar{C})=.6.$$
Hence,
$$\Pr(A\cup B\cup C)=1-\Pr(\bar{A}\cap\bar{B}\cap\bar{C})=
1-\Pr(\bar{C}|\bar{A}\cap\bar{B})\Pr(\bar{A}\cap\bar{B})$$
$$=1-\Pr(\bar{C})\Pr(\bar{A}\cap\bar{B})=1-.6*.26=.844.$$
\end{itemize}

{\bf Problem III}
(Note: Some students used the information from problem II.  If the information was used correctly, no penality resulted.)
\begin{itemize}
\item[\bf A.] $\Pr(\mbox{Find on C})$
$$\begin{array}{rl}
=&\Pr(\mbox{Find on C}|\mbox{Correct submenu on C is chosen
})\Pr(\mbox{Correct submenu on C is chosen})\\=&.7*.9=.63.
\end{array}$$
\item[\bf B.] Using previous result we find that
$$\begin{array}{rcl}
\Pr(\mbox{Find the page})&=&\Pr(\mbox{Find the page}|\mbox{A is
chosen})\Pr(\mbox{A is chosen})\\ &&+\Pr(\mbox{Find the
page}|\mbox{B is chosen})\Pr(\mbox{B is chosen})\\
&&+\Pr(\mbox{Find the page}|\mbox{C is chosen})\Pr(\mbox{C is
chosen})\\ &=&.5*1/3+.56*1/3+.63*1/3=169/300 =0.56(3).
\end{array}$$
\item[\bf C.] The probability of finding the page on A is .5; for
B this probability is .56; for C it is .63. So search engine C is
best.
\item[\bf D.] First we note that
$$\begin{array}{rcl}
\Pr(\mbox{The page is found})&=& \Pr(\mbox{A is chosen}\cap\mbox{A
is accessed}\cap\mbox{The page is found on A})\\ &&+\Pr(\mbox{B is
chosen}\cap\mbox{B is accessed}\cap\mbox{The page is found on
B})\\ &&+\Pr(\mbox{C is chosen}\cap\mbox{C is
accessed}\cap\mbox{The page is found on C})\\ & =&
1/3*1*.5+1/3*.9*.56+1/3*.8*.63=1508/3000=.502(6).
\end{array}$$
So,
$$\Pr(\mbox{A is chosen}|\mbox{The page is
found})=\frac{1/3*1*.5}{1/3*1*.5+1/3*.9*.56+1/3*.8*.63}=500/1508.$$
\end{itemize}
{\bf Problem IV}
\begin{itemize}
\item[\bf A] The probability distribution of $X$ is shown below:
\begin{center}
\begin{tabular}{|c|c|c|c|c|}
  \hline
  $X$   & -1000 & 0 & 500 & 1000 \\
  \hline
  $\Pr$ & .1 & .2 & .2 & .5 \\ \hline
\end{tabular}
\end{center}
\item[\bf (i)]   $\Pr(X>0)=.2+.5=.7.$
\item[\bf (ii)]  $\E(X)=-1000*.1+0*.2+500*.2+1000*.5=500.$
\item[\bf (iii)] ${\bf
Var}(X)=(-1000-500)^2*.1+(0-500)^2*.2+(500-500)^2*.2+(1000-500)^2*.5=400000$
Respectively, standard deviation $\sigma=\sqrt{400000}$.
\item[\bf B]
\item[\bf (i)]
$\Pr(X>0)=\displaystyle\int_0^1(x+1)/2\,dx=x^2/4+x/2\Big|_0^1=3/4.$
\item[\bf (ii)]
$\displaystyle\E(X)=\int_{-1}^1x(x+1)/2\,dx=x^3/6+x^2/4\Big|_{-1}^1=1/3.$
\item[\bf C]
The distribution of daily profits, $Y$, is given by
\begin{center}
\begin{tabular}{|c|c|c|c|c|}
  \hline
  $Y$   & -1500 & -500 & 500 & 1500 \\
  \hline
  $\Pr$ & .1 & .2 & .2 & .5 \\ \hline
\end{tabular}
\end{center}
Hence, $\E(Y)=-1500*.1+-500*.2+500*.2+1500*.5=600.$

\end{itemize}
\end{document}
