\documentclass[14pt]{extarticle}
\usepackage{hyperref}
\usepackage{enumerate}
\renewcommand{\baselinestretch}{1.3}

\begin{document}
\section*{Assignment 5: Paranoia}

(\href{assignment5.pdf}{pdf version}) 

\begin{enumerate}
\item Suppose you have three friends ...

\item On argument as to why the proof I gave in class works is that
the log is a too crude summary of wealth.  So if we had to work with
real money, the proof would not work.  Let's show that theory wrong.

In this story, we will assume you have two friends who will help you
pick ponies at a race track.  Now since gambling is illegal, we will
have your maximum amount you can win be a single penny.  To avoid all
issues of wealth, we won't worry about stakes and odds and all that.
So your problem is:

\begin{itemize}
\item Before each race is run, decide who you want to bet for you,
 friend A or friend B.  Call this $w_t$.  (Of course you can be
 wishywashy and got with a fraction of each as long as it adds up to
 exactly one.)
\item When the race $t$ is run, friend A will win $a_t$ pennies, and
friend $B$ will win $b_t$ pennies.
\item You collect $c_t = w_t a_t + (1-w_t) b_t$.
\end{itemize}

Your goal is to have your average winnings be about the same as the
better of the two average winnings of your two friends.  So,
\begin{displaymath}
\frac{\sum_{t=1}^T c_t}{T} \approx \max(\frac{\sum_{t=1}^T b_t}{T},\frac{\sum_{t=1}^T b_t}{T})
\end{displaymath}
So, no tricks this time.  No compounding.  No logs.  Just the real
average. 
\begin{enumerate}
\item ...
\end{enumerate}

\item For each of the following loss function, say what probability
you would announce if you believe the true probability was 50/50.
\begin{enumerate}
\item loss$(p,X) = I_{p>.9}X$
\item ...
\end{enumerate}

\item Consider the following three forecasting rules,
\begin{eqnarray*}
\overline{X}_t & = & \sum_{i=1}^{t-1} \frac{X_i}{t-1}\\
{\cal s}_t & = & 2/3 - X_{t-1}/3 \\
{\cal E}_t & = & .9 {\cal E}_{t-1} + .1 X_{t-1}
\end{eqnarray*}
compute a calibration curve for each of them

\end{enumerate}

\end{document}
