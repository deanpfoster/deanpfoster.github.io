\documentclass[12pt]{extarticle} % 14, 17, 20 all exist
\usepackage{hyperref}

\usepackage[usenames]{color}\definecolor{mypurple}{rgb}{.6,.0,.5}\newcommand{\note}[1]{\noindent{\textcolor{mypurple}{\{{\bf note:} \em #1\}}}}
\newcommand{\tech}[1]{\noindent{\textcolor{red}{\{{\bf technical note:} \em #1\}}}}
\usepackage{xypic}

\renewcommand{\baselinestretch}{1.4}
\begin{document}
\title{Loss Functions}
\maketitle
\href{class_loss_functions.pdf}{pdf version}
\subsection*{Models: A review}
Recall our usual model;
\begin{displaymath}
(\forall i) Y_i = \alpha + \sum_j \beta_j X_{ij} + \epsilon_i
\end{displaymath}
where
\begin{displaymath}
\epsilon_i \sim_{iid} N(0,\sigma^2)
\end{displaymath}
\begin{itemize}
\item loss = - log(probability)
\item loss = $(Y -EY)^2$ = Var(Y)
\item Can we use other loss functions?
\end{itemize}

\subsection*{Alternative loss function}

Game:
\begin{itemize}
\item Guess a probability $p$ of rain
\item (Utility) If it rains, you get paid $p$, if it doesn't rain you get $1-p$.
\item (Loss) If it rains, you lose $1-p$, if it doesn't lose $p$.
\item Gratitous mathematics: Called absolute loss,
\begin{eqnarray*}
\hbox{loss}(p,X) &= &|X - p|\\
&= &\left\{\begin{array}{l@{\quad}l}1-p &
\hbox{if X = 1}\\ p & \hbox{if X = 0}
\end{array}
\right.
\end{eqnarray*}
\item How would you play if it rains about 2 out of 3 times?
\end{itemize}

\subsection*{A better version of same loss function}
Given no one will tell the true, why not admit it and change the
reward structure.
\begin{itemize}
\item If $p > .5$ they will say ``p = 1'' so give them the same loss
as if they said that.
\item If $p < .5$ they will say ``p = 0'' so give them the same loss
as if they said that.
\item Magic function: $I_{p > .5}$ does the conversion
\item Loss = $|I_{p > .5} - X|$.
\item No no lead to ``lie'' the loss function does it for us.
\item Whole area of {\em mechanism design} works on these problems
\end{itemize}

\subsection*{What if we want a differnt split point?}

Symetric loss (like above) \\
\begin{tabular}{r|c|c}
{\bf LOSS} & X=0 & X=1 \\ \hline
guess low  & 0   & 1  \\ \hline
guess high & 1   & 0 
\end{tabular}

General loss\\
\begin{tabular}{r|c|c}
$\hbox{\bf loss}_a$ & X=0 & X=1 \\ \hline
guess low  & 0   & 1-a  \\ \hline
guess high & a   & 0 
\end{tabular}

\begin{itemize}
\item requires biasing your probabilitic guess
\item If $p>a$ guess high
\item If $p<a$ guess low
\item General loss function
\begin{eqnarray*}
\hbox{\bf loss}_a(p,X) &= & aI_{p > a}(1-X) + (1-a)I_{p \le a}X \\
\end{eqnarray*}

\end{itemize}



\subsection*{Proper scoring rules}

({\em Management Science}, {\bf Vol. 40}, No. 11. (1994), pp. 1395-1405.)

A proper scoring rule is one for which it is rational to tell the
truth if you know it.

Examples:
\begin{itemize}
\item Quadratic:
\begin{displaymath}
\hbox{loss}(p,X) = (p - X)^2
\end{displaymath}
\item log:
\begin{displaymath}
\hbox{loss}(p,X) = -log(p)X - log(1-p)(1-X)
\end{displaymath}
\item Weird:
\begin{displaymath}
\hbox{loss}(p,X) = \frac{1 - X- p}{\sqrt{1 - 2p - p^2}}
\end{displaymath}
\item General:
\begin{displaymath}
\hbox{loss}(p,X) = \int_0^1  \hbox{\bf loss}_a(p,X) w(a) da
\end{displaymath}
for some $w(\cdot)\ge 0$.
\end{itemize}


\end{document}
