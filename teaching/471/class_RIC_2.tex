\documentclass[12pt]{extarticle} % 14, 17, 20 all exist
\usepackage{hyperref}

\usepackage[usenames]{color}\definecolor{mypurple}{rgb}{.6,.0,.5}\newcommand{\note}[1]{\noindent{\textcolor{mypurple}{\{{\bf note:} \em #1\}}}}
\newcommand{\tech}[1]{\noindent{\textcolor{red}{\{{\bf technical note:} \em #1\}}}}
\usepackage{xypic}

\renewcommand{\baselinestretch}{1.4}
\begin{document}
\title{RIC 2: Risk Inflation Criterion continued}
\maketitle
\href{class_RIC_2.pdf}{pdf version}
\section{Admistrivia}
\begin{itemize}
\item Homework has been posted
\item Do we want to do projects? 
  \begin{itemize}
  \item teams of 2
  \item 10 minute presentations in class (last 2 days of class)
  \item plus write up
  \item Analyse your own data using ideas from class
  \end{itemize}
\end{itemize}
\subsection{Risk inflation definition}

The risk inflation of an estimator is its risk compared to the best
regression model.

Story version: Supposed 20 years later, science knows which variables
should have been used when you solved your regression problem.
Looking back on it, you say, ``Gee I wish I'd only used, $X17$ and
$X23$, that would have been really smart.''  If you had done that,
your error would have been 50, but instead you used a AIC estimator,
so your error was 400.  This gives you a risk inflation of 8.

Definition:
\begin{displaymath}
RI \equiv \frac{\hbox{Risk}}{\hbox{best Risk}}
\end{displaymath}
More symbols, which estimator are we talking about?  ($\hat{\beta}$) Which data set?  ($\beta$)
\begin{displaymath}
RI(\hat{\beta},\beta) \equiv \frac{\hbox{Risk}(\hat{\beta},\beta)}{\hbox{best Risk}}
\end{displaymath}
What does best mean?
\begin{displaymath}
RI(\hat{\beta},\beta) \equiv
\frac{\hbox{Risk}(\hat{\beta},\beta)}{\displaystyle\min_{\hbox{\scriptsize all $2^p$ models}} \hbox{Risk}(\hat{\beta}^{mle},\beta)}
\end{displaymath}

\note{We can actually precompute the best:
\begin{displaymath}
RI(\hat{\beta},\beta) \equiv \frac{\hbox{Risk}(\hat{\beta},\beta)}{q(\beta)\sigma^2)}
\end{displaymath}}




\centerline{\begin{tabular}{r|l|l}
\bf{cut off} & \bf{Name} & \bf{Risk Inflation} \\ \hline
 0 &  MLE & p \\
 1 &  $\min s^2$ & $.8p$ \\
 2 &  AIC/Cp & $.57p$ \\
 $\log n$ & BIC & $\infty$ \\
 $2 \log p$& RIC & $2 \log p$\\ 
 $2 \log (p/q)$ & FDR & best? (open problem) \\
 $\infty$ & ``null'' & $\infty$ 
\end{tabular}}

Clearly the ``$\infty$'''s are bad!
\centerline{\begin{tabular}{r|l|l}
\bf{cut off} & \bf{Name} & \bf{Risk Inflation} \\ \hline
 $\infty$ & ``null'' & $\infty$ \\
 $\log n$ & BIC & $\infty$ \\
 0 &  MLE & p \\
 1 &  $\min s^2$ & $.8p$ \\
 2 &  AIC/Cp & $.57p$ \\
 $2 \log p$& RIC & $2 \log p$\\ 
 $4 \log p$& RIC & $4 \log p$\\ 
 $2 \log (p/q)$ & FDR & best? (open problem) 
\end{tabular}}
\subsection*{Driving force}
Two issues need to be balanced:
\begin{itemize}
\item Over fitting (putting in to many variables)
\item missing signal
\end{itemize}

Hence a trade off:
\begin{itemize}
\item As you increase your penality, you start missing signal.  So keep penality small.
\item As you decrease you penality, you start adding zeros.  So keep penality large.
\end{itemize}

We can graph each of these seperaly.  (see slides)


Take home messages from Risk Inflation:
\begin{itemize}
\item Stepwise regression can work well
\item Use $\sqrt{2 \log p}$ 
\item In JMP, use Prob-to-enter = $1/p$ is a good approximation
\item Puts in more variables than Bonferroni ($.05/p$).
\item Lots and lots of variables are fine
\end{itemize}
\section{Too many variables for JMP/R}

What if you have too many variables for JMP to handle?  What if there
are too many for R to handle?  SAS?  

Now we are talking big data.
\subsection{Example: Predicting Bankruptcy  }
\begin{description}

\item[ Predict onset of personal bankruptcy ] \ \\
  Estimate probability customer declares bankruptcy.
  
\item[Challenge] \ \\
  Can {\bf stepwise regression} predict as well as
  commercial ``data-mining'' tools or substantive models?

\item[ Many features ] \ \  About 350 ``basic'' variables
  \begin{itemize}
      \item Short time series for each account
      \item Spending, utilization, payments, background
      \item Missing data and indicators
      \item Interactions are important (cash advance in Vegas)
      \item {\bf 67,000 predictors!}
  \end{itemize}
\end{description}

\subsection{Faster than stepwise}

How many calculations are needed for one step of stepwise regression? 
\begin{itemize}
\item each row must be processed
\item each variable must be looked at
\item CPU = $n * p$
\item To add the variable also takes time
\end{itemize}
Cute Hack:
\begin{itemize}
\item Try each variable only once
\item Streaming features rather than stepwise
\item \href{http://gosset.wharton.upenn.edu/~foster/auction.html}{On line code}
\item Can't be done in JMP
\end{itemize}

\newpage
\subsection{Faster algorithm: Alpha investing}

\begin{verbatim}
let Wealth = .05
WHILE (Wealth > 0)
{
   let bid = amount to bid
   let Wealth = Wealth - bid
   let X be the next variable to try
   IF p-value of X is less than bid
     {
       Wealth = Wealth + .05
       Add X to the model
     }
}
\end{verbatim}
\begin{itemize}
\item No nested loops improves speed 
\item Allows for more user control
\end{itemize}

\end{document}
