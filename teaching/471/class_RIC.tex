\documentclass[14pt]{extarticle}

\begin{document}
\title{RIC: Risk Inflation Criterion}

\section{Admistrivia}
\begin{itemize}
\item Edward:
\begin{itemize}
\item Ed said you were a lovely class
\item Hopefully you enjoyed having him as much as he enjoyed teacing you
\item We'll continue where he left off
\end{itemize}
\item Personal:
\begin{itemize}
\item Broke my collarbone
\item Same surgeon as I had a year ago--but this time it is my left side
\item So we will have to adapt.  I can't raise my arm--so no black board.
\end{itemize}
\end{itemize}
\section{From last time}
Goal, find a good fit/forecast:
\begin{displaymath}
\min E(Y_{\hbox{future}} - \hat{Y}_{\hbox{future}})^2
\end{displaymath}
The empirical trick, estimate an expectation by an average, so here is an
``achievable goal:''
\begin{displaymath}
\min (1/n)\sum_{i=1}^n(Y_i - \hat{Y}_i)^2
\end{displaymath}
Now use the sum of squares trick, (Called Patheragean theorem to those
of a historical perspective) and we have
\begin{displaymath}
\min \sum_j t^2_{X_j} + SSE
\end{displaymath}
where each $X_j$ is a possible explanitory variable.
\section{But what to actually do?}
This is useful for estimating our error--but it doesn't help us pick a
model.  So we now need the variable selection trick:
\begin{displaymath}
\hbox{estimate $\beta_i$ by} = \left\{ 
\begin{array}{l@{\quad}l}
0 & \hbox{some times}\\
\hat\beta_i & \hbox{other times}
\end{array}
\right.
\end{displaymath}
But when to do each?

\section{Variable selection}
Heuristic: If $t$ is large, then estimate $\beta$ otherwise call it
zero.  (Tukey called this s yrdyimsyt0

\end{document}
