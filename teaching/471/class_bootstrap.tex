\documentclass{article}

\usepackage{hyperref}

\begin{document}

\title{Class: bootstrap}

(\href{class_bootstrap.pdf}{pdf version})

Suggested readings
\begin{itemize}
\item \href{http://en.wikipedia.org/wiki/Bootstrapping_(statistics)}{Wiki article}.
\item \href{http://www.ats.ucla.edu/stat/R/library/bootstrap.htm}{How
to bootstrap in R}.
\item
\href{http://cran.r-project.org/doc/contrib/Fox-Companion/appendix-bootstrapping.pdf}{More
theoretical description of bootstrap} 
\item Efron came up with the idea.  See for example, Efron,
Bradley. Tibshirani, Robert J. {\em An introduction to bootstrap}.
\end{itemize}


\section{Typical data}

Suppose that when we plot $Y$ vs $x$ we get a hetroskadastic pattern?
(Think baseball example.)  
\begin{itemize}
\item Sometimes a transformation will fix it:
\begin{displaymath}
\log(Y) = \alpha + \beta \log(x) + \epsilon
\end{displaymath}
\item Sometimes doing weighted least squares will fix it:
\begin{displaymath}
Y/x = \alpha/x + \beta + \epsilon
\end{displaymath}
\item But sometimes we just don't know how to model the noise.  This
is when a bootstrap would be useful.
\end{itemize}

\section{If only data were cheap}

If we could get our hands on a new data set, we could do the
following: 
\begin{eqnarray*}
\hat\beta_1 &=& \hbox{Least squares estimate of $\beta$ on data set \#1}\\
\hat\beta_2 &=& \hbox{Least squares estimate of $\beta$ on data set \#2}\\
\hat\beta_3 &=& \hbox{Least squares estimate of $\beta$ on data set \#3}\\
\vdots &=& \vdots \\
\hat\beta_M &=& \hbox{Least squares estimate of $\beta$ on data set \#M}
\end{eqnarray*}
Now we could compute $SD(\hat\beta)$.  This would be the true standard
error of $\hat\beta$.
\begin{itemize}
\item In an ideal world, (i.e. under the normal linear model) then this
would be the number that JMP would print out as the standard error of
$\hat\beta$.
\item If $\hat\beta$ is unbiased, (i.e. nothing correlates with the
errors) then this $SD$ will tell us how far $\hat\beta$ is from the
true $\beta$.
\item Of course, if we had all this extra data, we wouldn't use it
this way, we could put all the datasets together and generate a new
and improved estimate of $\beta$.
\end{itemize}

\section{What might these data sets look like?}
\begin{itemize}
\item All these datasets are drawn from the same population
\item Whatever model generates them, will generate all the data.
\item So for any row which might be drawn as the first row in the 3rd
data set, there should be some row in the 5th dataset that looks like
it.
\item In other words, if we scrambled the rows between the various
data sets, we would get the same answer.
\item This scrambling is the idea behind the bootstrap.
\end{itemize}

\section{Bootstrap methodology}
The goal of the bootstrap is to fake a bunch of new datasets.  The
method is assume that any row in the new data set is basically like
some row that already exists in our current data set.  We just don't
know which one it is.  So we can simulate this by grabbing a row at
random.

Procedure for the bootstrap standard error:
\begin{itemize}
\item Generate a new dataset
\begin{itemize}
\item For each row in the new data set, pick a row in the real dataset
and copy it
\item Sample with replacement so rows 
\end{itemize}
\item Compute the $\hat\beta$ on the new dataset
\item Repeat this 1000s of times
\item Compute the standard deviation of all the $\hat\beta$'s.
\end{itemize}

If you are in a programming langauge, you can write the above as a
simple compute algorithm (say in R).  But what to do if you are using
JMP? 

\section{Faking bootstrap in JMP}

How many times will the first row be duplicated in the data set?  On
average the number of times, will be one.  If this weren't true, the
duplicate data set would have more rows (or fewer) than the original
dataset.  But what will the distribution?

The probability that the first row isn't used, will be
$(1-\frac{1}{n})^n$.  This is very close to $1/e$.  Likewise, the
chance that it is used exactly once, is $n$ chances each with
probability $1/n$.  So it is 
\begin{displaymath}
P(\hbox{duplicated once}) = n (1-\frac{1}{n})^{n-1}\frac{1}{n} \approx e^{-1}
\end{displaymath}
Continuing in this fashion we see that
\begin{displaymath}
P(k \hbox{ duplicates}) = {n \choose k} (1-\frac{1}{n})^{n-k}(\frac{1}{n})^k \approx e^{-1}/k!
\end{displaymath}
If you spent too much time memorizing distributions in your last
probability class, you will recognize this as a poisson distribution. 

So we want to duplicate the first row a poisson(1) number of times.  

JMP bootstrap
\begin{itemize}
\item Create a column called ``bootstrap weights''
\item Use the formula editor to insert $1/Poisson(1)$ (located inside ``random'')
\item Run a weighted regression using these weights
\item Record the estimates
\item Recalculate
\end{itemize}



\end{document}
