\documentclass[14pt]{extarticle}
\renewcommand{\baselinestretch}{1.2}
\usepackage{hyperref}
\usepackage{Sweave}
\begin{document}


\title{Class: Calibration}
\maketitle
There are several versions of notes for calibration:
\begin{itemize}
\item These notes as in a \href{class_calibration.pdf}{pdf version}.
\item The R commands from this file
\item Notes targeted at \href{calibration.pdf}{JMP}.
\end{itemize}

\section{Admistrivia}

\begin{itemize}
\item Quotes and data due
\item Homework questions?
\end{itemize}

\section{Calibration notes}

Let's first grab some data about how sales are related to how many feet of display are made.

\begin{Schunk}
\begin{Sinput}
> display <- read.table("http://www-stat.wharton.upenn.edu/~waterman/fsw/datasets/txt/Display.txt", 
+     header = TRUE)
> names(display)
\end{Sinput}
\begin{Soutput}
[1] "DisplayFeet" "Sales"      
\end{Soutput}
\end{Schunk}
Let's take a first look at the data:

\begin{Schunk}
\begin{Sinput}
> plot(display$Sales ~ display$DisplayFeet)
> regr <- lm(Sales ~ DisplayFeet, data = display)
> abline(regr$coef)
\end{Sinput}
\end{Schunk}
\includegraphics{.figures/calibration-basicplot}

Doesn't look all that good.  But, hey the $R^2 = 0.711975$
is great and the coeficients have a good fit:

% latex table generated in R 2.8.1 by xtable 1.5-6 package
% Wed Apr  7 10:27:12 2010
\begin{table}[ht]
\begin{center}
\begin{tabular}{rrrrr}
  \hline
 & Estimate & Std. Error & t value & Pr($>$$|$t$|$) \\ 
  \hline
(Intercept) & 93.0323 & 18.2278 & 5.10 & 0.0000 \\ 
  DisplayFeet & 39.7565 & 3.7695 & 10.55 & 0.0000 \\ 
   \hline
\end{tabular}
\end{center}
\end{table}
\begin{Schunk}
\begin{Sinput}
> poly <- lm(Sales ~ poly(DisplayFeet, 5), data = display)
\end{Sinput}
\end{Schunk}

\begin{Schunk}
\begin{Sinput}
> xtable(summary(poly))
\end{Sinput}
% latex table generated in R 2.8.1 by xtable 1.5-6 package
% Wed Apr  7 10:27:12 2010
\begin{table}[ht]
\begin{center}
\begin{tabular}{rrrrr}
  \hline
 & Estimate & Std. Error & t value & Pr($>$$|$t$|$) \\ 
  \hline
(Intercept) & 268.1300 & 5.5764 & 48.08 & 0.0000 \\ 
  poly(DisplayFeet, 5)1 & 544.1254 & 38.2297 & 14.23 & 0.0000 \\ 
  poly(DisplayFeet, 5)2 & -191.8805 & 38.2297 & -5.02 & 0.0000 \\ 
  poly(DisplayFeet, 5)3 & 86.7360 & 38.2297 & 2.27 & 0.0286 \\ 
  poly(DisplayFeet, 5)4 & -124.3232 & 38.2297 & -3.25 & 0.0023 \\ 
  poly(DisplayFeet, 5)5 & 7.4251 & 38.2297 & 0.19 & 0.8470 \\ 
   \hline
\end{tabular}
\end{center}
\end{table}\end{Schunk}

Compute the F by hand

\begin{Schunk}
\begin{Sinput}
> smallerR2 <- summary(regr)$r.squared
> biggerR2 <- summary(poly)$r.squared
> F <- ((biggerR2 - smallerR2)/4)/((1 - biggerR2)/poly$df.residual)
\end{Sinput}
\end{Schunk}


we can get the p-value from R:
\begin{Schunk}
\begin{Sinput}
> df(10.2, 4, poly$df.residual)
\end{Sinput}
\begin{Soutput}
[1] 7.621355e-06
\end{Soutput}
\end{Schunk}

\section{Or from R directly}

Using some magic of the anova command:

\begin{Schunk}
\begin{Sinput}
> xtable(anova(regr, poly))
\end{Sinput}
% latex table generated in R 2.8.1 by xtable 1.5-6 package
% Wed Apr  7 10:27:12 2010
\begin{table}[ht]
\begin{center}
\begin{tabular}{lrrrrrr}
  \hline
 & Res.Df & RSS & Df & Sum of Sq & F & Pr($>$F) \\ 
  \hline
1 & 45 & 119774.47 &  &  &  &  \\ 
  2 & 41 & 59921.84 & 4 & 59852.64 & 10.24 & 0.0000 \\ 
   \hline
\end{tabular}
\end{center}
\end{table}\end{Schunk}


\end{document}
