\documentclass[20pt]{extarticle} % 14, 17, 20 all exist
\usepackage{hyperref}

\usepackage[usenames]{color}\definecolor{mypurple}{rgb}{.6,.0,.5}\newcommand{\note}[1]{\noindent{\textcolor{mypurple}{\{{\bf note:} \em #1\}}}}
\newcommand{\tech}[1]{\noindent{\textcolor{red}{\{{\bf technical note:} \em #1\}}}}
\usepackage{xypic}

\renewcommand{\baselinestretch}{1.4}
\begin{document}
\title{Paranoid, huh?}
\maketitle
\href{class_paranoid.pdf}{pdf version}
\subsection*{Models: A review}
Recall our usual model;
\begin{displaymath}
(\forall i) Y_i = \alpha + \sum_j \beta_j X_{ij} + \epsilon_i
\end{displaymath}
where
\begin{displaymath}
\epsilon_i \sim_{iid} N(0,\sigma^2)
\end{displaymath}
\begin{itemize}
\item Strong stuff, lots of data aren't plausibly normal
\item Wonderful conclusions, MLE, regression, effeciency, RIC, etc,
all revolve around this model
\item Which of these statments are still true if the model is wrong?
\item This is the question of model free statistics
\item Championed by modern machine learning
\end{itemize}
\subsection*{An always true model}
The equation:
\begin{displaymath}
(\forall i) Y_i = \alpha + \sum_j \beta_j X_{ij} + \epsilon_i
\end{displaymath}
always holds for some $\epsilon_i$'s.  So if we drop our conditions on
$\epsilon$'s we have an always true model.
\begin{itemize}
\item What theorems are still true? (None as previously stated)
\item Can we replace the old theorems with new ones? (amazinly, this
is pretty much always yes)
\end{itemize}
\subsection*{Typical statements}
\begin{itemize}
\item MR. Afterdefact looks at all the data after it was collected and says, ``You know, a
regression of $Y$ on $X$ fits pretty darn well.''
\item Ms. Statisitican, looks at the data sequentially and predicts
the next observation.
\item Competive comparison: Can Ms. Statistican have almost as good as
fit as Mr. Afterdefact?
\end{itemize}
\subsection*{Model based on money}
One of your close friends goes to Stanford business school and the
other goes to Chicago.  Each claims they are going to kick the markets
butt over the next 50 years.  They have very different aproaches.
Your goal is 
\begin{itemize}
\item (If you are in Wharton) Defend our reputation!
\item (If you aren't in Wharton) Skip finance and still get rich.
\end{itemize}

\subsection*{Choosing between two investments}

\begin{itemize}
\item {\bf Problem:} Suppose I have two friends who are hot-shot financial
wizards.  They come from different schools of thought and both believe
the other to be totally clueless.  So, in fact, I have one friend
who is a financial wizard, and one friend who is an impostor.  But,
I don't know which is which!

\item {\bf Goal:} I want to get as rich as my financial wizard
friend--whichever that empirically turns out to be.

\item {\bf No assumptions:} I will not make any probabilistic assumptions.
\end{itemize}

\subsection*{Setup for finance}

\begin{itemize}
\item {\bf Notation:}
  \begin{itemize}
  \item  $A_t$ is the wealth of my first friend at time $t$
  \item  $B_t$ is the wealth of my second friend 
  \item  $C_t$ is my  wealth 
  \item  $w_t$ is fraction of my wealth A invests for me at time $t$
  \end{itemize}
\item $A_0 = B_0 = C_0 = 1$
\item {\bf Returns:}
  \begin{itemize}
  \item  $R^A_t  =  A_{t}/A_{t-1}$ is A's return 
  \item  $R^B_t  =  B_{t}/B_{t-1}$ is B's return
  \item $R^C_t  =  w_{t-1} R^A_t + (1-w_{t-1}) R^B_t$ is my return
  \end{itemize}
  
\item  {\bf Goal:}
$$C_{t}\cong  \max(A_{t},B_{t})$$
All three are growing ``exponentially,'' so use $\log(C_t)$ instead.
Now growing ``linearilly.'' So use 
    \begin{displaymath}
      \frac{\log(C_{t})}{t} \cong \max(\frac{\log(A_{t})}{t},
      \frac{\log(B_{t})}{t})
    \end{displaymath}
    as our goal.
\end{itemize}

\subsection*{Invest with my best friend}

\begin{itemize}
\item {\bf Scheme:} Whichever friend is currently wealthier is ``more
  likely'' to be the financial wizard.  So have her invest all my 
  wealth:
\begin{displaymath}
  w_t = \left\{
  {
    \begin{array}{l@{\qquad\hbox{if}\quad}l}
      1 & A_{t-1} \ge B_{t-1} \\
      0 & A_{t-1} < B_{t-1}
    \end{array}
    }
  \right.
\end{displaymath}

\item {\bf Evil data:} Will this scheme always work? No, the following
  example shows this:

\begin{table}[htbp]
  \begin{center}
    \leavevmode
    \begin{tabular}{r|c|c|c|c|c|c|c|c|c|c}
      time  & $0$ & $1$ & $2$ & $3$ & $4$ & $5$ & $6$ & $7$  & $8$ &$\cdots$
      \\ \hline 
      $A_t$ & $1$ & $1$ & $2$ & $2$ & $4$ & $4$ & $8$ & $8$  & $16$
      &$\cdots$ \\ 
      $B_t$ & $1$ & $2$ & $2$ & $4$ & $4$ & $8$ & $8$ & $16$ & $16$
      &$\cdots$ \\ \hline
      $w_t$ & $1$ & $0$ & $1$ & $0$ & $1$ & $0$ & $1$ & $0$  & $1$
      &$\cdots$  \\
      $C_t$ & $1$ & $1$ & $1$ & $1$ & $1$ & $1$ & $1$ & $1$  & $1$
      &$\cdots$  \\
    \end{tabular}
  \end{center}
\end{table}%

\item {\bf growth rates:} 
  \begin{itemize}
  \item A's growth rate: $ \ln(2)/2$
  \item B's growth rate: $ \ln(2)/2$
  \item C's growth rate: $ 0$
  \end{itemize}
\end{itemize}

\subsection*{Equal weight}

\begin{itemize}
\item {\bf Scheme:} Always have each friend invest 1/2 of my wealth:
\begin{displaymath}
  w_t = 1/2
\end{displaymath}


\item {\bf Evil data:} This scheme isn't sensitive enough.  If one
  friend is doing better, it won't notice.  This motivates the
  following bad data:


\begin{center}
  \begin{tabular}{r|c|c|c|c|c|c|c|c}
    time  & $0$ & $1$ & $2$ & $3$ & $4$ & $5$ & $6$ & $\cdots$   \\ \hline 
    $A_t$ & $1$ & $2$ & $4$ & $8$ & $16$ & $32$ & $64$ & $\cdots$  \\ 
    $B_t$ & $1$ & $1$ & $1$ & $1$ & $1$ & $1$ & $1$ & $\cdots$   \\ \hline
    $w_t$ & $1/2$ & $1/2$ & $1/2$ & $1/2$ & $1/2$ & $1/2$ & $1/2$  & $\cdots$  \\  
    $C_t$ & $1$ & $1.5$ & $2.25$ & $3.4$ & $5.1$ & $7.6$ & $11.4$
    & $\cdots$  \\ 
  \end{tabular}
\end{center}

\item {\bf growth rates:} 
  \begin{itemize}
  \item A's growth rate: $ \ln(2)/2$
  \item B's growth rate: $ 0$
  \item C's growth rate: $ \ln(1.5)/2$
  \end{itemize}
\end{itemize}

\subsection*{Value weighted}

\begin{itemize}

\item {\bf Scheme:} Have each invest in proportion to how well the have
  done so far.
  \begin{displaymath}
    w_{t}=\frac{A_{t-1}}{A_{t-1}+B_{t-1}}
  \end{displaymath}

\item {\bf No evil data exist:}
\begin{center}
  \renewcommand{\arraystretch}{1.5} 
  \begin{tabular}{r|c|c|c|c|c}
    time  & $0$ & $1$   & $2$   & $\ldots$ & $i$   \\ \hline 
    $A_t$ & $1$ & $A_1$ & $A_2$ & $\ldots$ & $A_i$ \\
    $B_t$ & $1$ & $B_1$ & $B_2$ & $\ldots$ & $B_i$ \\ \hline
    $w_t$ & $1/2$ & $\frac{A_1}{A_1 +B_1}$ & $\frac{A_2}{A_2+B_2}$ & $\ldots $ &
    $\frac{A_i}{A_i+ B_i}$ \\
    $C_t$ & $1$ & $\frac{A_1+B_1}{2}$ & $\frac{A_2+B_2}{2}$ & $\ldots$
    & $\frac{A_i+B_i}{2}$
  \end{tabular}
  \renewcommand{\arraystretch}{1} 
\end{center}

\item {\bf Growth rate of C:} 
  \begin{displaymath}
    \frac{\ln(C_t)}{t} = \frac{\ln(A_t/2+B_t/2)}{t} \ge
    \max\{ \frac{\ln(A_t)}{t},\frac{\ln(B_t)}{t} \} - \frac{\ln(2)}{t} 
  \end{displaymath}
  In particular:
  \begin{displaymath}
    \lim_{t \to \infty} \frac{\ln(C_t)}{t} - \max\{
    \frac{\ln(A_t)}{t},\frac{\ln(B_t)}{t} \} = 0
  \end{displaymath}
\end{itemize}
\subsection*{Theorem}
\begin{itemize}
\item Without any assumptions on the returns
\item We can statistically combine two investments
\item such that our growth rate is as good as the better of these two
growth rates
\end{itemize}
\subsection*{What is the trick?}
\begin{itemize}
\item It is all in the ``log''
\item We will have 1/2 the wealth of our richer friend
\item log(2 Billion) $\approx$ log(1 billion)
\end{itemize}

\subsection*{But this trick works in statistics too}

\begin{itemize}
\item It works for regression, and other estimation problems
\item It works for creating a good definition of probability (we'll
look at this next time)
\item It turns out to be equivalent to our usual aproach--so we
inherently already are using the trick
\end{itemize}

\end{document}
