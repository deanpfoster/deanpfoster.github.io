\documentclass[12pt,twocolumn]{article}
\usepackage{simplemargins}
%\usepackage{wasysym}  % trying to get the /mars and \venus working
\setleftmargin{.5in}
\setrightmargin{.5in}
\settopmargin{0.5in}
\setbottommargin{0.in}
\renewcommand{\baselinestretch}{1.07}
\begin{document}
\pagestyle{empty}
\title{200th Darwin day: Heterocarpy in daisies}
\maketitle
\thispagestyle{empty}
\pagestyle{empty}

\subsection*{1: Review of Evolution}
\begin{itemize}
\item Genes are in it for themselves: 
\begin{itemize}
\item (Read the original Darwin, {\em
The Origin of Species} or a modern version Dawkins, {\em The selfish
gene} or my favorite {\em The extended phenotype}, or a very readable
version, Dennet {\em Darwin's Dangerous Idea}.) 
\end{itemize}
\item Genes don't work for the species
\item Evolution is:
\begin{itemize}
\item Variation in genes
\item Selection of genes
\item Reproduction of genes
\end{itemize}
\end{itemize}

\subsection*{2: Problem, Why have sex?}
\begin{itemize}
\item Count number of genes in great-grandchildren
\begin{itemize}
\item Sex: $2 * 2 * 2$ children, but $1/2 * 1/2 * 1/2$ genes, so net 1
gene in great grandchildren
\item asexual: $2 * 2 * 2$ children, with $1 * 1 * 1$ genes, so net 16
copies. 
\end{itemize}
\item Males are useless (called 2 fold cost of sex)
\item Example: To take over human species (about 1 billion people) would take
30 doublings, or about 600 years.
\item So, why have sex?
\end{itemize}

\newpage
\subsection*{3: Previous stories}

\begin{itemize}
\item Species go extinct if they don't have sex
\begin{itemize}
\item Oops--genes don't care
\item Oops--dandelions seem to be doing OK
\end{itemize}
\item Asexual children would be genetically defective
\begin{itemize}
\item No empirical evidence since not tried in most species
\item In equilbrium this wouldn't be true
\end{itemize}
\item ``Red queen'' Out-crossed children are twice as effective at
fighting desease as clones.  
\begin{itemize}
\item Best theory so far
\item Suggests Rip Van Winkle would die of a cold when he woke up. 
\end{itemize}
\end{itemize}

\subsection*{4: Model organism}

\begin{itemize}
\item Need a species that actively does both (sex and asexual
reproduction) 
\item Needs it to be in equilbrium
\item Daises will be our working example.
\begin{itemize}
\item When do they do sex?  
\item When do the self?
\item Why?
\end{itemize}
\end{itemize}
\newpage
\subsection*{5: Model}
\fbox{\mbox{
Survival prob $\approx k e^{\displaystyle \{\vec{g}\vec{\cal E}-
  \vec{g}{\bf A}\vec{g}\}}$}}
\begin{itemize}
\item definitions
   \begin{itemize}
     \item $\vec{g}$ describes the genes themselves
     \item $\vec{\cal E}$ describes the environment
     \item ${\bf A}$ forces an optimum for the genes
     \item $k$ scaling 
   \end{itemize}
\item $\vec{\cal E}$ is random 
\end{itemize}

\subsection*{6: Economic model}
\begin{itemize}
\item Diminishing marginal returns for each type
\item Different types don't have this suppression
\end{itemize}

\subsection*{6: Strategies}
\begin{itemize}
\item The generalist: Matches the typical environment
\item The adaptive: Changes behaviour for different environments
\item The hopeful: Tries different strategies--hoping one will work
\end{itemize}

\subsection*{8: Galton and regression}
\begin{itemize}
\item Like height / intelligence
\item Naive: E(Child) $= .5 M+ .5 F $
\item Better: 
\begin{displaymath}
E(\vec{g}\vec{\cal E}|M,F) = .25 M + .25 F + .5(\hbox{``other''})
\end{displaymath}
\item More important: lots of noise!
\end{itemize}
\newpage
\subsection*{9: Daisies}
\begin{itemize}
\item seed size, shape, etc determine distribution
\begin{itemize}
\item small seeds with a pappus travel far
\item heavy seeds fall near by
\item seeds surrounded by fruit do both
\end{itemize}
\end{itemize}
\begin{tabular}{r|c|c}
              & Sex      &    Self    \\ \hline
Large seeds   &          &            \\ \hline
small seeds   &          &            \\
\end{tabular}
\begin{itemize}
\item many possible patterns: I.e. most plants only produce one sort
of seed: acorn is large, dandelion is small 
\item Daisies are heterocarpy: i.e. both sizes
\item Naive story: large = self, small = sex
\item Data: large = sex, small = self
\end{itemize}

\subsection*{10: Applying model to daisies}
\begin{itemize}
\item Competition at current location, so variance is good
\item No competition at ``far'' location, so variance is bad
\item If $\sigma^2_{\hbox{sex}} \ge \sigma^2_{\hbox{self}} + 2\log 2$
and a competitive environment, then sex is better than self
\end{itemize}

%\end{onecolumn}
\end{document}
