\documentclass[12pt]{article}

\usepackage{hyperref}

\begin{document}

\title{Class: Residuals}

(\href{class_residuals.pdf}{pdf version})

\section{Visual tests for residuals}

Most testing of residuals is done by looking at plots.  This is fast
and intuitive.  For the standard linear model:
\begin{displaymath}
Y = \alpha + \beta X + \epsilon
\end{displaymath}
the residuals are created by looking at
\begin{displaymath}
\hat\epsilon = Y - (\hat\alpha + \hat\beta X)
\end{displaymath}
or more succinctly
\begin{displaymath}
\hat\epsilon = Y - \hat Y
\end{displaymath}
The standard plots to look at are ones like the following:
\begin{enumerate}
\item $\hat\epsilon$ vs $X^2$, looking for curvature
\item $\hat\epsilon^2$ vs $X$, looking for heteroskadasticity
\item $\hat\epsilon_t$ vs $\hat\epsilon_{t-1}$ auto correlation
\item $\hat\epsilon_t$ vs $\hat\epsilon_{t-1}^2$ leverage
\item $\hat\epsilon_t^2$ vs $\hat\epsilon_{t-1}^2$ hetroskadasticty / ARCH like effects
\item $\hat\epsilon$ vs $Z$ for any left out $Z$
\item $\hat\epsilon$ vs $\hat Y$ for multiple regression (otherwise
you have seen this plot already when you plotted vs $X$).
\item any thing else involving $\hat\epsilon$!
\end{enumerate}
The problem with these visual tests is two fold, 
\begin{itemize}
\item How do you tell if the test fails or not?
\item How many test are you actually doing in your head when you look
at a plot?
\end{itemize}
Both of these are addressed in the following formal approach.

\section{Formal Test of residuals}

The idea of having a formal testing proceedure is that it helps tell
 when something real is actually wrong.  You don't actually have to
 follow this proceedure exactly (unless you are programming a computer
 to do it for you).  But it helps address the problem of finding too
many problems that you might run into if you just keep looking and
looking and looking until you find something.

The idea is to make every possible test for residuals into an actual
t-test.  So instead of looking at whether there is curvature in the
$\hat\epsilon$ vs $\hat Y$ plot, instead, just fit a polynomial of
$\hat\epsilon$ on $\hat Y$ and see if the quadradic term is
significant.  If it is--you have found curvature, if not, well you
haven't proven there isn't curvature, but at least it won't be an
obvious mistake.

So we want to translate every visual test we could do into a t-test.
Most of these are fairly easy to do:
\begin{itemize}
\item To test $\hat\epsilon$ vs $X^2$ for curvature, test the
slope in $\hat\epsilon$ vs $(X - \bar X)^2$.  (The trick of
subtracting out the mean saves on having to do the polynomial
regression itself.)
\item To test $\hat\epsilon^2$ vs $X$ for heteroskadasticity, look at a
plot of $\hat\epsilon^2$ vs $X$ and see if the slope is non-zero.
\item To test $\hat\epsilon_t$ vs $\hat\epsilon_{t-1}$ for auto
correlation, simply look at t-statistic for the slope
\item To test $\hat\epsilon_t$ vs $\hat\epsilon_{t-1}^2$ for
 leverage, again look at the slope
\item To test $\hat\epsilon_t^2$ vs $\hat\epsilon_{t-1}^2$ for
hetroskadasticty /ARCH like effects, look at the slope.
\item To test $\hat\epsilon$ vs $Z$ for any left out $Z$, add $Z$ to
your regression and look at the t-statistic.  (NOTE: you get a bad
answer if you save your residuals and plot them vs $Z$.  The problem
is colinearity.)
\item To test $\hat\epsilon$ vs $\hat Y$ in a multiple regression,
save the residuals, and then use ``Fit Y by X'' to see if a polynomial
fits better.  Each degree of the polynomial you look at costs you one
test! 
\item Any thing else involving $\hat\epsilon$, just make up a test
that seems to capture what you are looking for in a hypothesis test.
\end{itemize}

\section{So many tests?  Won't one of the always reject?}

If you do 20 test at .05, you will probably get a rejection by chance
alone.  If you do 100 tests at .05, you will certainly get a rejection
by chance alone.  So how do you do all the above tests without finding
something spurious?

The trick is to do all the tests at a much more stringent p-value than
just using .05.  In class I suggested testing the first one at
$(1/20)^2$, and the second at $(1/21)^2$, etc.  This means that if you
did 100 tests, the entire chance of rejecting something by chance
alone would be:
\begin{displaymath}
1/20^2 + 1/21^2 + 1/22^2 + 1/23^2 \cdots + 1/120^2 < .05
\end{displaymath}
What is cool about this sequence if that even if you end up doing a
million tests, 
\begin{displaymath}
1/20^2 + 1/21^2 + 1/22^2 + 1/23^2 \cdots + 1/1000020^2 < .05
\end{displaymath}
it is still less than .05.  


\end{document}
