\documentclass{article}
\usepackage{hyperref}
\begin{document}

\title{Class: Doglegs}

(\href{class_doglegs.pdf}{pdf version})

\title{Fitting using broken sticks}

\section{Review of the standard linear model}

The standard linear regression model is:
\begin{displaymath}
Y_i = \alpha + \beta x_i + \epsilon_i \quad \epsilon_i \sim_{iid}
N(0,\sigma^2)
\end{displaymath}
You will see this equation written in almost any research paper which
uses data.  The names are often changed, but it is there somewhere.
For example, it is basically equation 2.17 in Berndt of the reading.
The entire chapter is designed to motivate that one equation.

Let's break it down into pieces.
\begin{itemize}
\item The fit:
\begin{displaymath}
Y_i = \underbrace{\alpha + \beta x_i}_{\hbox{the fit}} + \epsilon_i \quad \epsilon_i \sim_{iid}
N(0,\sigma^2)
\end{displaymath}
the most fun part is ``the fit''.  It describes the relationship
between $x$ and $Y$.  This version describes a linear relationship. 

\item Residuals / errors:
\begin{displaymath}
Y_i = \alpha + \beta x_i + \underbrace{\epsilon_i \quad \epsilon_i
\sim_{iid} N(0,\sigma^2)}_{\hbox{The residuals}}
\end{displaymath}
The residuals (aka errors) themselves.  Describing them,
looking at them, investigating them is the primary activity of a
statistician.  It is all about error!
\begin{itemize}
\item The ``i.d.'':  The i.i.d. part can be broken into two pieces, ``i.''
and ``i.d.''  The easier is the identically distributed.  It means
each error looks like any other error.

\item The ``i.'': The first ``i'' in IID is for independence.  We will
spend an entire class on this piece.  It is the most important
assumption in the entire model.

\item the ``N'': Means normal.  Look at a q-q plot to check it.  It is
easy to check (hence we cover it in intro classes).  We won't discuss
it here since I assume you already know how to check it.

\item Style: iid = i.i.d. = IID = I.I.D. = independent and identically
distributed.  It is often even left off entirely since it is always
assumed. 
\end{itemize}

\item $Y$ is upper case, $x$ is lower case: Recall from probability that
random variables are often writtten as upper case letters.  This is
why $Y$ is written as an upper case--it is random.  The $x$ are
thought of as inputs, and hence not random.

\item $i$ is the row index.  We might even say how many rows we have
by the cryptic addition to the equation:
\begin{displaymath}
(i = 1,\ldots,n) \quad
Y_i = \alpha + \beta x_i + \epsilon_i \quad \epsilon_i \sim_{iid}
N(0,\sigma^2)
\end{displaymath}

\section{Is linear good enough?}

Taylor (\href{http://en.wikipedia.org/wiki/Brook_Taylor}{wiki}) tells
us that ``everything'' can be approximated by a linear equation.  So
if there is a true relationship between $Y$ and $x$ that is
non-linear, then we could say 
\begin{displaymath}
E(Y|x) = f(x)
\end{displaymath}
(This is yet another cryptic for of our main equation.  It could be
written as $Y = f(x) + \epsilon$ to make it look more like our
previous equation.)  So Taylor's theorem says that
\begin{displaymath}
E(Y|x) \approx \alpha + \beta x
\end{displaymath}
and even tells us what $\alpha$ and $\beta$ are.  

In this more enlighted age, we no longer believe all functions are
basically polynomials on steriods.  They can be very weird.  So
Taylors naive idea of all functions being linear is no longer held
true.  But not to worry, in the modern age we can quote Littlewood who
said, that {\it almost} all functions are {\it almost} locally
linear. 

So what isn't linear?  Jumps, and bends.

\section{Create your own fit}

In most setting, your best bet is still to keep things smooth by using
polynomials.  This is done in JMP by asking it to fit a polynomial for
you.  Hopefully you have seen this already.  


\end{document}
