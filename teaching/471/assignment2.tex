\documentclass[11pt]{article}

\begin{document}
\title{Assignment 2: Computer exercises}

These exercises are to show you computer techniques.  So do simple
 prints to confirm that they worked for you.

\begin{enumerate}
\item Dogleg construction and analysis:
\begin{enumerate}
\item Spend 20 minutes trying to find a new historical temperature
series.  (If you can't find one, use New Hampshire which was discussed
in class.).   Fit it using a dogleg at 1900 and print it.
\item Create doglegs for 1800 and 1700 also (and 1600 if you have
it).  Fit a multiple regression to these.  Find a smooth that looks
close to your fit.  Print both predictions on one graph. 
\item Fit simple regressions to all of your dog legs and save the
predictions.  Compute an average prediction (pred1 + pred2 + pred3 +
pred4)/4.  Find a smooth that looks close to the average prediction.
Print both predictions on the same graph.
\item Fit a 3rd degree polynomial to your data.  As usual, find a
smooth that looks similar to your 3rd degree polynomial.  Print both
predictions. 
\end{enumerate}

\item Predictions: Insert an artifical row with a date of 2100 into
your historical data table.  Leave the temperature field blank.  For
each of the 4 models you created above, predict the temperature at
2100.   For the first, second and 4th model, you should be able to
also save a prediction interval.  Which of these 3 intervals makes the
most dramatic claim about temperature?

\item Residuals: Create residuals from your favorite model of
temperature above.  
\begin{enumerate}
\item Do the following checks (use regression):
\begin{itemize}
\item residuals vs time squared
\item residuals squared vs time
\item residuals vs previous residuals
\item residuals squared vs previous residuals squared
\item residuals vs any other columns you might have in your data table
\item etc
\end{itemize}
\item Using the $1/20^2, 1/21^2, 1/22^2,\ldots$ rule, which of the
above tests (if any) show problems with the data?
\item If you have a problem, do we know how to fix it yet?
\end{enumerate}

\item Hetroskadasticity: Do a regression of income vs ERA.
\begin{enumerate}
\item Simple regression:
\begin{itemize}
\item By eye, you can see the problem of hetroskadasticity.  Make the
appropiate plot and confirm that it is significant.  Clearly if it
were the first test you were going to do, it would be significant.  If
it were the second one, it would also be significant.  How many OTHER
tests would you have to do before this test, so that it would no
longer be considered significant?
\item For a player who has an ERA of .35, how many more dollars would
he earn if he increased it to .36?  Give a confidence interval based
on your simple regression.
\end{itemize}
\item log-log:  Do a model of log(income) vs log(ERA):
\begin{itemize}
\item  Save the residuals.  Check if they are hetroskadastic.
\item For a player who has an ERA of .35, how many more dollars would
he earn if he increased it to .36?  Give a confidence interval. (HINT:
this isn't easy--it will take a bit of calculation.  Ask in class.)
\end{itemize}
\item Weighted least squares:  Do a model of income/ERA vs 1/ERA:
\begin{itemize}
\item  Save the residuals.  Check if they are hetroskadastic.
\item Using the weighted regression feature in JMP, run this
regression as a weighted regression.  You should get the exact same
coefficients and standard errors.  If not, try it again using
$1/ERA^2$ as your weighting instead of $1/ERA$.
\item For a player who has an ERA of .35, how many more dollars would
he earn if he increased it to .36?  Give a confidence interval. 
\end{itemize}
\item Bootstrap a simple regression.
\begin{itemize}
\item Do at least 10 bootstrap recalcuations of your estimate of
slope.  From these compute a standard deviation.
\item For a player who has an ERA of .35, how many more dollars would
he earn if he increased it to .36?  Give a confidence interval based
on your bootstrap standard deviation.
\end{itemize}
\end{enumerate}
\end{enumerate}
\end{document}
