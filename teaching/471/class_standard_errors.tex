\documentclass{article} % -*- auctex -*-

\usepackage{hyperref}

\begin{document}
\title{Class: standard errors}

%\href{http://twitter.com/home}{Twitter}.

(\href{class_standard_errors.pdf}{pdf version})


\section{Status so far}

The model
\begin{displaymath}
Y_i = \alpha + \beta x_i + \epsilon_i 
\end{displaymath}
where $\epsilon_i$ are iid and
\begin{displaymath}
\epsilon_i \sim N(0,\sigma^2).
\end{displaymath}

\begin{itemize}
\item First we discussed fitting ($\alpha + \beta x_i$)
\item Then we discussed the residuals
\item Now we want to discuss how to estimate the error in $\hat\beta$
\end{itemize}

\section{Why we care}

If the normal linear model holds, we know that $\hat\beta$ has close
to a normal distribution.  In particular,
$\frac{\hat\beta}{SE(\hat\beta)}$ is a t-distribution.  We often want
to know how accurately we know $\beta$ purely for its own sake.  For
example, if our model is $Y = $ sales, and $X = $ advertisments, then
$\beta = $sales/ad.  So if we know that each sale generates \$10
profit, and each add costs \$1 (think web based advertisements) then
we need $\beta > .1$ before we make more money in sales than we spent
in advertisements.  

We can also use $\hat\beta$ to make predictions:
\begin{displaymath}
\hat{Y} = \hat\alpha + \hat\beta x_i
\end{displaymath}
 So before we can now how accurate a forecast is, we need to know how
accurate $\hat\beta$ is.

Either of these require knowing that $\hat\beta$ is a good estimate of
$\beta$ and exactly how good an estimate of it it actually is.  So we
need the standard error for $\hat\beta$.

\section{Hetroskadasticity}

One problem that we can fix is that of hetroskadasticity.  Suppose
that we are regression salary (Y) on runs (X).  Then we might expect
that 
\begin{displaymath}
Y = \alpha + \beta x
\end{displaymath}
will display hetroskadastic errors.  In particular, we might expect
that the errors grow with $x$.  

\paragraph{log-log model}  If we use logs, we can consider the model
\begin{displaymath}
\log(Y) = \alpha + \beta \log(x) + \epsilon
\end{displaymath}
In this model, we now expect the errors to be homoskadastic.  
\begin{eqnarray*}
\log(Y) & = & \alpha + \beta \log(x) + \epsilon\\
e^{\log(Y)} & = & e^{\alpha + \beta \log(x) + \epsilon}\\
Y & = & e^{\alpha} e^{\beta \log(x)} e^{\epsilon}\\
Y & = & e^{\alpha} (e^{\log(x)})^\beta  e^{\epsilon}\\
Y & = & k x^\beta  e^{\epsilon}\\
\end{eqnarray*}
where we inserted a $k$ for $e^{\alpha}$ so the equation looked
prettier. 

The problem with this analysis is that we can no longer address the
question, ``How much is a hit worth?''  Instead, we can say, ``How
many log(dollars) is a log(hit) worth?''  If this sounds resonable to
you, then you are definitely an economists:  A slope between
log(Y) and log(x) is called the elasticity.

\paragraph{Weighted least squares:} Alternatively, we can do weighted
least squares.  If we believe that the errors scale with $x$, then we
could write our model precisely as:
\begin{displaymath}
Y = \alpha + \beta x + x \epsilon
\end{displaymath}
So when $x$ is large, the errors are large and when $x$ is small the
errors are small.  This would show up nicely in a plot of $\epsilon^2$
vs $x$.  

We can modify our equation by dividing both sides by $x$:
\begin{displaymath}
Y/x = \alpha/x + \beta +  \epsilon
\end{displaymath}
Now if we do a regression of $Y/x$ on $1/x$ we have a homoskadastic
regression.  This methodology is called weighted least squares.  The
weights in this case are $1/x$.

There are two ways of doing a weighted least squares. 
\begin{itemize}
\item  First, you can
ask JMP to simply use a column of $1/x$ as weights.  This is nice
since the equation it generates will still use $\alpha$ and $\beta$ as
we have been using them in the equations already.  
\item Second, you can create the two new variables $Y/x$ and $1/x$ and
do the regression yourself.  This allows you control and will work in
any regression package (i.e. R or even excel).  But you then have to
interpret the slope and intercept carefully.
\end{itemize}


\end{document}
